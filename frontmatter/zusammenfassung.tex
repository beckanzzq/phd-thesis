\chapter*{Zusammenfassung}

\begin{otherlanguage}{ngerman}

Die vorliegende Arbeit beschäftigt sich mit der Realisierung und der Untersuchung verschiedener
Modellsysteme im Rahmen der ultrakalten dipolaren Quantengase.
Im Mittelpunkt stehen dabei Quantenzustände mit topologisch nichttrivialen Eigenschaften.
Ein prominentes Beispiel für einen solchen Zustand zeigt sich im Quanten-Hall Effekt.
Die exakte und robuste Quantisierung der Hall-Leitfähigkeit kann dabei durch das Auftreten einer topologischen Invarianten verstanden werden.
Traditionell tritt der Quanten-Hall Effekt in zweidimensionalen Elektronengasen bei extrem tiefen Temperaturen und hohen Magnetfeldern auf.
Besonders im Bereich des fraktionalen Quanten-Hall Effekts gibt es ...
Seit einigen Jahren sind die Experimente auf dem Gebiet der ultrakalten Quantengase soweit fortgeschritten, ..

Ein Abschnitt dieser Arbeit beschäftigt sich daher mit der Realisierung des Quanten-Hall Effekts in ultrakalten Quantengases.
Im Laufe der Jahre wurden verschiedene Wege vorgeschlagen, den Effekt eines Magnetfelds auf Ladungsträger mit elektrisch neutralen Atomen zu simulieren.
Eine Methode bedient sich dabei einer exakten Analogie zwischen geladenen Teilchen im Magnetfeld und neutralen Teilchen in einem rotierenden System.
Die Rotationsfrequenz übernimmt dabei die Rolle des Magnetfeldes.
Die Lorentzkraft stimmt in diesem Bild beispielsweise mit der Corioliskraft überein.
Um das Quantengas in Rotation zu versetzten, verwenden wir den Relaxierungsmechanismus in Systemen mit Dipol-Dipol Wechselwirkung.
Dabei wird interner Drehimpuls der Atome in eine externe Rotation umgewandelt.

Über diesen Mechanismus lässt sich nicht die Drehfrequenz des Systems steuern, sondern direkt der Gesamtdrehimpuls.
Dies umgeht eine ...

Der zweite große Teil dieser Arbeit beschäftigt sich auf eine andere Weise mit topologischen Quantenzuständen.
Wie Haldane 1988 zeigte, lässt sich die quantisierte Hall-Leitfähigkeit und die damit verbundene topologische Eigenschaft auch ohne die Präsenz eines Magnetfeldes realisieren.
Er betrachtete ein Modell auf dem hexagonalen Gitter, wobei die Zeitumkehrinvarianz durch das Auftreten von komplexen Tunnelraten gebrochen wird.

In unserem Modell gehen wir von dipolaren Teilchen, beispielsweise polaren Molekülen oder Rydberg Atomen, aus, die in einem zweidimensionalen Gitter fixiert sind.
Wir sind an der Dynamik der internen Anregungen dieser Dipole interessiert, die durch die Dipol-Dipol Wechselwirkung getrieben wird.
Insbesondere können diese Anregungen zwischen verschiedenen Teilchen ausgetauscht werden, und sich damit ähnlich wie tunnelnde Elektronen verhalten.
Betrachtet man zwei verschiedene Anregungen mit unterschiedlichem internen Drehimpuls, dann können diese ineinander umgewandelt werden.
Dabei tritt ein komplexer Faktor auf, der durch die Gesamtdrehimpuls-Erhaltung der Dipol-Dipol Wechselwirkung kommt.



\end{otherlanguage}
