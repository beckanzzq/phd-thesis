\chapter*{Zusammenfassung}

\begin{otherlanguage}{ngerman}

Die vorliegende Arbeit beschäftigt sich mit der Untersuchung verschiedener
Modellsysteme im Rahmen der ultrakalten Quantengase.
Im Mittelpunkt stehen dabei neuartige Verfahren um Quantenzustände mit topologisch nichttrivialen Eigenschaften mittels dipolaren Wechselwirkungen zu realisieren.
Ein berühmtes Beispiel für einen Zustand mit topologischen Eigenschaften zeigt sich im Quanten-Hall Effekt.
Die exakte Quantisierung der Hall-Leitfähigkeit kann durch das Auftreten einer topologischen Invarianten verstanden werden.
Die Robustheit physikalischer Effekte gegenüber äußeren Störungen macht topologische Materialien dabei interessant für Anwendungen.
Entdeckt wurde der Quanten-Hall Effekt in zweidimensionalen Elektronengasen bei extrem tiefen Temperaturen und hohen Magnetfeldern.
Die schwierigen experimentellen Bedingungen, sowie eine Reihe offener Fragen, besonders im Bereich des fraktionalen Quanten-Hall Effekts, motivieren daher die Suche nach alternativen Systemen.

Seit einigen Jahren sind Experimente auf dem Gebiet der ultrakalten Quantengase so weit fortgeschritten, dass routinemäßig neuartige Modellsysteme simuliert werden können.
Ein Abschnitt dieser Arbeit beschäftigt sich mit der Realisierung des Quanten-Hall Effekts in ultrakalten Gasen.
Ein Problem besteht darin, den Effekt des Magnetfelds auf Elektronen mit elektrisch neutralen Atomen zu simulieren.
Eine mögliche Lösung, die auf Ideen von Larmor zurückgeht, bedient sich einer exakten Analogie zwischen geladenen Teilchen im Magnetfeld und neutralen Teilchen in einem rotierenden System, wobei die Rotationsfrequenz die Rolle des Magnetfeldes übernimmt.
Die Corioliskraft im rotierenden System verhält sich dabei beispielsweise wie die Lorentzkraft im Magnetfeld.
Unser Ansatz besteht darin, die Relaxierung in dipolaren Systemen zu nutzen, um das zweidimensionale Quantengas in Rotation zu versetzten.
Dabei wird der interne Drehimpuls der Atome durch die Dipol-Dipol Wechselwirkung in eine externe Rotation umgewandelt.
Um den Vorgang mehrmals zu wiederholen, kann der interne Zustand anschließend durch ein externes Magnetfeld zurückgesetzt werden.
Der Vorteil dieser Methode besteht darin, dass nicht die Rotationsfrequenz des Systems gesteuert wird, sondern direkt der Gesamtdrehimpuls.
Hierdurch kann eine intrinsische Instabilität umgangen werden, die auftritt, wenn die Rotationsfrequenz mit der Fallenfrequenz vergleichbar wird.
Bei Kenntnis der genauen Atomzahl können mit dieser Methode dann bestimmte Quanten-Hall Zustände realisiert werden, da deren Gesamtdrehimpuls bekannt ist.
Weiterhin untersuchen wir den Einfluss der Wechselwirkung im Rahmen einer vollständigen numerischen Simulation und studieren die dipolar wechselwirkenden Zustände bei fraktionaler Füllung.

Der zweite große Teil dieser Arbeit beschäftigt sich mit dipolaren Spin-Systemen und topologischen Bandstrukturen.
Wir gehen dabei von einer vorgegebenen zweidimensionalen Gitterstruktur aus, auf deren Gitterplätzen sich einzelne fest angebrachte Dipole in Form von polaren Molekülen oder Rydberg Atomen befinden.
Wir sind an der Dynamik der Anregungen dieser Dipole interessiert, die durch die Dipol-Dipol Wechselwirkung getrieben wird.
Insbesondere können diese Anregungen zwischen verschiedenen Dipolen ausgetauscht werden.
Damit verhalten sie sich ähnlich wie tunnelnde Elektronen in einem Ionengitter, wobei die Prozesse jedoch aufgrund der Dipol-Dipol Wechselwirkung langreichweitig sind.
Dies führt in zwei Dimensionen zu Veränderungen bei kleinen Impulsen.
Wir untersuchen dipolare Spin-Systeme im Rahmen der Spinwellen-Theorie, die unter anderem eine spontan gebrochene kontinuierliche Symmetrie bei endlichen Temperaturen vorhersagt.

Des Weiteren zeigen wir, dass in dipolaren Systemen topologische Bandstrukturen realisiert werden können.
Betrachtet man zwei verschiedene Anregungen mit unterschiedlichem internen Drehimpuls, dann können diese über die dipolare Wechselwirkung ineinander umgewandelt werden.
Dabei tritt ein komplexer Faktor auf, welcher der Gesamtdrehimpuls-Erhaltung Rechnung trägt.
Diese Spin-Bahn Kopplung kann dann zu nichttrivialen Phasen in Tunnelprozessen auf geschlossenen Wegen führen.
Das entspricht aber gerade dem Effekt eines Magnetfeldes auf ein geladenes Teilchen, wobei die Phase den magnetischen Fluss in Einheiten des Flussquants angibt.
Wird außerdem die Zeitumkehr-Symmetrie gebrochen, können in diesen Systemen topologische Bänder auftreten, deren Charakter von der Geometrie des Gitters abhängt.
Wir studieren das Verhalten der auftretenden chiralen Zustände auf dem Rand und untersuchen den Einfluss von Unordnung auf die topologische Struktur.


\end{otherlanguage}
