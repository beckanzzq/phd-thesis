\chapter*{Zusammenfassung}

\begin{otherlanguage}{ngerman}

Möglichkeiten für Anfänge:

Diese Arbeit beschäftigt sich mit der Realisierung und Untersuchung
verschiedener Modellsysteme im Rahmen der ultrakalten dipolaren
Quantengase.

Im Mittelpunkt stehen dabei Materialien mit topologisch geschützten
Eigenschaften.

Im Jahre... entdeckt K.v.Klitzing den QHE.. Heute als Standard für
das Ohm verwendet. Grundlage ist Robustheit der quantisierten Leitfähigkeit,
eine topologisch geschützte Eigenschaft

Laughlin, Thouless: topologischer Effekt, öffnet neues Feld.

Topologische Materialen, Robustheit gegenüber kleinen Störungen
Generell ultrakalte Atome: Simulation, reine Systeme, etc.
Dipolare Systeme: Polare Moleküle, Rydberg Atome, NV Center

\end{otherlanguage}
