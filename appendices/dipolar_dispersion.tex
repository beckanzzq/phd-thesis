\chapter{Dipolar dispersion relation}
\label{dipolar_dispersion}

\section{Definition and properties}
This appendix is concerned with the properties of the dipolar dispersion relation~\cite{Peter2012b,Syzranov2014,Peter2014}
\begin{align}
    \epskm = \sum_{j\ne 0} \frac{a^3}{|\vecR_j|^3}\ef{i\veck \vecR_j + i m \phi_{j}},\qquad m \in \{0, \pm 2\}
\end{align}
on a general two-dimensional Bravais lattice.
Here, $a$ is the lattice constant, $\vecR_j=(X_j, Y_j)^t$ is the position of the $j$-th lattice site and $\phi_j = \arg(X_j+i Y_j)$ is the polar angle in the lattice plane, i.e. the angle between the vector $\vecR$ and the positive $x$ axis. Note that $\epsk^{\pm 2}$ changes under a redefinition of the angle $\phi_j \goesto \phi_j + \phi_0$. The absolute value $\absv{\epsk^{\pm 2}}$ is invariant, however. For the remainder of this section we will measure lengths in units of $a$ and suppress the $j$ index, such that
\begin{align}
    \epskm = \sum_{\vec{R}\ne0} \frac{1}{R^3}\ef{i\veck \vecR + i m \phi_{\vecR}},
\end{align}
where $R=|\vecR|$ and the sum runs over all lattice sites except the origin.
Using $\phi_{-\vecR} = \phi_\vecR + \pi$ and the inversion symmetry of Bravais lattices,
it is easy to derive the following properties of the dispersion relation ($m$ is always even):
\begin{align}
    \eps_{-\veck}^m &= \epskm, \\
    \epsk^{-m} &= \bb{ \epskm }^*.
\end{align}
The second property also shows that $\epsZ^0 \in \mathbb{R}$.

\fig{.5}{epsk/epsk}{Plot of the function $\epskm$ on a high-symmetry path in the Brillouin zone of the square lattice. The linear behavior at $\veck=\vecGamma$ is apparent as well as the zeros of $\epsk^2$ at $\veck=\vecGamma, \vecM$.}

\section{Symmetries and zeros}
Let $\sop$ be a symmetry operation which leaves the lattice invariant, i.e. $\sop \, \{ \vecR \} = \{ \vecR \}$.
Since the scalar product is left invariant,
we can derive the property
\begin{align}
    \eps_{\sop\veck}^m = \sum_{\vecR\ne 0} R^{-3} \ef{i\veck \vecR + i m \phi_{\sop\vecR}}.
\end{align}
Now let $\sop=C_p$ be a rotation by $2\pi/p$. Then, we find
\begin{align}
    \eps_{\sop \veck}^m = \ef{2\pi i m/p} \epskm.
\end{align}
If $\veck^*$ is a high-symmetry point which is left invariant under the rotation, that is
$\sop\veck^* = \veck^* + \vec{G}$ with an arbitrary reciprocal lattice vector $\vec{G}$, we find $\eps_{\veck^*}^m = \ef{2\pi i m/p} \eps_{\veck^*}^m$, leading to a condition for zeros of the dispersion relation:
\begin{align}
    \eps^m_{\veck^*} = 0\qquad\text{if } m \notin p\mathbb{Z}.
\end{align}
For $m=\pm 2$, we can use any symmetry $C_n$ with $n>2$.

\paragraph{Triangular lattice:} The points $\vecGamma=(0,0)$ as well as
$\vecK=(4\pi/3,0)$ and $\vecK'=- \vecK$ are invariant under $C_3$ rotations. Therefore,
$\epsZ^2=\epsK^2=\eps^2_{\vecK'}=0$ on the triangular lattice.

\paragraph{Square lattice:} The points $\vecGamma$ and $\vecM=(\pi,\pi)$ are
invariant under $C_4$ rotations, leading to the two zeros $\epsZ^2=\epsM^2=0$, see \tref{epsk/epsk}.


\section{Low-momentum behavior}
\tlabel{lowmomentum}
For small $|\veck| \ll 1$ we can make a crude approximation and replace the discrete Fourier series
by a continuous Fourier transform (we will re-derive the results in this section in a \qu{cleaner} way later). We set $\veck\vecR = k R \cos(\phi-\psi)$ where
$\phi$ is the angle between $\vecR$ and the $x$-axis and $\psi$ is the
corresponding angle between $\veck$ and the $x$-axis. Then
\begin{align}
    \epskm &= \sum_{\vecR\ne 0} R^{-3} \ef{i\veck\vecR+i m \phi_\vecR}
    \approx \integralb{R_c}{\infty}{R} \integralb{0}{2\pi}{\phi} \ef{ikR \cos(\phi-\psi) + i m \phi} / R^2\\
    &= k \ef{im \psi} \integralb{z_c}{\infty}{z} \integralb{0}{2\pi}{\phi} \ef{i z \cos(\phi) + i m \phi} / z^2\\
    &= 2\pi k (-1)^{m/2} \ef{i m \psi} \integralb{z_c}{\infty}{z} J_{\absv{m}}(z)/z^2
\end{align}
where we have to introduce a lower cutoff $R_c$ for the radial integration which
we cannot determine. Consequently, this calculation is only useful
if we find quantities like $\epskm-\epsZ^m$ which (a) allow
us to safely take the limit $z_c \goesto 0$ while keeping $R_c$ constant and
(b) do not depend on $R_c$ in this limit.

\paragraph{$m=0$:} In the case of $\epsk^0$, the integral in the expression
\begin{align}
    \epsk^0 \approx 2\pi k \integralb{z_c}{\infty}{z} J_0(z)/z^2
\end{align}
diverges as $z_c^{-1}$ in the limit $z_c \goesto 0$. However, we can subtract this diverging part and calculate
\begin{align}
    \epskm - \frac{2\pi}{R_c} \approx 2\pi k \bc{ \, \integralb{z_c}{\infty}{z} \frac{J_0(z)}{z^2} - \frac{1}{z_c} } = -2\pi k,
\end{align}
giving us the correct expression for the linear part in $k$, i.e. $\epsk^0=\epsZ^0 - 2\pi |\veck|$.

\paragraph{$m=2$:} For any $|m|>1$, the integral can be evaluated in the limit $z_c\goesto 0$. We find
\begin{align}
    \epskm \approx \frac{2\pi (-1)^{m/2}}{m^2-1} k \ef{im \psi}
\end{align}
which allows us to determine $\epsk^{\pm 2} \approx -\frac{2\pi}{3} |\veck| \ef{\pm 2 i \psi}$.

% We can rewrite it in the following form
% \begin{align}
%     \eps^2_{\veck} = -\frac{2\pi}{3} \bb{\frac{k_x^2-k_y^2}{k} + i \frac{2k_x k_y}{k} } = -\frac{2\pi}{3} \frac{(k_x+i k_y)^2}{k} = -\frac{2\pi}{3} k \ef{2i \psi}
% \end{align}
% where $\psi = \text{arg}(k_x+i k_y)$. We see that
% \begin{align}
%     \absv{\eps^2_{\veck}} = \frac{2\pi}{3} k
%

\section{Exact results and Ewald summation}
\tlabel{ewald}
This section is mainly of interest for a numerically efficient determination of the function~$\epskm$ as well as a precise description of the low momentum behavior.
To start, it is convenient to rewrite $\epskm$ such that the explicit angular dependence is removed from the sum. With $\vecR = (X, Y)^t$ and $\ef{im\phi_\vecR} = (X+iY)^m/R^m$ we can write
\begin{align} \tlabel{epstochi}
    \epskm
    &= \sum_{\vec{R}\ne 0} \frac{(X+i Y)^m}{R^{3+m}}\ef{i\vec{k}\vec{R}}
    = \sum_{\vec{R}\ne 0} \frac{(-i \partial_{k_x}+ \partial_{k_y})^m}{R^{3+m}} \ef{i\vec{k}\vec{R}} \\
    &= (-i)^m (\partial_{k_x}+ i \partial_{k_y})^m \chi^{3+m}(\veck)
\end{align}
where $\chi^s(\veck) = \sum_{\vecR\ne 0} \ef{i \veck \vecR} R^{-s}$.
In the following, we will see how to derive exact results for the function $\chi^{3+m}(\veck)$.
% Using \tref{epstochi} we can use these to get

\subsection{Exact values on the square lattice}
\tlabel{exact}
At the $\vecGamma = (0, 0)$ point, the exact value of the function
$\chi^s(0) = 4\beta(\sfrac{s}{2})\zeta(\sfrac{s}{2})$
can be expressed with the help of the Riemann $\zeta$-function and the Dirichlet $\beta$-function~\cite{Glasser1973}.
This immediately leads to
\begin{align}
    \epsZ^0 = \chi^3(0) = 4 \zeta(\sfrac{3}{2}) \beta(\sfrac{3}{2})\approx 9.03.
\end{align}
To get the exact value at the $\vecM=(\pi,\pi)$ point, we denote the two sublattices of the bipartite square lattice by $A$ and $B$. Both have a lattice constant of $\sqrt{2}$.
By $A$, we denote the sublattice that includes the origin and $B$ is the sublattice which includes the nearest neighbors of the origin. The full lattice is denoted by $A+B$.
Then, $\epsZ^0 = \epsZ^0(A+B) = \epsZ^0(A) + \epsZ^0(B)$
and
$\epsM^0 = \epsZ^0(A) - \epsZ^0(B)$. With $\epsZ^0(A)=2^{-3/2} \epsZ^0$, we find
\begin{align}
    \epsM^0 = 2\epsZ^0(A) - \epsZ^0 = \bb{ 1/\sqrt{2} - 1 } \epsZ^0 \approx -2.65.
\end{align}
Using a similar technique with four sublattices, we can also find
\begin{align}
    \epsX^0 = \frac{1}{4}\bb{ 1- \sqrt{2} } \epsZ^0 \approx -0.94
\end{align}
at the $\vecX=(\pi, 0)$ point. These exact values can serve as a useful benchmark for any kind of approximation.

\subsection{Ewald summation}
This section uses ideas from \refcite{Muller2010} and extends the results to $m\ne 0$.
First, using the relation
\begin{align}
    \frac{1}{R^s}=\frac{1}{\Gamma(s/2)} \integralb{0}{\infty}{\lambda} \lambda^{s/2-1} \ef{-\lambda R^2},\qquad s>0
\end{align}
we can rewrite
\begin{align}
    \chi^s(\veck)&=\frac{1}{\Gamma(s/2)}\sum_{\vec{R}\ne 0} \bb{\int_0^\eta + \int_\eta^\infty}\! \mathrm{d}\lambda\, \lambda^{s/2-1} \ef{-\lambda R^2 + i \veck \vec{R}} \\
                 &= \frac{\eta^{s/2}}{\Gamma(s/2)}\sum_{\vec{R}\ne 0} \bb{\integralb{1}{\infty}{\lambda} \lambda^{-s/2-1} \ef{-\eta R^2 / \lambda} + \integralb{1}{\infty}{\lambda} \lambda^{s/2-1} \ef{-\eta \lambda R^2}} \ef{i\veck \vec{R}}
\end{align}
where we have substituted $\lambda \rightarrow \eta/\lambda$ in the first and $\lambda \rightarrow \eta \lambda$ in the second integral.
The parameter $\eta > 0$ determines the border between the real-space and the $k$-space summation.
Using the Poisson summation formula we transform
\begin{align}
    \sum_{\vec{R}\ne 0} \ef{-\eta R^2/\lambda + i \veck \vec{R}} = \sum_{\vec{R}} (\dots) - 1 = \frac{\pi\lambda}{\eta} \sum_{\vecq} \ef{-\frac{\lambda}{4\eta} \absvsq{\vecq+\veck}} - 1
\end{align}
which leads to
\begin{align} \tlabel{twosum}
    \chi^s(\veck) &= \frac{\eta^{s/2}}{\Gamma(s/2)} \bb{\frac{\pi}{\eta} \sum_{\vecq} E_{s/2}\!\bb{\frac{\absvsq{\vecq+\veck}}{4\eta}} - \frac{2}{s} + \sum_{\vec{R}\ne 0} E_{1-s/2}\!\bb{R^2 \eta} \ef{i \veck \vec{R}} }.
\end{align}
Here, we have introduced the exponential integral function
\begin{align}
    E_n(z) = \integralb{1}{\infty}{t} \frac{\ef{-z t}}{t^n}
\end{align}
Both sums in \tref{twosum} are fast-converging since high
values of $\vecq$ and $\vec{R}$, respectively, are exponentially suppressed by $E_n$.
Using this expression, we can accurately plot the function $\epskm$ in the Brillouin zone, see \tref{epsk/epsk}.

\subsection{Non-analytic behavior at low momenta}
For $s=3, m=0$ we can simplify \tref{twosum}:
\begin{align}
    \epsk^0 = \chi^3(\veck) = 2\pi \bb{ \sum_{\vecq} E_{3/2}\!\bb{\frac{\absvsq{\vecq+\veck}}{4\pi}} - \frac{2}{3} + \sum_{\vec{R}\ne 0} E_{-1/2}\!\bb{\pi R^2} \ef{i \veck \vec{R}} }
\end{align}
where we have explicitly chosen $\eta = \pi$. We can combine the two sums by writing the $\vecq=0$ term separately and by making the replacement $\vecq \goesto 2\pi\vec{R}$ for the rest of the $\vecq$-sum:
\begin{align}
    \epsk^0 &= 2\pi  E_{3/2}(k^2/4\pi) - \frac{4\pi}{3}\\
                 &\qquad+ 2\pi\sum_{\vec{R}\ne 0} \bb{ E_{3/2}( \pi \absvsq{\vecR+\veck/2\pi} ) + E_{-1/2}\!\bb{\pi R^2} \ef{i \veck \vec{R}} }.
\end{align}
Expanding the definitions of the exponential integral function, this can be written as
\begin{align} \tlabel{epsqewald}
    \epsk^0 &= -2 \pi  |\veck| \operatorname{erfc}( |\veck|/2 \sqrt{\pi} ) + 4 \pi \bb{ e^{-\frac{|\veck|^2}{4 \pi }}-\frac{1 }{3} } \\
            &\qquad+ 2\pi \sum_{\vecR\ne 0} \integralb{1}{\infty}{\lambda} \bc{ \lambda^{-3/2} \ef{-\pi\lambda\absvsq{\vecR+\veck/2\pi}} + \lambda^{1/2} \ef{-\pi \lambda R^2+ i\veck \vecR} }
\end{align}
This is the result stated in \refandpage{ab:Ewaldsummation}.
For general (even) $m$, i.e. for any odd $s=3+m$, we can expand
\begin{align}
    E_{s/2}(z) = \Gamma(1-s/2)\, z^{s/2-1} + \sum_{k=0}^\infty \frac{(-1)^{k+1} z^k}{k! (1+k-s/2)}
\end{align}
which means that the $\vecq=0$ term has a contribution
\begin{align}
\frac{\eta^{s/2}}{\Gamma(s/2)} \frac{\pi}{\eta} \Gamma(1-s/2) \bb{\frac{\absvsq{\veck}}{4\eta}}^{s/2-1} = \frac{2^{2-s}\pi\, \Gamma(1-s/2)}{\Gamma(s/2)} \absv{\veck}^{s-2}
\end{align}
Substituting $s=3+m$ and taking $m$ derivatives, the prefactor of the linear term becomes
\begin{align}
    \frac{\pi\, (1+m)!! \,\Gamma[-(m+1)/2]}{\Gamma[(3+m)/2]} = 2\pi \frac{(-2)^{1+m/2} (1+m/2)!}{(2+m)!}
\end{align}
For $m=0$ the prefactor is given by $-2\pi$, see \tref{epsqewald}, and for $m=2$ we find $2\pi/3$. Both results are in accordance with the estimation in \tref{lowmomentum}.
