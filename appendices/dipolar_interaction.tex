\chapter{From dipolar interactions to spin systems}
\section{Dipole-dipole interaction in spherical tensor notation}
The tensor part of the dipole-dipole interaction $D\equiv V_\text{dd} R_{ij}^3/C_\text{dd}=\vec{d_1}\vec{d_2}-3(\vec{d_1}\vec{\hat{r}})(\vec{d_2}\vec{\hat{r}})$ where $C_\text{dd} = \mu_0 /4\pi$ or $1/4\pi\epsilon_0$ can be written in terms of a spherical tensor $T^2(\vec{d_i},\vec{d_j})$ of rank 2, build from the two dipole moments, with compontents~\cite{Moller2009}
\begin{align*}
T^2_0(\vec{d_i},\vec{d_j}) &= \frac{1}{\sqrt{6}} (d^i_+ d^j_- + 2d^i_0 d^j_0 + d^i_{-} d^j_{+})\\
T^2_{\pm 1}(\vec{d_i},\vec{d_j}) &= \frac{1}{\sqrt{2}} (d^i_{\pm} d^j_0 + d^i_0 d^j_{\pm})\\
T^2_{\pm 2}(\vec{d_i},\vec{d_j}) &= d^i_{\pm} d^j_{\pm}
\end{align*}
where the dipole moments are spherical tensors with components
\begin{align}\label{eq:dipsph}
d_0 = d_z\qquad d_\pm = \mp\frac{1}{\sqrt{2}}\bb{d_x\pm i d_y}
\end{align}
The dipole-dipole interaction is then given by
\begin{align}
D &= -\sqrt{6} \,T^2(C)\cdot T^2(\vec{d_1},\vec{d_2})\\
&\equiv-\sqrt{6}  \sum_p (-1)^p C^2_{-p}(\theta,\phi) T^2_p(\vec{d_1},\vec{d_2})\nonumber
\end{align}
where $C^k_p(\theta,\phi)=\sqrt{\frac{4\pi}{2k+1}} Y^k_p(\theta,\phi)$ are modified (unormalized) spherical harmonics. We can expand
\begin{align*}
D &= (1-3\cos^2 \theta) \left[d_0 d_0 + \frac{1}{2}\bb{d_+ d_- + d_- d_+} \right] \\
&\,\,\,-\frac{3}{\sqrt{2}}\sin\theta \,\cos\theta \left[ \ef{+i\phi} (d_0 d_- + d_- d_0) - \ef{-i\phi} (d_0 d_+ + d_+ d_0) \right]\\
&\,\,\,-\frac{3}{2}\sin^2\theta \left[ \ef{+2i\phi} d_- d_- + \ef{-2i\phi} d_+ d_+\right]
\end{align*}
where in each term the first dipole operator acts on $i$ and the second on $j$. Notice that this coincides with $V_\text{dd}$ from the first part if we change from $d_\pm \rightarrow \mp j_\pm/\sqrt{2}$ and $d_0\rightarrow j_0$.

For a 2D geometry we set $\theta=\pi/2$ and get
\begin{align*}
D = d_0 d_0 + \frac{1}{2}\bb{d_+ d_- + d_- d_+} -\frac{3}{2} \left[ \ef{+2i\phi} d_- d_- + \ef{-2i\phi} d_+ d_+\right]
\end{align*}

For a 1D geometry in $x$-direction we set $\phi=0$ to get
\begin{align*}
D &= d_0 d_0 + \frac{1}{2}\bb{d_+ d_- + d_- d_+} -\frac{3}{2} \bb{ d_- d_- + d_+ d_+}\\
 &= d_z d_z + d_y d_y - 2 d_x d_x
\end{align*}

\subsection{Specializing to polar molecule states}
We assume that we are in $d\le 2$ dimensions and that a static DC field has been applied perpendicular to the 2D plane / 1D lattice. We keep the same notation $\ket{J,M}$ for the perturbed eigenstates after applying the field.
