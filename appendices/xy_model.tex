\chapter{Classical dipolar XY model}
\label{xy_model}

The classical XY model with ferromagnetic long-range dipolar couplings is given by
\begin{align}
    H = -J a^3 \sum_{i\ne j} \frac{\vec{S}_i \cdot \vec{S}_j}{|\vecR_i - \vecR_j|^3}
  = -J \sum_{i\ne j} \frac{\cos(\theta_i - \theta_j)}{|\vecr_{ij}|^3}.
\end{align}
Here, $J>0$ is the coupling constant, $\vec{S}_i=(\cos \theta_i, \sin \theta_i)^t$ are classical spins restricted to two dimensions and $\vecr_{ij}=(\vecR_i - \vecR_j)/a$ is the dimensionless relative distance, used for conciseness.

\section{High temperature expansion}
With the reduced inverse temperature $\beta = J/kT$, we can write the partition function as
\begin{align}
    Z=\int\prod_{k} \frac{\mathrm{d}\theta_k}{2\pi}  \operatorname{exp}\!\bigg(\beta \sum_{i\ne j} \cos(\theta_i - \theta_j) |\vecr_{ij}|^{-3} \bigg).
\end{align}
% \begin{align}
%     Z=\int\prod_{k} \frac{\mathrm{d}\theta_k}{2\pi}  \prod_{i\ne j}  \bb{1 + \beta \cos(\theta_i - \theta_j) |\vecr_{ij}|^{-3}} + \mathcal{O}(\beta^2)
% \end{align}
In the high temperature limit $\beta \ll 1$, keeping only linear terms in $\beta$,
the two-point correlation function $\meanv{\vec{S}_\alpha \vec{S}_\beta}$ is given by
\begin{align}
    \meanv{\vec{S}_\alpha \vec{S}_\beta} &= \meanv{\cos(\theta_\alpha - \theta_\beta)} \\ &= \frac{1}{Z} \int\prod_{k} \frac{\mathrm{d}\theta_k}{2\pi}  \cos(\theta_\alpha - \theta_\beta)
    \prod_{i\ne j}  \bb{1 + \beta \cos(\theta_i - \theta_j) |\vecr_{ij}|^{-3}}
\end{align}
After integration over all angles, only two terms in the expansion of the product contribute to the lowest order: $\beta \cos(\theta_\alpha - \theta_\beta) |\vecr_{\alpha\beta}|^{-3}$ and $\beta \cos(\theta_\beta - \theta_\alpha) |\vecr_{\beta\alpha}|^{-3}$. Then, the correlation function is given by
\begin{align}
% \meanv{\vec{S}_\alpha \vec{S}_\beta} &= \frac{1}{Z} \int\prod_{k} \frac{\mathrm{d}\theta_k}{2\pi} 2 \beta \cdot \frac{\cos(\theta_\alpha}-\vec{\beta}}^3} \\
% &= \frac{\beta}{Z} \frac{1}{\absv{\vec{\alpha}-\vec{\beta}}^3} + O\bb{\beta^2}
    \meanv{\vec{S}_\alpha \vec{S}_\beta} &= \frac{2 \beta}{Z} \int\prod_{k} \frac{\mathrm{d}\theta_k}{2\pi} \cos(\theta_\alpha - \theta_\beta)^2 \absv{\vecr_{\alpha\beta}}^{-3}
    = \frac{\beta}{Z} \frac{1}{\absv{\vecr_{\alpha\beta}}^3} = \frac{\beta}{|\vecr_{\alpha\beta}|^3}.
\end{align}
In the last step, we have used that $Z = 1+ \mathcal{O}(\beta^3)$, as the smallest loop includes three terms.

% TODO: update to \integral* commands if we use this section

% \section{Low temperature expansion}
% In the spin-wave approximation
% \begin{align}
% H=-J \sum_{i\ne j} \frac{\cos(\theta_i - \theta_j)}{|\vecr_{ij}|^3} \approx \text{const.} + \frac{J}{2} \sum_{i\ne j} \frac{(\theta_i - \theta_j)^2}{|\vecr_{ij}|^3}
% \end{align}
% Using
% \begin{align}
% G^{-1}(\vec{i}-\vec{j}) = \begin{cases}2\epsilon_{\vec{0}} \qquad  &i=j\\-\frac{2}{|\vecr_{ij}|^3}\qquad &i\ne j\end{cases}
% \end{align}
% with
% \begin{align}
% \epsilon_{\vec{q}} \equiv \sum_{\vec{j}\ne \vec{0}} \frac{\text{e}^{i \vec{q} \vec{j}}}{\absv{\vec{j}}^3}
% \end{align}
% the sum can be rearranged
% \begin{align}
% \sum_{i\ne j} \frac{(\theta_i - \theta_j)^2}{|\vecr_{ij}|^3} &= \sum_{i\ne j} \frac{\theta_i^2 + \theta_j^2- 2 \theta_i \theta_j}{|\vecr_{ij}|^3} \\
% &=
% 2 \sum_i \theta_i^2 \sum_{j\ne i} \frac{1}{|\vecr_{ij}|^3} - 2 \sum_{i\ne j} \frac{\theta_i \theta_j}{|\vecr_{ij}|^3} \\
% &=
% 2 \sum_i \epsilon_{\vec{0}} \theta_i^2 - 2 \sum_{i\ne j} \frac{\theta_i \theta_j}{|\vecr_{ij}|^3} \\
% &= \sum_{i, j} \theta_i G^{-1}(\vec{i}-\vec{j}) \theta_j
% \end{align}
% leading to (dropping the constant)
% \begin{align}
% H=\frac{J}{2} \sum_{i, j} \theta_i G^{-1}(\vec{i}-\vec{j}) \theta_j
% \end{align}
% The Fourier transform of $G^{-1}$ is given by
% \begin{align}
% G^{-1}(\vec{q})&=\sum_{\vec{j}} \text{e}^{i \vec{q} \vec{j}} G^{-1}(\vec{j}) \\
% &=2(\epsilon_{\vec{0}} - \epsilon_{\vec{q}})
% \end{align}
% Then the Hamilton function can be diagonalized
% \begin{align}
% H=J \int \frac{\mathrm{d}^2q}{v_0} (\epsilon_{\vec{0}} - \epsilon_{\vec{q}}) \absvsq{\theta_{\vec{q}}}
% \end{align}
% giving the dispersion relation for the $\vec{q}$ mode
% \begin{align}
% E(\vec{q}) = J (\epsilon_{\vec{0}} - \epsilon_{\vec{q}})
% \end{align}
% The correlation function
% \begin{align}
% \meanv{\vec{S_x} \vec{S_y}} &= \mathop{\text{Re}}\meanv{\ef{i(\theta_i-\theta_j)}}
% \end{align}
% can be evaluated using
% \begin{align}
% \meanv{\ef{i(\theta_i-\theta_j)}} &= \frac{1}{Z} \int\prod_{i} \frac{\mathrm{d}\theta_i}{2\pi} \expf{-\frac{\beta}{2} \sum_{i, j} \theta_i G^{-1}(\vec{i}-\vec{j}) \theta_j + i (\theta_i-\theta_j)} \\
% &= \expf{-\frac{G(\vec{0})-G(\vec{i}-\vec{j})}{\beta}}
% \end{align}
% In the Fourier representation
% \begin{align}
% G(\vec{l})=\int \frac{\mathrm{d}^2q}{v_0} \frac{\ef{i\vec{q}\vec{l}}}{2(\epsilon_{\vec{0}} - \epsilon_{\vec{q}})}
% \end{align}
% %For $q\ll1$ we have
% %\begin{align}
% %\epsilon_{\vec{0}} - \epsilon_{\vec{q}} \approx 2\pi \absv{\vec{q}}
% %\end{align}
% %Using this approximation for large $l$ we have
% %\begin{align}
% %G(\vec{l})&\approx\frac{1}{2\pi}\int \frac{\mathrm{d}^2q}{v_0} \frac{\ef{i\vec{q}\vec{l}}}{q}\\
% %&\approx \frac{1}{(2\pi)^3}\integralb{0}{\pi}{q}\integralb{0}{2\pi}{\phi} \ef{i q l \cos{\phi}} \\
% %&\approx \frac{1}{(2\pi)^2}\integralb{0}{\pi}{q} J_0(ql)
% %\end{align}
% For large $l\gg a$ the function $G(\vec{l})$ is isotropic and converges to a constant
% \begin{align}
% \lim_{l\rightarrow\infty} G(\vec{0})-G(l) = G_\infty \approx 0.063
% \end{align}
% This leads to
% \begin{align}
% \lim_{l\rightarrow\infty}\meanv{\vec{S_0} \vec{S_l}} = \expf{-\frac{G_\infty k_BT}{J}}
% \end{align}

