\chapter{Harmonic oscillator matrix elements of the dipolar interaction}
\label{matrix_elements}

\section{Talmi-Moshinsky transformation}
In this section, we show how to simplify the matrix elements of the dipolar interaction $V_{ijkl}=\smallbraketop{ij}{\Vdd}{kl}$ with each index $ijkl$ representing a set of 2D harmonic oscillator quantum numbers $i=(n_i,m_i,\sigma_i)$ and
\begin{align}
\Vdd(R,\phi) =  \frac{\Cdd}{R^3} \left[
\sigma^z_1 \sigma^z_2 - (\sigma^+_1 \sigma^-_2 +3 \ef{2i\phi} \sigma^-_1 \sigma^-_2  +\text{h.c.})
\right]
\end{align}
the dipolar interaction in terms of the polar coordinates $\vec{R}=\vecR_1-\vecR_2=(R\cos \phi, R\sin \phi)$ of the relative vector between two particles. The spin-part is easily resolved and we can concentrate on matrix elements of the form
\begin{align}
V^{\Delta m}\equiv\smallbraketop{n_1'm_1'n_2'm_2'}{\frac{\ef{i\Delta m\, \phi}}{R^3}}{n_1m_1n_2m_2}
\end{align}
with the difference in angular momentum $\Delta m = 0, \pm 2$.
It is useful to change the basis to center-of-mass and relative coordinate states with
\begin{align}
    \vec{Q} = \frac{\vec{R_1} + \vec{R_2}}{\sqrt{2}}, \qquad \vec{q} = \frac{\vec{R_1} - \vec{R_2}}{\sqrt{2}}.
\end{align}
Note the symmetric definition with the additional factor of $1/\sqrt{2}$ compared to the usual definition of the relative vector.
Due to the quadratic character of the potential, the new degrees of freedom $\vec{Q}$ and $\vec{q}$ are subject to the same harmonic potential. Thus, the product state $\ket{n_1m_1n_2m_2}=\ket{n_1m_1}\ket{n_2m_2}$ can be decomposed in terms of harmonic oscillator states $\ket{NM}\ket{nm}$ of the $\vec{Q},\vec{q}$ coordinates via
\begin{align}
\ket{n_1m_1n_2m_2} = \!\!\! \sum_{N,M,n,m} \!\!\! T^{n_1m_1n_2m_2}_{NMnm} \ket{NMnm}
\end{align}
where the $T^{n_1m_1n_2m_2}_{NMnm}$ are called Talmi-Moshinsky coefficients \cite{Moshinsky1959,Talmi1952}. Since the center-of-mass is not affected by the interaction, the relevant matrix elements are
\begin{align}
\smallbraketop{n'm'}{\frac{\ef{i\Delta m \phi}}{(\sqrt{2} q)^3}}{nm} = \delta_{m+\Delta m,m'} \integralb{0}{\infty}{q} \frac{R_{n'}^{m'}(q) R_n^m(q) \,q}{\bb{\sqrt{2} q}^3}
\end{align}
where the radial functions $R_n^m$ are given in terms of the generalized Laguerre polynomials $L_n^{\absv{m}}$ as
\begin{align}
R_n^m(q) = \sqrt{\frac{2n!}{(n+\absv{m})!}}\, q^{\absv{m}}\expf{-\frac{q^2}{2}}\cdot L_n^{\absv{m}}(q^2).
\end{align}
The matrix element $V^{\Delta m}$ is thus given by
\begin{align}
V^{\Delta m}= \!\!\! &\sum_{N,M,n,n',m} \!\!\! (T^*)^{n_1'm_1'n_2'm_2'}_{NMn'(m+\Delta m)} \cdot T^{n_1m_1n_2m_2}_{NMnm} \cdot
 \integralb{0}{\infty}{q} \frac{R_{n'}^{m+\Delta m}(q) R_n^m(q) \,q}{\bb{\sqrt{2} q}^3}.
\end{align}
where the remaining integral can be calculated analytically for specific values of $n,n',m$ and $\Delta m$.

\section{Lowest Landau level}
In the lowest Landau level where $n_i=0$ for all particles, this expression can be further simplified. The decomposition of a state $\ket{0m_10m_2}$ into relative and center of mass coordinates shows that only $\ket{0M0m}$ states appear due to energy conservation. If we consider a single spin component, the only relevant matrix element is the $\sigma^z_i \sigma^z_j$ part with $\Delta m=0$ and
\begin{align}
V^0 &= 2^{-3/2} \sum_{M,m}  (T^*)^{m_1'm_2'}_{Mm} \cdot T^{m_1m_2}_{Mm} \smallbraketop{m}{q^{-3}}{m}.
\end{align}
For $m\ge 0$ the integration yields
\begin{align}
\smallbraketop{m}{q^{-3}}{m} &= \frac{2}{m!} \integralb{0}{\infty}{q} q^{2m-2} \ef{-q^2}
= \frac{\Gamma\big(m-\frac{1}{2}\big)}{\Gamma\big(m+1\big)}
\end{align}
For $m=0$, the integration diverges as the integrand behaves like $q^{-2}$ for small $q$. However, if we consider the antisymmetrized matrix element
\begin{align}
\overline{V}_{m_1',m_2',m_1,m_2} \equiv V_{m_1',m_2',m_1,m_2} - V_{m_2',m_1',m_1,m_2},
\end{align}
the diverging term for $m=0$ cancels, as any of the other even-$m$ terms, and we are left with
\begin{align} \tlabel{vlll}
\overline{V}_{m_1',m_2',m_1,m_2} = 2^{-1/2} \!\! \sum_{m = 1,3,\dots}^{m_1+m_2} \!\! (T^*)^{m_1'm_2'}_{Mm} \cdot T^{m_1m_2}_{Mm} \frac{\Gamma\big(m-\frac{1}{2}\big)}{\Gamma\big(m+1\big)}
\end{align}
where $m_1+m_2 = m_1'+m_2'$ and  $M=m_1+m_2-m$. The Talmi-Moshinsky coefficients in the LLL are given by
\begin{align}
T^{m_1m_2}_{Mm} = \bb{\frac{-1}{\sqrt{2}}}^{m+m_1} \!\! \sqrt{\frac{m_1!\,m_2!}{m!\, M!}} \,\sum_{k=0}^{m_1} (-1)^k \binom{M}{k} \binom{m}{m_1-k}
\end{align}
