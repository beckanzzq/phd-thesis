\chapter{Creutz ladder model with dipolar spins}
\label{creutz_ladder}


\newcommand{\et}{\operatorname{\eta}}
\newcommand{\tbar}{\bar{t}}


\newcommand{\orbital}[2]{TODO}
\newcommand{\op}{\orbital{plus}{+}}
\newcommand{\om}{\orbital{minus}{-}}

\newcommand{\ladderKet}[3]{TODO}
\newcommand{\upp}{\ladderKet{0.7ex}{+}{plus}}
\newcommand{\dnp}{\ladderKet{-0.7ex}{+}{plus}}
\newcommand{\upm}{\ladderKet{0.7ex}{-}{minus}}
\newcommand{\dnm}{\ladderKet{-0.7ex}{-}{minus}}

\section{Setup}
% \begin{figure}[th]
%     \centering
%     \begin{tikzpicture}[scale=3.5]
%         \pgfmathsetmacro{\widthA}{1}
%         \pgfmathsetmacro{\widthH}{0.8}
%         \pgfmathsetmacro{\angleD}{atan(\widthH / \widthA)}

%         % Rectangles
%         \foreach \s in {0, ..., 3}
%             \draw [very thick] (\s * \widthA, 0) rectangle (\s * \widthA + \widthA, \widthH);

%         % Geometry labels
%         \node [below] at (\widthA / 2, 0) {$a$};
%         \node [left] at (0, \widthH / 2) {$h$};

%         % Delta angle
%         \draw [] (0, 0) -- (\widthA, \widthH);
%         \draw (\widthA / 3, 0) arc (0:\angleD:\widthA / 3);
%         \node [] at (\widthA * 0.4, \widthH * 0.15) {$\delta$};

%         % Tunneling labels
%         \node [below] at (\widthA * 2.5, 0) {$w_h$};
%         \node [left] at (\widthA * 2, \widthH / 2) {$w_v$};
%         \draw [color=plus,thick] (2 * \widthA, 0) -- (3 * \widthA, \widthH);
%         \node [below] at (\widthA * 2.65, \widthH / 2) {$w_d \ef{\pm 2i\delta}$};

%         % Molecules
%         \foreach \x in {0, ..., 4} {
%             \foreach \y in {0, 1}
%                 \node at (\x * \widthA, \y * \widthH) [fill=gray,circle,draw,inner sep=0] {$\pm$};
%         };
%     \end{tikzpicture}
%     \caption{Setup for the double-Creutz ladder\label{setup}}
% \end{figure}

We consider the system depicted in~\cref{setup}.
Polar molecules are fixed on a double-chain of lattice spacing $a$ and a separation of $h$ between the two chains.
The level scheme is the one described in \href{http://arxiv.org/abs/1410.5667}{arXiv:1410.5667}: every molecule can be excited into the states $\ketp$ or $\ketm$.

\section{Bloch Hamiltonian}
In the basis \upp, \upm, \dnp, \dnm, the Bloch Hamiltonian for this setup has the following form
\begin{align}
    H(k) =
    \begin{pmatrix}
        -t_+ \eps_k + \mu & w \eps_k & -t_+ \et^0_k & w \et^{-2}_k \\
        w \eps_k & -t_- \eps_k - \mu & w \et^{+2}_k & -t_- \et^0_k \\
        -t_+ \et^0_{k} & w \et^{+2}_{k}  & -t_+ \eps_k + \mu & w \eps_k \\
        w \et^{-2}_{k} & -t_- \et^0_{k} & w \eps_k & -t_- \eps_k - \mu
    \end{pmatrix}
\end{align}
Here, we have defined
\begin{align}
    \eps_k &= a^3 \sum_{x\ne 0} \frac{\ef{i k x}}{|x|^3} = 2 \sum_{j>0} \frac{\cos(k a j)}{|j|^3} \\
    \et^m_k &= a^3 \sum_{x} \frac{\ef{i k x + i m \phi_x}}{\bb{x^2 + h^2}^{3/2}}
    = \frac{1}{(h/a)^3} + \sum_{j>0} \frac{2\cos(ka j+m\phi_j)}{\bb{j^2+(h/a)^2}^{3/2}}
\end{align}
where $\phi_x = \arg(x+ i h)=\phi_j=\arg(j+ i h/a)$. Note, that $\eps_k=\eps_{-k}$ and $\et^m_k=\et^{-m}_{-k}$.

\subsection{Structure of the Hamiltonian}
We can write the Hamiltonian in the form
\begin{align}
    H(k) =
    \begin{pmatrix}
        S_k & T_k \\
        T_k^t & S_k \\
    \end{pmatrix}
\end{align}
with the two real $2\times2$ matrices $S_k$ and $T_k$.
$S_k$ is symmetric and $T_k$ satisfies $T_k^t=T_{-k}$.


\subsection{Symmetries}
Here, we use the notation from the 2D paper $\tbar = (t^-+t^+)/2$ and $t=(t^--t^+)/2$.
\paragraph{Inversion symmetry}
The system has inversion symmetry ($180^\circ$ rotation symmetry), given by:
\begin{align}
    H(-k) &= \mathcal{P} H(k) \mathcal{P} \\
    \mathcal{P} &= \sigma_x \otimes \id =
    \begin{pmatrix}
        0 & 0 & 1 & 0 \\
        0 & 0 & 0 & 1 \\
        1 & 0 & 0 & 0 \\
        0 & 1 & 0 & 0
    \end{pmatrix}
\end{align}
with $\mathcal{P}^2=\id$.
This symmetry can be broken -- for example -- by an on-site-potential depending on the sublattice (chain).

\paragraph{NOTE} Since the Hamiltonian is real, the operator $\mathcal{P}\mathcal{K}$, with $\mathcal{K}$ being complex
conjugation, can be considered as (another) time-reversal operation in the sense of the
classification scheme.


\paragraph{Time-reversal symmetry}
For $t=\mu=0$, the system is time reversal symmetric with $\mathcal{T}=U_\mathcal{T} \mathcal{K}$:
\begin{align}
    H(-k)&=\mathcal{T} H(k) \mathcal{T}^{-1} = U_\mathcal{T}H(k)^* U_\mathcal{T}^\dagger \\
    U_\mathcal{T} &= \id \otimes \sigma_x =
    \begin{pmatrix}
        0 & 1 & 0 & 0 \\
        1 & 0 & 0 & 0 \\
        0 & 0 & 0 & 1 \\
        0 & 0 & 1 & 0
    \end{pmatrix}
\end{align}
with $\mathcal{T}^2=\id$.

In combination with the inversion symmetry, at $t=\mu=0$, the Hamiltonian commutes with $\mathcal{P}\mathcal{T} = \sigma_x \otimes \sigma_x \, \mathcal{K}$:
\begin{align}
    [\mathcal{P}\mathcal{T}, H(k)] = 0\qquad \Rightarrow \qquad [\sigma_x \otimes \sigma_x, H(k)]= 0
\end{align}

\paragraph{Particle-hole symmetry}
For $\bar{t}=0$, the system is particle-hole symmetric:
\begin{align}
    -H(-k) &= \mathcal{C}^{-1} H(k) \mathcal{C} =  U_\mathcal{C} H^*(k) U_\mathcal{C}^\dagger \\
    U_\mathcal{C} &= \id \otimes \sigma_y
\end{align}
with $\mathcal{C}^2=-1$.

In combination with inversion symmetry, this leads to a sublattice symmetry with
\begin{align}
    \mathcal{S}_\mathcal{P} = U_\mathcal{C} \mathcal{P} = \sigma_x \otimes \sigma_y
\end{align}
which anti-commutes with the Hamiltonian.

In combination with TRS (at $\tbar=t=\mu=0$), PHS leads to a sublattice-symmetry with
\begin{align}
    \mathcal{S}_\mathcal{T} = U_\mathcal{C}U_\mathcal{T} =  \id\otimes (-i\sigma_z)
\end{align}
which anti-commutes with the Hamiltonian.

Also note that the product of the two operators which anti-commute with H is just the operator which commutes with H: $S_\mathcal{T}S_\mathcal{P} = \mathcal{T}\mathcal{P}$.

\section{Zak phase}
The Zak phase is defined by
\begin{align}
    \gamma_\nu = i\integralb{-\pi/a}{\pi/a}{k} \braketop{u_\nu(k)}{\partial_k}{u_\nu(k)}
\end{align}
where $\ket{u_\nu(k)}$ is the Bloch eigenstate of the $\nu$-th band.
If a one-dimensional system is inversion symmetric, the Zak phase can only be\footnote{Zak, PRL 1988} $\gamma=0$ or $\gamma=\pi$.

\section{Topological phase diagram}
(In this section, we set $\tbar=0$, as it has no influence for the topological structure).
We investigate the structure of the Hamiltonian at the two time-reversal invariant momenta $k=0$ and $k=\pi/a$.
The structure of the Hamiltonian (at these points) simplifies to:
\begin{align}
    H(k) =
    \begin{pmatrix}
        S_k & T_k \\
        T_k^t & S_k \\
    \end{pmatrix}
    =
    \begin{pmatrix}
        S_k & T_k \\
        T_k & S_k \\
    \end{pmatrix}
\end{align}
Then, we can use:
\begin{align}
    \det
    \begin{pmatrix}
        S_k & T_k \\
        T_k & S_k
    \end{pmatrix}
    = \det (S_k+T_k) \det (S_k-T_k)
\end{align}
Both $A$ and $B$ are traceless in our case, so we can see that the eigenvalues come in two pairs with $E_1 = -E_4$ and $E_2 = -E_3$. First, we are interested in the touching of the lower two bands (then, the upper two will also touch).
This is the case if $\det (S_k+T_k) = \det (S_k-T_k)$. We find
\begin{align}
    (t(\eps_k+\eta_k^0)+\mu)^2 + w^2 \absvsq{\eps_k+\eta^{2}_k} = (t(\eps_k-\eta_k^0)+\mu)^2 + w^2 \absvsq{\eps_k-\eta^{2}_k}
\end{align}
After simplification, we have
\begin{align}
    t^2\eps_k\et_k^0+t\mu \et_k^0 + w^2 \eps_k \et_k^ = 0
\end{align}
From this, we can derive the following boundaries for the topological phase diagram:
\begin{align}
    \mu_1 &= -\frac{\eps_0}{t}\bb{t^2+w^2 \frac{\et_0^2}{\et_0^0}} \\
    \mu_2 &= -\frac{\eps_\pi}{t}\bb{t^2+w^2 \frac{\et_\pi^2}{\et_\pi^0}}
\end{align}
We can explicitly write
\begin{align}
    \eps_0 &= 2\zeta(3) \approx 2.40 \\
    \eps_\pi &= -3 \zeta(3) / 2 \approx -1.80
\end{align}
The topological phase diagram crucially depends on the two parameters $w^2 \et^2_0/\et^0_0$ and $w^2 \et^2_\pi/\et^0_\pi$.

\subsection{Vanishing spin-orbit coupling}
For $w=0$, we find only two straight lines in the phase diagram, corresponding to
\begin{align}
    \mu_1 &= -\eps_0 t \\
    \mu_2 &= -\eps_\pi t
\end{align}


% \newpage


% We immediately find
% \begin{align}
%     E_{1/2}(k) = -\sqrt{\bb{t(\eps_k \pm \et^0_k)+\mu}^2+w^2\absvsq{\eps_k\pm \et_k^2}}
% \end{align}
% At the $k=0$ point, we have the relations
% \begin{align}
%     \eps_0 &= 2\zeta_3 \\
%     \et^0_0&=\sum_{j} \frac{1}{(j^2+(h/a)^2)^{3/2}}
% \end{align}

% \subsection{Alternative basis}
% In the basis U+, L+, U-, L- with U/L = upper/lower, the Bloch Hamiltonian has the following form:
% \begin{align}
%     H(k) =
%     \begin{pmatrix}
%         t \eps_k  & t \et^0_k & w \eps_k & w \et^{-2}_k \\
%         t \et^0_{-k} & t \eps_k   & w \et^{-2}_{-k} & w \eps_k \\
%         w \eps_k & w \et^{+2}_k & -t \eps_k  & -t \et^0_k \\
%         w \et^{+2}_{-k} & w \eps_k & -t \et^0_{-k} & -t \eps_k
%     \end{pmatrix}
% \end{align}
% we have set $\mu = 0$. Now we define:
% \begin{align}
%     \Omega^m_k =
%     \begin{pmatrix}
%         \eps_k & \et^{m}_k \\
%         \et^{m}_{-k} & \eps_k
%     \end{pmatrix}
% \end{align}
% (notice the $-k$ in the lower left) and write
% \begin{align}
%     H(k)/t =
%     \begin{pmatrix}
%         \Omega^0_k & \omega\Omega^{-2}_k \\
%         \omega\Omega^{+2}_k & -\Omega^0_k
%     \end{pmatrix}
% \end{align}
% where $\omega = w/t$. The matrix $\Omega^m_k$ has the following property:
% \begin{align}
%     \bb{\Omega^{m}_k}^\dagger = \Omega^{-m}_k
% \end{align}
% In addition, at the time-reversal invariant momenta $k=0, \pi$, the matrix becomes symmetric: $(\Omega^m_k)^t=\Omega^m_k$.

\section{Minimal cutoff}
Including just the necessary terms in the long-range tunneling, we have
\begin{align}
    \eps_k &= 2 \cos(ka) \\
    \et^m_k &= \frac{1}{(h/a)^3} + \frac{2}{(l/a)^3}\cos(k a+ m\delta)
\end{align}
where $l^2=a^2+h^2$ and $\tan \delta = h/a$ (see~\cref{setup}).


\section{Mapping to two Creutz ladders}
In the following, we assume that $t=\mu=0$. We consider a geometry with a rectangular ladder where the spacing is given by $a$ and the separation between the chains is given by the height $h$.
We only keep NN and NNN terms in the remaining spin-orbit tunneling terms.
In the basis $\upp, \dnm, \dnp, \upm$, the Bloch Hamiltonian has the form
\begin{align}
    H(k) =
    \begin{pmatrix}
        0 & \et^{-2}_{k} & 0 & \eps_k \\
        \et^{-2}_{k} & 0 & \eps_k & 0 \\
        0 & \eps_k & 0 & \et^{+2}_{k} \\
        \eps_k & 0 & \et^{+2}_k & 0
    \end{pmatrix}
\end{align}
where $\eps_k = 2 w_h \cos(ka)$ and $\eta_k^m = -w_v + 2 w_d \cos(ka + m\delta)$. The term $w_h$ describes the horizontal (intra-chain) tunneling, $w_v$ is the vertical (inter-chain) coupling and $w_d$ is the strength of the diagonal (inter-chain) tunneling. The angle $\delta$ is given by $\tan \delta = h/a$.

We first treat the vertical coupling $w_v$. On each link of the chain the $w_v$ term leads to a binding and anti-binding ``molecule'' of the shape
\begin{align}
    \ketup_\pm = \frac{1}{\sqrt{2}} \bb{ \upp \pm \dnm } \\
    \ketdn_\pm = \frac{1}{\sqrt{2}} \bb{ \dnp \pm \upm }
\end{align}
Rotating into this new basis $\ketup_-, \ketdn_-, \ketup_+, \ketdn_+$, we find the Bloch Hamiltonian
in block diagonalized form\footnote{These states are eigenstates of the operator $\mathcal{P}\mathcal{T}=\sigma_x \otimes \sigma_x$ which commutes with the Hamiltonian for $\mu=t=0$.}
\begin{align}
    H(k) =
    \begin{pmatrix}
        -\et_k^{-2} & -\eps_k & 0 & 0 \\
        -\eps_k & -\et_k^{+2} & 0 & 0 \\
        0 & 0 & \et^{-2}_{k} & \eps_k \\
        0 & 0 & \eps_k & \et^{+2}_k
    \end{pmatrix} =
    \begin{pmatrix}
        -H_+(k) & 0 \\
        0 & H_+(k)
    \end{pmatrix}
\end{align}
In the following, we will concentrate on the ``binding'' part with the states $\ketup_+, \ketdn_+$. We can write the lower block as:
\begin{align}
    H_+(k) = &\begin{pmatrix}
        -w_v + 2w_d \cos(ka-2\delta) & 2w_h \cos(ka) \\
        2w_h\cos(ka) & -w_v + 2w_d \cos(ka+2\delta)
    \end{pmatrix} \\
    = &+\id \; \times (-w_v + 2w_d \cos(2\delta) \cos(ka)) \\
    &+\sigma_z \times 2w_d \sin(2\delta) \sin(ka) \\
    &+\sigma_x \times 2w_h \cos(ka)
\end{align}

% \begin{figure}[th]
%     \centering
%     \begin{tikzpicture}[scale=3.5]
%         \pgfmathsetmacro{\widthA}{1}
%         \pgfmathsetmacro{\widthH}{0.8}
%         \pgfmathsetmacro{\angleD}{atan(\widthH / \widthA)}

%         % Rectangles
%         \foreach \s in {0, ..., 3} {
%             \draw [very thick] (\s * \widthA, 0)       -- (\s * \widthA + \widthA, 0);
%             \draw [very thick] (\s * \widthA, \widthH) -- (\s * \widthA + \widthA, \widthH);
%             \draw [dashed] (\s * \widthA, \widthH)     -- (\s * \widthA + \widthA, 0);
%             \draw [dashed] (\s * \widthA, 0)           -- (\s * \widthA + \widthA, \widthH);
%             \draw [very thick,-{Stealth}] (\s * \widthA, \widthH) -- (\s * \widthA + 0.5 * \widthA, \widthH);
%             \draw [very thick,-{Stealth}] (\s * \widthA + \widthA, 0)       -- (\s * \widthA + 0.5 \widthA, 0);
%         }

%         % Tunneling labels
%         \node at (\widthA * 1.35, 0.2 * \widthH) {$w_h$};
%         \node at (\widthA * 1.90, 0.2 * \widthH) {$w_h$};
%         \node [below] at (1.5 \widthA, 0) {$w_d \ef{2i\delta}$};
%         \node [above] at (1.5 \widthA, \widthH) {$w_d \ef{2i\delta}$};

%         % Molecules
%         \foreach \x in {0, ..., 4} {
%             \node at (\x * \widthA, \widthH) [fill=lightgray,circle,draw,inner sep=0] {$\uparrow$};
%             \node at (\x * \widthA, 0) [fill=lightgray,circle,draw,inner sep=0] {$\downarrow$};
%         };
%     \end{tikzpicture}
%     \caption{\label{creutz}Tunneling links for $\ketup_+, \ketdn_+$. The flux through each plaquette is $4\delta$.}
% \end{figure}

The resulting model is depicted in~\cref{creutz}.
This sector of the full Hamiltonian is equivalent to a Creutz ladder\footnote{M. Creutz, Phys. Rev. Lett. 83, 2636 (1999).} with parameters\footnote{For the paper M. Tovmasyan, E. P. L. van Nieuwenburg, and S. D. Huber, Phys. Rev. B 88, 220510 (2013), the parameters are $m=0, \delta=0, t=w_h$ and $\epsilon=w_d/w_h-1$} $K=w_h, M=0, r=w_d/w_h$ and $\theta=2\delta$. The magnetic flux per plaquette is given by $2\theta=4\delta$. The Creutz ladder supports two perfectly flat bands for $M=0, r=1$ and $\theta=\pi/2$. In our parameters this happens for $w_h=w_d$ and $\delta=\pi/4$.

% \fig{0.8}{dispersion-creutz}{Dispersion relation for the double-Creutz-ladder.
% The lower band has a flatness ratio of $\sim 2$. The plot on the right shows the spectrum of an open ladder with two edge states in each of the two band gaps.}

\subsection{Symmetries}
The model is time reversal symmetric with $\mathcal{T}=\sigma_x \mathcal{K}$:
\begin{align}
    \mathcal{T}H_+(k)\mathcal{T}^{-1}= H_+(-k)
\end{align}
For $w_v=0$ (energy offset) and $\delta=\pi/4$, the system is additionally particle-hole symmetric\footnote{This only works because of the cutoff!}:
\begin{align}
    \mathcal{C}H_+(k)\mathcal{C}^{-1}= -H_+(-k)
\end{align}
with $\mathcal{C}= \sigma_z \mathcal{K}$. This leads to a chiral symmetry $\mathcal{S} = i\sigma_y$ which anti-commutes with the Hamiltonian.

In total, at the point $\delta=\pi/4$, the system has $\mathcal{T}^2=1, \mathcal{C}^2=1, \mathcal{S}^2=1$ and is in symmetry class BDI.

\subsection{Edge states}
In the flat band limit, the edge state (on the left side) is given by
\begin{align}
    \ket{E}_+ = \ketup_+ + i \ketdn_+ = \upp + \dnm + i\dnp + i\upm
\end{align}
The edge state of the ``anti-binding'' ladder is given by
\begin{align}
    \ket{E}_- = \ketup_- + i \ketdn_- = \upp - \dnm + i\dnp - i\upm
\end{align}


\subsection{Adding \texorpdfstring{$\tbar$}{t-bar} terms}
Adding NN and NNN terms with $t_+=t_-\ne 0$, we find an on-site-coupling:
\begin{align}
    \braketop{\downarrow_+}{H}{\uparrow_+} = -\tbar_v
\end{align}
as well as the following tunneling contributions
\begin{align}
    \bra{\uparrow_+}_{i}H\ket{\uparrow_+}_{i+1} &= -\tbar_h \\
    \bra{\downarrow_+}_{i}H\ket{\downarrow_+}_{i+1} &= -\tbar_h \\
    \bra{\uparrow_+}_{i}H\ket{\downarrow_+}_{i+1} &= -\tbar_d
\end{align}
The Creutz-ladder parameters are now modified:
\begin{align}
    M &= -\tbar_v \\
    K &= w_h - \tbar_d\\
    K r \ef{i \theta} &= w_d \ef{2i\delta} - \tbar_h
\end{align}
the last equation yields
\begin{align}
    r &= \sqrt{\frac{(w_d\cos 2\delta - \tbar_h)^2 + (w_d \sin 2\delta)^2}{(w_h-\tbar_d)^2}} \\
    \theta &= \atan \frac{\sin 2\delta}{\cos 2\delta-\tbar_h/w_d}
\end{align}

\section{Basis transformation for the full Hamiltonian}
Writing the full Bloch Hamiltonian in the basis $\ketup_-, \ketdn_-, \ketup_+, \ketdn_+$, we find
\begin{align}
    H(k) &=
    \begin{pmatrix}
        -\tbar\eps_k -w\et_k^{-2} & -\tbar\et_k^0 -w\eps_k & \mu+t\eps_k & t\et_k^0 \\
        -\tbar\et^0_k - w\eps_k & -\tbar \eps_k - w\et_k^{+2} & t\et_k^0 & \mu + t\eps_k \\
        \mu+t\eps_k & t\et_k^0 & -\tbar\eps_k +w\et_k^{-2} & -\tbar\et_k^0 +w\eps_k \\
        t\et_k^0 & \mu + t\eps_k & -\tbar\et^0_k + w\eps_k & -\tbar \eps_k + w\et_k^{+2}
    \end{pmatrix} \\
    &= \id \otimes (-\tbar \, M_k) + \sigma_x \otimes (t M_k + \mu) - \sigma_z \otimes (w N_k)
\end{align}
where
\begin{align}
    M_k = \begin{pmatrix}
        \eps_k & \et_k^0 \\
        \et_k^0 & \eps_k
    \end{pmatrix},\qquad
    N_k = \begin{pmatrix}
    \et_k^{-2} & \eps_k \\
    \eps_k & \et_k^{+2}
    \end{pmatrix}
\end{align}

\subsection{New representation of the symmetry operators}
\begin{align}
    \mathcal{P} &= \id \otimes \sigma_x \\
    \mathcal{T} &= -\sigma_z \otimes \sigma_x \\
    \mathcal{P}\mathcal{T} &= -\sigma_z \otimes \id \text{ (now diagonal)} \\
    \mathcal{C} &= \sigma_y \otimes \sigma_x \\
    \mathcal{S}_\mathcal{T} &= \sigma_x \otimes \id \\
    \mathcal{S}_\mathcal{P} &= \sigma_y \otimes \id
\end{align}

\subsection{Time-reversal invariant point: two Creutz ladders}
For $t=\mu=0$, the Hamiltonian block-diagonalizes into the two sectors corresponding to the two Creutz ladders. We have
\begin{align}
    H(k) = \id \otimes (-\tbar M_k) - \sigma_z \otimes (w N_k)
\end{align}
We focus on a single Creutz ladder (lower block). Then, we find
\begin{align}
    H_-(k) &= -\tbar M_k + w N_k \\
           &= \bc{-\tbar \eps_k + \frac{w}{2} \bb{\et^{-2}_k + \et^{+2}_k}}\id + \frac{w}{2}\bb{\et^{-2}_k - \et^{+2}_k} \sigma_z + (-\tbar \et^0_k + w \eps_k) \sigma_x \\
           &= \bb{-\tbar \eps_k + w \et^{s}_k} \id + w\et^{a}_k \sigma_z + (-\tbar \et^0_k + w \eps_k) \sigma_x
\end{align}
The band energies are given by
\begin{align}
    E_\pm = -\tbar \eps_k + w \et^s_k \pm \sqrt{(w \et^a_k)^2+(-\tbar\et^0_k+w\eps_k)^2}
\end{align}

\subsection{Flat band}
The term $w \et^a_k$ is much smaller than $-\tbar\eta_k^0+w \eps_k$. We rotate into a new basis
$\ket{\psi}_\pm = (\ketup_- \pm \ketdn_-)/\sqrt{2}$ where the role of $\sigma_x$ and $\sigma_z$ are interchanged. Now the Hamiltonian has the shape
\begin{align}
    H_-(k) &= \bb{ -\tbar \eps_k + w \et^{s}_k }\id +(-\tbar \et^0_k + w \eps_k) \sigma_z +  w\et^{a}_k \sigma_x
\end{align}
The small term is the coupling between the two bands. Ignoring this coupling, the lower band will be flat if
\begin{align}
    E_-(k) = -\tbar \eps_k + w \et^s_k - (-\tbar \et^0_k + w \eps_k) = E_-(0)
\end{align}
Rearranging, we find
\begin{align}
    \frac{\tbar}{w} = \frac{-E_-(0)/w + \et^s_k \pm \eps_k}{\eps_k \pm \et^0_k}
\end{align}
It turns out, that the function on the right hand side can be made quite flat for a certain value of $\tbar/w$.

\section{Edge state properties}
% \fig{0.9}{energy-diff}{Scaling of the energy difference between the two edge states, proportional to $1/L^3$.}
\subsection{State}
There are two states on each edge with approximate energies $\pm w_v$. In the following, we will focus on one particular edge states (with energy $-w_v$). In the presence of long-range tunneling,
the amplitudes of each edge state decay with $1/x^3$ into the bulk.
Correspondingly, the coupling $\sim \sum \frac{1}{x^3(L-x)^3}$ between the two edge states is proportional to $1/L^3$ (see \cref{fig:scaling}).

If the dipolar interaction is cutoff at a finite distance $R_c < L$, the edge state decays exponentially into the bulk.

% \begin{figure}
%     \ig{0.32}{scaling-cutoff120}
%     \ig{0.32}{scaling-cutoff5}
%     \ig{0.32}{scaling-randomOnsite}
%     \caption{(a) Algebraic scaling of the energy difference between the left and the right edge state. (b) Exponential scaling for cutoff $5a$. (c) Splitting for random on-site potential which destroys inversion symmetry.\label{fig:scaling}}
% \end{figure}

% \fig{.7}{top}{Topological phase diagram for the double Creutz ladder system as a function of $\tbar/w$ and $h/a$. Thick lines indicate band touchings at the $k=0$ or $k=\pi/a$ point. The shaded areas have nontrivial winding numbers for the upper, lower or both Creutz ladders.}

