\chapter[Realizing the Creutz ladder model with dipolar spins]{Realizing the Creutz ladder model\texorpdfstring{\\}{ }with dipolar spins}
\label{creutz_ladder}


\newcommand{\ep}{\operatorname{\epsilon}}
\newcommand{\et}{\operatorname{\eta}}

\newcommand{\ladderKet}[2]{\textstyle|\genfrac{}{}{0pt}{}{#1}{#2}\rangle}
\newcommand{\upp}{\ladderKet{+}{}}
\newcommand{\dnp}{\ladderKet{}{+}}
\newcommand{\upm}{\ladderKet{-}{}}
\newcommand{\dnm}{\ladderKet{}{-}}

\fig{.7}{setup}{Setup for the realization of the Creutz ladder. One dipole is pinned at each lattice site of a double chain with lattice spacing $a$ in horizontal direction and a distance $h$ between the two chains. A unit cell at site $i$ includes both the upper and lower dipole.}

Consider the one-dimensional system depicted in~\tref{setup}.
Single dipoles are located at each site of a double-chain lattice with spacing $a$ and a separation of $h$ between the two chains.
The level scheme is the one discussed in \cref{topological_bands}, but without any microwave field (see \refandpage{tb:fig1/fig1b}). Every dipole initially starts in the ground state $\ketz$ and can be excited into one of the two orbitals $\ketp$ or $\ketm$.
We use the notation $\ladderKet{\alpha}{}_i = \ket{\alpha}_{i,\text{upper}} \ket{0}_{i,\text{lower}}$ to indicate a state at lattice site $i$ where the dipole on the upper chain has been excited into the orbital $\alpha \in \{+, -\}$ and the dipole on the lower chain is still in the ground state.
Conversely, $\ladderKet{}{\alpha}_i$ describes an excitation on the lower chain.
Then, we can describe the local basis for each lattice site with four different states $\upp, \upm, \dnp, \dnm$.
Introducing hard-core bosons for each of these states and transforming to momentum space in the same way as in \cref{topological_bands}, we find that the Bloch Hamiltonian takes the form
\begin{align}
    H(k) =
    \begin{pmatrix}
        -t \ep_k  & w \ep_k & -t \et^0_k & w \et^{-2}_k \\
        w \ep_k & -t \ep_k  & w \et^{+2}_k & -t \et^0_k \\
        -t \et^0_{k} & w \et^{+2}_{k}  & -t \ep_k  & w \ep_k \\
        w \et^{-2}_{k} & -t \et^0_{k} & w \ep_k & -t \ep_k
    \end{pmatrix}.
\end{align}
Here, $t$ is the orbital-preserving tunneling strength ($t \equiv t^+ = t^-$) and $w$ is the spin-changing tunneling rate, as defined in \cref{tb:tunnelingRates}.
The one-dimensional variant of the dipolar dispersion relation comes in two forms. The function
\begin{align}
    \ep_k &= a^3 \sum_{x\ne 0} \frac{\ef{i k x}}{|x|^3} = 2 \sum_{j>0} \frac{\cos(k a j)}{|j|^3}
\end{align}
includes all intra-chain processes and the function
\begin{align}
    \et^m_k &= a^3 \sum_{x} \frac{\ef{i k x + i m \phi_x}}{\bb{x^2 + h^2}^{3/2}}
    = \frac{1}{(h/a)^3} + \sum_{j>0} \frac{2\cos(ka j+m\phi_j)}{\bb{j^2+(h/a)^2}^{3/2}}
\end{align}
covers all inter-chain processes with $\phi_x = \arg(x+ i h)=\phi_j=\arg(j+ i h/a)$ the angle between the horizontal lattice axis and the interconnection line between the two dipoles.
Note, that $\ep_k=\ep_{-k} \in \mathds{R}$ is symmetric in $k$ and $\et^m_k \in \mathds{R}$ satisfies the relation $\et^m_k=\et^{-m}_{-k}$.

\section{Symmetries}
The described system has an inversion symmetry (or $180^\circ$ rotation symmetry), described by $H(-k) = \mathcal{P} H(k) \mathcal{P}$, where the unitary operation $\mathcal{P} = \sigma_x \otimes \id$ flips the upper and lower chains.
In addition, the system is time-reversal symmetric, i.e.~$H(-k)=\mathcal{T} H(k) \mathcal{T}^{-1}$.
Here, time-reversal is described by the anti-unitary operator $\mathcal{T}=U_\mathcal{T} \mathcal{K}$, where $\mathcal{K}$ is complex conjugation and $\mathcal{U}_\mathcal{T} = \id \otimes \sigma_x$ is a unitary operator that exchanges the two orbitals.
The time-reversal operation satisfies $\mathcal{T}^2 = \id$.
The combination of these two symmetries gives rise to an operator $\mathcal{P}\mathcal{T} = \sigma_x \otimes \sigma_x \, \mathcal{K}$ which commutes with the Hamiltonian:
\begin{align}
    [\mathcal{P}\mathcal{T}, H(k)] = 0\qquad \Rightarrow \qquad [\sigma_x \otimes \sigma_x, H(k)]= 0.
\end{align}
The second commutation relation follows from the fact that $H(k)$ is real-valued.


\section{Mapping to two Creutz ladders}
Using the knowledge about the symmetry, we can block-diagonalize the Hamiltonian. To do so, we change to a basis which diagonalizes the operator $\sigma_x \otimes \sigma_x$ that commutes with the Hamiltonian. We define
\begin{align}
    \ketup_\pm = \frac{1}{\sqrt{2}} \bb{ \upp \pm \dnm }, \\
    \ketdn_\pm = \frac{1}{\sqrt{2}} \bb{ \dnp \pm \upm }.
\end{align}
Notice how these four states are invariant (up to a phase) under a combined flip of the chains $\ladderKet{\alpha}{} \leftrightarrow \ladderKet{}{\alpha}$ and the orbitals $+ \leftrightarrow -$.
In the new basis $\ketup_-, \ketdn_-, \ketup_+, \ketdn_+$ the Hamiltonian takes the form
\begin{align}
    H(k) &=
    \begin{pmatrix}
        -t\ep_k -w\et_k^{-2} & -t\et_k^0 -w\ep_k &  &  \\
        -t\et^0_k - w\ep_k & -t \ep_k - w\et_k^{+2} &  &  \\
         &  & -t\ep_k +w\et_k^{-2} & -t\et_k^0 +w\ep_k \\
         &  & -t\et^0_k + w\ep_k & -t \ep_k + w\et_k^{+2}
    \end{pmatrix} \\
    &= - t \, \id \otimes M_k - w \, \sigma_z \otimes N_k.
\end{align}
We see that the Hamiltonian factors into two blocks with $H_\mp(k) = -t M_k \mp w N_k$ where we have defined
\begin{align}
    M_k = \begin{pmatrix}
        \ep_k & \et_k^0 \\
        \et_k^0 & \ep_k
    \end{pmatrix},\qquad
    N_k = \begin{pmatrix}
    \et_k^{-2} & \ep_k \\
    \ep_k & \et_k^{+2}
    \end{pmatrix}.
\end{align}

% We first treat the vertical coupling $w_v$. On each link of the chain the $w_v$ term leads to a binding and anti-binding ``molecule'' of the shape

\subsection{Single block}

\fig{.7}{creutz}{Tunneling links for the two states $\ketup_+$ and $\ketdn_+$. Notice that the two depicted chains live in an abstract space which is not to be confused with the real space of the original ladder. The tunneling along a single chain is determined by the \emph{diagonal} elements of the real-space model while the inter-chain hopping is given by the \emph{horizontal} elements. A constant magnetic flux of $4\delta$ threads through each plaquette in this abstract.}

To understand the structure of the Hamiltonian, we can focus on one of the blocks, say $H_+(k) = w N_k$ with the states $\ketup_+, \ketdn_+$.
For simplicity, let us first assume that $t=0$.
Furthermore, we introduce a cut-off in the dipolar tunneling such that only terms within one plaquette remain: horizontal, vertical and diagonal processes.
Then, we have
\begin{align}
    \ep_k &= 2 w_h \cos(ka), \\
    \eta_k^m &= -w_v + 2 w_d \cos(ka + m\delta).
\end{align}
Here, the term $w_h = w$ describes the horizontal (intra-chain) tunneling, $w_v = w \cot^3(\delta)$ is the vertical (inter-chain) coupling and $w_d = w \cos^3(\delta)$ is the strength of the diagonal (inter-chain) tunneling.
The angle $\delta$ is given by $\tan \delta = h/a$ (see \tref{setup}).
We can depict the model in this new basis by considering a ladder in an abstract space, where the upper chain is made up of $\ketup_+$ states and the lower chain is made up of $\ketdn_+$ states.
The resulting tunneling elements are shown in~\tref{creutz}.
Notice how the phases induced by the dipolar exchange interactions lead to the appearance of a \emph{constant} magnetic field with a flux of $4\delta$ per unit cell, determined entirely by the geometric angle of the original real-space model.
Using the explicit expressions for $\ep_k$ and $\et^m_k$, we can write the lower block of the Hamiltonian as
\begin{align}
    H_+(k) = &\begin{pmatrix}
        -w_v + 2w_d \cos(ka-2\delta) & 2w_h \cos(ka) \\
        2w_h\cos(ka) & -w_v + 2w_d \cos(ka+2\delta)
    \end{pmatrix} \\
    = &+\id \; \times (-w_v + 2w_d \cos(2\delta) \cos(ka)) \\
    &+\sigma_z \times 2w_d \sin(2\delta) \sin(ka) \\
    &+\sigma_x \times 2w_h \cos(ka).
\end{align}
This two-by-two sector of the full Hamiltonian is identical to a cross-linked ladder model in a magnetic field which has been introduced by Creutz~\cite{Creutz1999} (see also related work by Tovmasyan~\etal~\cite{Tovmasyan2013a}).
For the simplified case we have considered so far ($t=0$ and artificial cut-off), the parameters of the original model are given by $K=w_h=w, M=0$ and $r=w_d/w_h=\cos^3(\delta)$.
The magnetic flux per unit cell in the Creutz model is given by $2\theta$ which translates to $4\delta$ in our model (see \tref{creutz}).

\section{Topological structure}
The Creutz ladder model supports two perfectly flat bands for $M=0, r=1$ and $\theta=\pi/2$.
In our case this translates to the condition $w_h=w_d$ and $\delta=\pi/4$.

% [For M. Tovmasyan, E. P. L. van Nieuwenburg, and S. D. Huber, Phys. Rev. B 88, 220510 (2013), the parameters are $m=0, \delta=0, t=w_h$ and $\epsilon=w_d/w_h-1$]

% \fig{0.8}{dispersion-creutz}{Dispersion relation for the double-Creutz-ladder.
% The lower band has a flatness ratio of $\sim 2$. The plot on the right shows the spectrum of an open ladder with two edge states in each of the two band gaps.}

% \subsection{Symmetries}
% The model is time reversal symmetric with $\mathcal{T}=\sigma_x \mathcal{K}$:
% \begin{align}
%     \mathcal{T}H_+(k)\mathcal{T}^{-1}= H_+(-k)
% \end{align}
% For $w_v=0$ (energy offset) and $\delta=\pi/4$, the system is additionally particle-hole symmetric (This only works because of the cutoff!):
% \begin{align}
%     \mathcal{C}H_+(k)\mathcal{C}^{-1}= -H_+(-k)
% \end{align}
% with $\mathcal{C}= \sigma_z \mathcal{K}$. This leads to a chiral symmetry $\mathcal{S} = i\sigma_y$ which anti-commutes with the Hamiltonian.

% In total, at the point $\delta=\pi/4$, the system has $\mathcal{T}^2=1, \mathcal{C}^2=1, \mathcal{S}^2=1$ and is in symmetry class BDI.

% \subsection{Edge states}
% In the flat band limit, the edge state (on the left side) is given by
% \begin{align}
%     \ket{E}_+ = \ketup_+ + i \ketdn_+ = \upp + \dnm + i\dnp + i\upm
% \end{align}
% The edge state of the ``anti-binding'' ladder is given by
% \begin{align}
%     \ket{E}_- = \ketup_- + i \ketdn_- = \upp - \dnm + i\dnp - i\upm
% \end{align}


% \subsection{Adding \texorpdfstring{$\tbar$}{t-bar} terms}
% Adding NN and NNN terms with $t_+=t_-\ne 0$, we find an on-site-coupling:
% \begin{align}
%     \braketop{\downarrow_+}{H}{\uparrow_+} = -\tbar_v
% \end{align}
% as well as the following tunneling contributions
% \begin{align}
%     \bra{\uparrow_+}_{i}H\ket{\uparrow_+}_{i+1} &= -\tbar_h \\
%     \bra{\downarrow_+}_{i}H\ket{\downarrow_+}_{i+1} &= -\tbar_h \\
%     \bra{\uparrow_+}_{i}H\ket{\downarrow_+}_{i+1} &= -\tbar_d
% \end{align}
% The Creutz-ladder parameters are now modified:
% \begin{align}
%     M &= -\tbar_v \\
%     K &= w_h - \tbar_d\\
%     K r \ef{i \theta} &= w_d \ef{2i\delta} - \tbar_h
% \end{align}
% the last equation yields
% \begin{align}
%     r &= \sqrt{\frac{(w_d\cos 2\delta - \tbar_h)^2 + (w_d \sin 2\delta)^2}{(w_h-\tbar_d)^2}} \\
%     \theta &= \atan \frac{\sin 2\delta}{\cos 2\delta-\tbar_h/w_d}
% \end{align}

% \section{Basis transformation for the full Hamiltonian}
% Writing the full Bloch Hamiltonian in the basis $\ketup_-, \ketdn_-, \ketup_+, \ketdn_+$, we find
% \begin{align}
%     H(k) &=
%     \begin{pmatrix}
%         -\tbar\ep_k -w\et_k^{-2} & -\tbar\et_k^0 -w\ep_k & \mu+t\ep_k & t\et_k^0 \\
%         -\tbar\et^0_k - w\ep_k & -\tbar \ep_k - w\et_k^{+2} & t\et_k^0 & \mu + t\ep_k \\
%         \mu+t\ep_k & t\et_k^0 & -\tbar\ep_k +w\et_k^{-2} & -\tbar\et_k^0 +w\ep_k \\
%         t\et_k^0 & \mu + t\ep_k & -\tbar\et^0_k + w\ep_k & -\tbar \ep_k + w\et_k^{+2}
%     \end{pmatrix} \\
%     &= \id \otimes (-\tbar \, M_k) + \sigma_x \otimes (t M_k + \mu) - \sigma_z \otimes (w N_k)
% \end{align}
% where
% \begin{align}
%     M_k = \begin{pmatrix}
%         \ep_k & \et_k^0 \\
%         \et_k^0 & \ep_k
%     \end{pmatrix},\qquad
%     N_k = \begin{pmatrix}
%     \et_k^{-2} & \ep_k \\
%     \ep_k & \et_k^{+2}
%     \end{pmatrix}
% \end{align}

% \subsection{New representation of the symmetry operators}
% \begin{align}
%     \mathcal{P} &= \id \otimes \sigma_x \\
%     \mathcal{T} &= -\sigma_z \otimes \sigma_x \\
%     \mathcal{P}\mathcal{T} &= -\sigma_z \otimes \id \text{ (now diagonal)} \\
%     \mathcal{C} &= \sigma_y \otimes \sigma_x \\
%     \mathcal{S}_\mathcal{T} &= \sigma_x \otimes \id \\
%     \mathcal{S}_\mathcal{P} &= \sigma_y \otimes \id
% \end{align}

% \subsection{Time-reversal invariant point: two Creutz ladders}
% For $t=\mu=0$, the Hamiltonian block-diagonalizes into the two sectors corresponding to the two Creutz ladders. We have
% \begin{align}
%     H(k) = \id \otimes (-\tbar M_k) - \sigma_z \otimes (w N_k)
% \end{align}
% We focus on a single Creutz ladder (lower block). Then, we find
% \begin{align}
%     H_-(k) &= -\tbar M_k + w N_k \\
%            &= \bc{-\tbar \ep_k + \frac{w}{2} \bb{\et^{-2}_k + \et^{+2}_k}}\id + \frac{w}{2}\bb{\et^{-2}_k - \et^{+2}_k} \sigma_z + (-\tbar \et^0_k + w \ep_k) \sigma_x \\
%            &= \bb{-\tbar \ep_k + w \et^{s}_k} \id + w\et^{a}_k \sigma_z + (-\tbar \et^0_k + w \ep_k) \sigma_x
% \end{align}
% The band energies are given by
% \begin{align}
%     E_\pm = -\tbar \ep_k + w \et^s_k \pm \sqrt{(w \et^a_k)^2+(-\tbar\et^0_k+w\ep_k)^2}
% \end{align}

% \subsection{Flat band}
% The term $w \et^a_k$ is much smaller than $-\tbar\eta_k^0+w \ep_k$. We rotate into a new basis
% $\ket{\psi}_\pm = (\ketup_- \pm \ketdn_-)/\sqrt{2}$ where the role of $\sigma_x$ and $\sigma_z$ are interchanged. Now the Hamiltonian has the shape
% \begin{align}
%     H_-(k) &= \bb{ -\tbar \ep_k + w \et^{s}_k }\id +(-\tbar \et^0_k + w \ep_k) \sigma_z +  w\et^{a}_k \sigma_x
% \end{align}
% The small term is the coupling between the two bands. Ignoring this coupling, the lower band will be flat if
% \begin{align}
%     E_-(k) = -\tbar \ep_k + w \et^s_k - (-\tbar \et^0_k + w \ep_k) = E_-(0)
% \end{align}
% Rearranging, we find
% \begin{align}
%     \frac{\tbar}{w} = \frac{-E_-(0)/w + \et^s_k \pm \ep_k}{\ep_k \pm \et^0_k}
% \end{align}
% It turns out, that the function on the right hand side can be made quite flat for a certain value of $\tbar/w$.

% \section{Edge state properties}
% % \fig{0.9}{energy-diff}{Scaling of the energy difference between the two edge states, proportional to $1/L^3$.}
% \subsection{State}
% There are two states on each edge with approximate energies $\pm w_v$. In the following, we will focus on one particular edge states (with energy $-w_v$). In the presence of long-range tunneling,
% the amplitudes of each edge state decay with $1/x^3$ into the bulk.
% Correspondingly, the coupling $\sim \sum \frac{1}{x^3(L-x)^3}$ between the two edge states is proportional to $1/L^3$ (see \tref{scaling}).

% If the dipolar interaction is cutoff at a finite distance $R_c < L$, the edge state decays exponentially into the bulk.

% % \begin{figure}
% %     \ig{0.32}{scaling-cutoff120}
% %     \ig{0.32}{scaling-cutoff5}
% %     \ig{0.32}{scaling-randomOnsite}
% %     \caption{(a) Algebraic scaling of the energy difference between the left and the right edge state. (b) Exponential scaling for cutoff $5a$. (c) Splitting for random on-site potential which destroys inversion symmetry.\tlabel{scaling}}
% % \end{figure}

\fig{.7}{phasediag}{Topological phase diagram for the double Creutz ladder system as a function of $t/w$ and $\delta = \arctan(h/a)$. Thick lines indicate band touchings at the $k=0$ or $k=\pi/a$ point. The shaded areas have nontrivial winding numbers $\nu_\pm$ for the $H_+$-ladder, the $H_-$-ladder or both Creutz ladders.}

% \section{Zak phase}
% The Zak phase is defined by
% \begin{align}
%     \gamma_\nu = i\integralb{-\pi/a}{\pi/a}{k} \braketop{u_\nu(k)}{\partial_k}{u_\nu(k)}
% \end{align}
% where $\ket{u_\nu(k)}$ is the Bloch eigenstate of the $\nu$-th band.
% If a one-dimensional system is inversion symmetric, the Zak phase can only be~\cite{Zak1989} $\gamma=0$ or $\gamma=\pi$.

% \section{Minimal cutoff}
% Including just the necessary terms in the long-range tunneling, we have
% \begin{align}
%     \ep_k &= 2 \cos(ka) \\
%     \et^m_k &= \frac{1}{(h/a)^3} + \frac{2}{(l/a)^3}\cos(k a+ m\delta)
% \end{align}
% where $l^2=a^2+h^2$ and $\tan \delta = h/a$ (see~\cref{setup}).

