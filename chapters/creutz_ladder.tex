\chapter[Realizing the Creutz ladder model with dipolar spins]{Realizing the Creutz ladder model\texorpdfstring{\\}{ }with dipolar spins}
\label{creutz_ladder}

In this part we apply the ideas of \cref{topological_bands} to one-dimensional systems.
Using a lattice with a double-chain structure, we show how dipolar exchange interactions can lead to the simulation of a homogeneous magnetic field and a symmetry-protected topological phase.
The appearance of a constant magnetic field in a time-reversal invariant system is only possible because the system decomposes into two completely separate sectors.
Each of them acquires a magnetic field of equal strength but opposite sign, in a similar way to the quantum spin Hall effect~\cite{Kane2005a}.
Furthermore, we show that each sector is a realization of a cross-linked ladder model introduced by Creutz~\cite{Creutz1999,Bermudez2009,Tovmasyan2013a,Takayoshi2013,Sticlet2014}.
This model has a topological phase with a perfectly flat band and symmetry protected edge states.
We examine the properties of the double Creutz ladder and describe the modifications due to the long-range dipolar hoppings.

\section{Setup}

\fig{.7}{setup}{Setup for the realization of the Creutz ladder. One dipole is pinned at each lattice site of a double chain with lattice spacing $a$ in horizontal direction and a distance $h$ between the two chains. A unit cell at site $j$ includes both the upper and lower dipole. Horizontal ($h$), vertical ($v$) and diagonal ($d$) tunneling links are indicated for the idealized model with a cut-off $R_c=\sqrt{a^2+h^2}$ and $t=0$.}

Consider the one-dimensional system depicted in~\tref{setup}.
Single dipoles are located at each site of a double-chain lattice with spacing $a$ and a separation of $h$ between the two chains.
The level scheme is the one discussed in \cref{topological_bands}, but without any microwave field (see \refandpage{tb:fig1/fig1b}). Every dipole initially starts in the ground state $\ketz$ and can be excited into one of the two orbitals $\ketp$ or $\ketm$.
We are only interested in the single-excitation dynamics and use the notation $\ladderKet{\alpha}{}_j = \ket{\alpha}_{j,\text{upper}} \ket{0}_{j,\text{lower}}$ to indicate a state at lattice site $j$ where the dipole on the upper chain has been excited into the orbital $\alpha \in \{+, -\}$ and the dipole on the lower chain is still in the ground state.
Conversely, $\ladderKet{}{\alpha}_j$ describes an excitation on the lower chain.
Then, the four states $\upp_j, \upm_j, \dnp_j, \dnm_j$ define a complete local basis at each lattice site.
Introducing hard-core bosons for each of these states and transforming to momentum space in the same way as in \cref{topological_bands}, we find that the Bloch Hamiltonian takes the form
\begin{align}
    H(k) =
    \begin{pmatrix}
        -t \ep_k  & w \ep_k & -t \et^0_k & w \et^{-2}_k \\
        w \ep_k & -t \ep_k  & w \et^{+2}_k & -t \et^0_k \\
        -t \et^0_{k} & w \et^{+2}_{k}  & -t \ep_k  & w \ep_k \\
        w \et^{-2}_{k} & -t \et^0_{k} & w \ep_k & -t \ep_k
    \end{pmatrix}.
\end{align}
Here, $t$ is the orbital-preserving tunneling strength ($t \equiv t^+ = t^-$) and $w$ is the spin-changing tunneling rate, as defined in \cref{tb:tunnelingRates}.
The one-dimensional variant of the dipolar dispersion relation comes in two forms. The function
\begin{align}
    \ep_k &= a^3 \sum_{x\ne 0} \frac{\ef{i k x}}{|x|^3} = 2 \sum_{j>0} \frac{\cos(k a j)}{|j|^3}
\end{align}
includes all processes within a single chain and the function
\begin{align}
    \et^m_k &= a^3 \sum_{x} \frac{\ef{i k x + i m \phi_x}}{\bb{x^2 + h^2}^{3/2}}
    = \frac{i^m}{(h/a)^3} + \sum_{j>0} \frac{2\cos(ka j+m\phi_j)}{\bb{j^2+(h/a)^2}^{3/2}}
\end{align}
covers all inter-chain processes. Here, $\phi_x = \arg(x+ i h)=\phi_j=\arg(j+ i h/a)$ is the polar angle of the position $(x, h)$ of the dipole on the opposite chain.
Note that both functions are real-valued since $m = 0, \pm 2$. Moreover, $\ep_k=\ep_{-k}$ is symmetric in $k$ whereas $\et^m_k$ satisfies the relation $\et^m_k=\et^{-m}_{-k}$.

\section{Symmetries}
\tlabel{symmetries}
The described system has an inversion symmetry (or $180^\circ$ rotation symmetry), described by $H(-k) = \mathcal{P} H(k) \mathcal{P}^{-1}$, where the unitary operation $\mathcal{P} = \sigma_x \otimes \id$ flips the upper and lower chains.
In addition, the system is time-reversal symmetric, i.e.~$H(-k)=\mathcal{T} H(k) \mathcal{T}^{-1}$.
Here, time-reversal is described by the anti-unitary operator $\mathcal{T}=U_\mathcal{T} \mathcal{K}$, where $\mathcal{K}$ is complex conjugation and $\mathcal{U}_\mathcal{T} = \id \otimes \sigma_x$ is a unitary operator that exchanges the two orbitals.
The time-reversal operation satisfies $\mathcal{T}^2 = +\id$.
The combination of these two symmetries gives rise to an operator $\mathcal{P}\mathcal{T} = \sigma_x \otimes \sigma_x \, \mathcal{K}$ which commutes with the Bloch-Hamiltonian:
\begin{align}
    [\mathcal{P}\mathcal{T}, H(k)] = 0\qquad \Rightarrow \qquad [\sigma_x \otimes \sigma_x, H(k)]= 0.
\end{align}
The second commutation relation follows from the fact that $H(k)$ is real-valued.
Using the knowledge about the symmetry, we can block-diagonalize the Hamiltonian. To do so, we change to a basis which diagonalizes the operator $\sigma_x \otimes \sigma_x$. We define
\begin{align}
    \ketup_\pm = \frac{1}{\sqrt{2}} \bb{ \upp \pm \dnm }, \\
    \ketdn_\pm = \frac{1}{\sqrt{2}} \bb{ \dnp \pm \upm }.
\end{align}
Notice how these four states are invariant up to a phase under a combined flip of the chains $\ladderKet{\alpha}{} \leftrightarrow \ladderKet{}{\alpha}$ \emph{and} the orbitals $+ \leftrightarrow -$.
In the new basis $\ketup_-, \ketdn_-, \ketup_+, \ketdn_+$ the Hamiltonian takes the form
\begin{align} \tlabel{hblock}
    H(k) &=
    \begin{pmatrix}
        -t\ep_k -w\et_k^{-2} & -t\et_k^0 -w\ep_k &  &  \\
        -t\et^0_k - w\ep_k & -t \ep_k - w\et_k^{+2} &  &  \\
         &  & -t\ep_k +w\et_k^{-2} & -t\et_k^0 +w\ep_k \\
         &  & -t\et^0_k + w\ep_k & -t \ep_k + w\et_k^{+2}
    \end{pmatrix} \\
    &= - t \, \id \otimes M_k - w \, \sigma_z \otimes N_k
\end{align}
where the left side of the tensor product selects the block ($+$ or $-$ sign) and the right side selects $\ketup$ or $\ketdn$. The two blocks $H_\mp(k) = -t M_k \mp w N_k$ are given in terms of the two matrices
\begin{align}
    M_k = \begin{pmatrix}
        \ep_k & \et_k^0 \\
        \et_k^0 & \ep_k
    \end{pmatrix},\qquad
    N_k = \begin{pmatrix}
    \et_k^{-2} & \ep_k \\
    \ep_k & \et_k^{+2}
    \end{pmatrix}.
\end{align}

\section{Idealized model: mapping to the Creutz ladder}

\fig{.7}{creutz}{Tunneling links for the two states $\ketup_+$ and $\ketdn_+$. Notice that the two depicted chains live in an abstract space which is not to be confused with the real space of the original ladder. The tunneling along a single chain is determined by the \emph{diagonal} elements of the real-space model while the inter-chain hopping is given by the \emph{horizontal} elements. A constant magnetic flux of $4\delta$ threads through each plaquette or unit cell in this abstract space.}

To understand the structure of the Hamiltonian, let us first assume that $t=0$.
We can focus on one of the blocks, say $H_+(k) = w N_k$ with the states $\ketup_+, \ketdn_+$.
Furthermore, we introduce a cut-off $R_c = \sqrt{a^2+h^2}$ in the dipolar tunneling, such that only terms within one plaquette remain, see \tref{setup}.
We use the symbols $w_h = w$ to denote the horizontal (intra-chain) tunneling, $w_v = w \cot^3(\delta)$ for the vertical (inter-chain) coupling and $w_d = w \cos^3(\delta)$ for the strength of the diagonal (inter-chain) tunneling.
The angle $\delta$ is given by $\tan \delta = h/a$.

We can visualize the model in the new basis by considering a ladder in an abstract space, where the upper chain is made up of $\ketup_+$ states and the lower chain is made up of $\ketdn_+$ states.
The resulting system with tunneling elements between the new basis states is shown in~\tref{creutz}.
Notice how the phases induced by the dipolar exchange interactions lead to the appearance of a \emph{constant} artificial magnetic field with a flux of $4\delta$ per unit cell, determined entirely by the geometric angle of the original real-space model.

It turns out that the model in this abstract space is identical to a cross-linked ladder model in a magnetic field; a system that has been introduced by Creutz~\cite{Creutz1999}.
For the simplified case we have considered so far ($t=0$ and artificial cut-off), the parameters of the original model are given by $K=w_h=w, M=0$ and $r=w_d/w_h=\cos^3(\delta)$.
The magnetic flux per unit cell in the Creutz model is given by $2\theta$ which translates to $4\delta$ in our model.

By performing the summation for the dipolar dispersion relation up to the cut-off radius explicitly, we get
\begin{align}
    \ep_k &= 2 w_h \cos(ka), \\
    \eta_k^m &= -w_v + 2 w_d \cos(ka + m\delta).
\end{align}
Using these expressions, we can write the lower block of the Hamiltonian as
\begin{align} \tlabel{hplus}
    H_+(k) = &\begin{pmatrix}
        -w_v + 2w_d \cos(ka-2\delta) & 2w_h \cos(ka) \\
        2w_h\cos(ka) & -w_v + 2w_d \cos(ka+2\delta)
    \end{pmatrix} \\
    = &+\id \; \times (-w_v + 2w_d \cos(2\delta) \cos(ka)) \\
    &+\sigma_z \times 2w_d \sin(2\delta) \sin(ka) \\
    &+\sigma_x \times 2w_h \cos(ka).
\end{align}
From the expansion into the $\id, \sigma_z, \sigma_x$ components, we can directly get the dispersion relation
\begin{align} \tlabel{disp}
    E_\pm(k) = -w_v &+ 2w_d \cos(2\delta)\cos(ka) \\
                    &\pm \sqrt{ \bb{ 2w_d\sin(2\delta)\sin(ka) }^2 + \bb{ 2w_h\cos(ka) }^2 }.
\end{align}

\subsection{Perfectly flat bands}
The Creutz ladder supports two perfectly flat bands. To see this, we first set the displacement between the two chains equal to the lattice constant, i.e. $h=a$.
Then, the angle $\delta$ is given by $\pi/4$ which results in a flux of $\pi$ per unit cell.
In this case, \tref{disp} simplifies to
\begin{align}
    E_\pm(k) = -w_v \pm \sqrt{ \bb{ 2w_d\sin(ka) }^2 + \bb{ 2w_h\cos(ka) }^2 }.
\end{align}
We can see that the system has flat bands if $w_d = w_h$, in which case the energy is given by $E_\pm(k) = -w_v \pm 2 w_h$.
The dispersion relation for $w_d=w_h$ and $w_d = 2^{-3/2} w_h$ is shown in \tref{disp_single/disp_single}.

In the flat band case, the horizontal tunneling elements in \tref{creutz} are given by $i w_h$, whereas the cross-link tunneling elements are given by $w_h$.
This leads to a destructive interference of all paths going from $\ketup_j$ to $\ketup_{j \pm 2}$ or $\ketdn_{j\pm 2}$, as shown in \tref{interference} (we suppress the $+$ index on the states as we only work in a single block).
Consequently, the excitations are localized on single plaquettes. Each plaquette hosts two states
\begin{align}
    \ket{P_j^\pm} = \frac{1}{2} \bb{ i \ketup_j + \ketdn_j \pm \ketup_{j+1} \pm i \ketdn_{j+1} }.
\end{align}
with exact energies $E_\pm=\smallbraketop{P_j^\pm}{H}{P_j^\pm}=-w_v \pm 2 w_h$.

\doublefig{.42}{disp_single/disp_single}{ }{.4}{interference}{ }{\sfA~Dispersion relation for a single two-by-two block $H_+(k)$ for $\delta=\pi/4$ and $w_d = w_h$ (solid lines, perfectly flat bands) and $w_d = 2^{-3/2} w_h$, which is the generic value that is obtained for a simple setup where $\sqrt{2}$ is the distance along the diagonal. \sfB~Destructive interference of the two paths going from $j$ to $j+2$ in the flat band limit for $\delta=\pi/4$ and $w_d=w_h$, cf. original figure by Creutz~\cite{Creutz1999}. The indicated tunneling elements are in units of $w$.}


\subsection{Edge states}
A finite system of length $L$ has $2L$ lattice sites but only $L-1$ plaquettes. This means that only $2L - 2$ states can be described in terms of the plaquette states $\ket{P_j^\pm}$.
Consequently, two states are left over.
These are the two edge states at each end of the ladder. In the flat-band limit, the left-hand side edge state is given by \qu{half} a plaquette state
\begin{align}
    \ket{E}_+ = \frac{1}{\sqrt{2}} \bb{ \ketup_{1,+} + i \ketdn_{1,+} } = \frac{1}{2} \bb{ \upp_1 + \dnm_1 + i\dnp_1 + i\upm_1 }.
\end{align}
We can see that the excitation is shared among all four orbitals in the original basis.
The exact localization at the edge is only true for the flat-band limit.
In the following, we look at the full model including the long-range dipolar hopping.


\section{The full dipolar model}
Leaving the idealized model, we remove the artificial cut-off and also add the spin-preserving tunneling terms proportional to $t$.
Then, we can write the block $H_+(k)$ from \tref{hblock} as
\begin{align} \tlabel{hplusfull}
    H_+(k) &= -t M_k + w N_k \\
           % &= \bc{-t \ep_k + \frac{w}{2} \bb{\et^{-2}_k + \et^{+2}_k}}\id + \frac{w}{2}\bb{\et^{-2}_k - \et^{+2}_k} \sigma_z + (-t \et^0_k + w \ep_k) \sigma_x \\
           &= \bb{-t \ep_k + w \et^{s}_k} \id - w\et^{a}_k \sigma_z + (w \ep_k - t \et^0_k) \sigma_x.
\end{align}
We have introduced the symmetric and anti-symmetric combinations
\begin{align}
    \et^s = \frac{1}{2} \bb{ \et^2_k + \et^{-2}_{k} }, \qquad
    \et^a = \frac{1}{2} \bb{ \et^2_k - \et^{-2}_{k} }.
\end{align}
Then, the dispersion relation for the lower block is given by
\begin{align}
    E_\pm(k) = -t \ep_k + w \et^s_k \pm \sqrt{(w \et^a_k)^2+(w \ep_k - t \et^0_k)^2}.
\end{align}
By replacing $w$ with $-w$, we immediately get the dispersion relation for the upper block $H_-(k)$, namely
\begin{align}
    E'_\pm(k) = -t \ep_k - w \et^s_k \pm \sqrt{(w \et^a_k)^2+(w \ep_k + t \et^0_k)^2}.
\end{align}

% \subsection{New representation of the symmetry operators}
% \begin{align}
%     \mathcal{P} &= \id \otimes \sigma_x \\
%     \mathcal{T} &= -\sigma_z \otimes \sigma_x \\
%     \mathcal{P}\mathcal{T} &= -\sigma_z \otimes \id \text{ (now diagonal)} \\
%     \mathcal{C} &= \sigma_y \otimes \sigma_x \\
%     \mathcal{S}_\mathcal{T} &= \sigma_x \otimes \id \\
%     \mathcal{S}_\mathcal{P} &= \sigma_y \otimes \id
% \end{align}

\subsection{Topological structure}

\fig{.7}{phasediag}{Topological phase diagram for the double Creutz ladder system as a function of the angle $\delta = \arctan(h/a)$ and the ratio of tunneling rates $t/w$. The lines indicate band touchings at the $k=0$ or $k=\pi/a$ point for the $H_-$-ladder (blue) and $H_+$-ladder (green). The shaded areas have nontrivial winding numbers $\nu_\pm$ for one of the Creutz ladders or both of them.}

Following the strategy of \cref{tb:chernnumber}, \tref{hplusfull} can be written in the form $H_+(k) = n^0_k \cdot \id + \vec{n}_k \cdot \vecsigma$. Here, $n^0_k = \frac{1}{2} \operatorname{tr} H_+(k)$ is the diagonal part and $\vec{n}_k$ is given by
\begin{align}
    \vec{n}_k = \begin{pmatrix}
        w \ep_k - t \et^0_k \\
        0 \\
        -w \et^a_k
    \end{pmatrix}.
\end{align}
To determine the topological phase diagram, we look for points where $\vec{n}_k=0$.
The odd function $\et^a_k$ in the $z$-component can only be zero at $k = 0$ and $k=\pi/a$.
Consequently, the bandgap closes if $n^x_0 = 0$ or $n^x_{\pi/a} = 0$, leading to the two conditions
\begin{align}
    t/w  &= \frac{\eps_0}{\et^0_0} = \frac{2\zeta(3)}{\et^0_0}, \\
    t/w  &= \frac{\eps_{\pi/a}}{\et^0_{\pi/a}} = - \frac{3 \zeta(3)}{2 \et^0_{\pi/a}}.
\end{align}
Similar conditions hold for the $H_-$ block, where the signs are simply reversed.
If $t/w$ is not at one of the critical values, we can normalize the vector and define the winding number
\begin{align} \tlabel{winding}
    \nu = \frac{1}{2\pi} \integralb{-\pi/a}{\pi/a}{k} \bb{ \hat{n}^x_k \partial_k \hat{n}^z_k - \hat{n}^z_k \partial_k \hat{n}^x_k }.
\end{align}
For values $t/w$ between the two critical values, each ladder has a non-trivial winding number of $\nu = 1$.
The resulting topological phase diagram for both sectors is shown in \tref{phasediag}.
It has overlapping regions where both of the ladders are in the topologically nontrivial phase.


\sidefig{.4\textwidth}{disp_dipolar/disp_dipolar}{Dispersion relation of the full dipolar model for realistic parameters with $h/a = 0.72$ and $t/w = 1/3$. Both ladders (blue and green bands) are in the topological phase with $\nu=1$. Note that the bands are allowed to cross due to the block-diagonal structure of the full Hamiltonian. The point-spectrum on the right side shows the energies of a finite ladder with the same parameters and a length of $L=100$. Two edge states appear in the bandgap of each sector.}

\subsection{Symmetry protection}
In both blocks of the full Hamiltonian, the symmetries found in \tref{symmetries} are given by $\mathcal{P} = \sigma_x$ and $\mathcal{T}=\sigma_x \mathcal{K}$ with respect to the new basis.
For the winding number in \tref{winding} to be a well-defined topological index, we rely on the fact that the $\vec{n}_k$ vector lies in the $xz$ plane, i.e. $n_y(k)=0$. As we shall see shortly, this is a consequence of time-reversal symmetry and inversion symmetry.
For a Hamiltonian $H(k) = \vec{n}_k \cdot \vecsigma$, the two symmetries lead to the following conditions on the $\vec{n}_k$ vector:
\begin{align}
    &\mathcal{P}       &\quad         &\mathcal{T} \\
    n^x_{-k} &= +n^x_k, &\quad n^x_{-k} &= +n^x_k, \\
    n^y_{-k} &= -n^y_k, &\quad n^y_{-k} &= +n^y_k, \\
    n^z_{-k} &= -n^z_k, &\quad n^z_{-k} &= -n^z_k.
\end{align}
Consequently, $n^x_k$ needs to be an even function of $k$, $n^z_k$ needs to be an odd function of $k$ and $n^y_k = 0$ for all $k$.
If the condition $n^y_k = 0$ is violated by breaking one of the two symmetries, the winding number $\nu$ is no longer a useful index, as it can change without a closing of the bandgap.

\subsection{Symmetry classification}
One might wonder to which symmetry class our model belongs according to the general classification scheme~\cite{Schnyder2008,Kitaev2009,Ryu2010}.
We have seen that the full model is time-reversal symmetric with $\mathcal{T}^2 = + \id$.
In the language of the classification scheme, the inversion symmetry can also be regarded as a time-reversal symmetry with $\mathcal{T}_2 = \mathcal{P} \mathcal{K}$ and $\mathcal{T}^2_2 = +\id$.
This construction is possible because the Hamiltonian is real-valued.
Considering just a single time-reversal symmetry is sufficient, as the combination of two time-reversal symmetries always gives rise to a unitary symmetry~\cite{Ryu2010}.
In our case this is just the operator $\sigma_x \otimes \sigma_x$ which we used to block-diagonalize the Hamiltonian.
The full dipolar model consequently belongs to symmetry class AI, apparently without any topological index in one dimension~\cite{Schnyder2008,Kitaev2009,Ryu2010}.
Note that this is not in contradiction to the existence of the winding number defined above: the presence of inversion symmetry can lead to the appearance of additional symmetry-protected topological phases.
In fact, adding inversion symmetry to the class AI leads to a $\mathbb{Z}$ topological index~\cite{Chiu2014,Lu2014}.

At a certain fine-tuned point, the \emph{idealized} model becomes \qu{particle-hole} symmetric.
To see this, consider a single block $H_+(k)$ from \tref{hplus} at $\pi$-flux, i.e. $\delta = \pi/4$. Subtracting the constant energy shift $-w_v$, we have
\begin{align}
    H'_+(k) = H_+(k) + w_v \id = &\begin{pmatrix}
        2w_d \sin ka &   2w_h \cos ka \\
        2w_h \cos ka & - 2w_d \sin ka
    \end{pmatrix}.
\end{align}
This Hamiltonian satisfies $\mathcal{C}H'_+(k)\mathcal{C}^{-1}= -H'_+(-k)$ where $\mathcal{C}$ is the anti-unitary operator $\mathcal{C}= \sigma_z \mathcal{K}$ that squares to $+\id$.
In combination with time-reversal symmetry, this also leads to a chiral symmetry $\mathcal{S} = i\sigma_y$ which anti-commutes with the Hamiltonian.
Summarizing these results, we find that the idealized model at the point $\delta=\pi/4$ belongs to the symmetry class BDI, because $\mathcal{T}^2=1$ and $\mathcal{C}^2=1$ (same conclusion as Tovmasyan~\cite{Tovmasyan2013a}).
In a one-dimensional system, the BDI class has a $\mathbb{Z}$ topological index.
However, since the \qu{fake} particle-hole symmetry $\mathcal{C}$ only appears at a fine-tuned point in the presence of inversion symmetry, this topological index is symmetry-protected in the same way as the winding number of the full dipolar model.
This is very similar to the case of the Su-Schrieffer-Heeger model~\cite{Su1979}, which can also be classified as BDI, but only in the presence of inversion symmetry~\cite{Wang2013}.

\subsection{Edge state properties}

\sidefig{.5\textwidth}{scaling/scaling}{Scaling behavior of the energy difference $\Delta E$ between the two edge states. In the full dipolar model without any cut-off ($R_c=\infty$), the energy scales algebraically with $L^{-3}$ (green line is a fit to $\Delta E_1/L^3$). If a cut-off $R_c = 5a < L$ is introduced, the scaling turns exponential. For a disordered system with a random on-site potential $\mu_\text{rand}/w \approx 0.01$, the edge states split off with a constant $\Delta E$.}

\Tref{disp_dipolar/disp_dipolar} shows the dispersion relation of the full dipolar model in the region where $\nu_+ = \nu_- = 1$.
The point spectrum for a finite ladder of length $L$ shows the appearance of four edge states; two for each block of the Hamiltonian.
In the following, we focus on the two edge states with the lower energy.
In the presence of long-range dipolar hopping, the edge states are not exactly localized at the edge, but the amplitudes decay with $1/x^3$ into the bulk.
In consequence, the coupling between the two edge states is proportional to $1/L^3$.
This leads to an energy difference $\Delta E$ which scales like $1/L^3$ with the length of the ladder.
Conversely, the edge state amplitude decays exponentially into the bulk if the dipolar interaction is artificially cut off at a finite distance $R_c < L$, leading to an exponentially small energy difference.
Finally, if the inversion symmetry is broken, the edge states split off independently of $L$ with a constant energy offset $\Delta E$.
Such a symmetry breaking could be introduced by a random on-site potential, for example.
The energy scaling results are summarized in \tref{scaling/scaling}.

\subsection{Outlook}
We have shown that dipolar exchange interactions can lead to the appearance of a constant magnetic flux in a very simple setup on a double-chain lattice.
The artificial magnetic field can be tuned by changing a single geometric parameter: the height between the two chains.
Moreover, the system can be understood as two decoupled copies of a Creutz-ladder with opposite-sign parameters.
This model has a topological phase with edge states that are protected by inversion symmetry.
The many-body physics of our model can be understood by studying hard-core bosons on the Creutz-ladder.
In the short-range version, this has been subject of a work by Tovmasyan \etal~\cite{Tovmasyan2013a}.
