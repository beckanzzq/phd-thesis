\chapter*{Introduction}
\addcontentsline{toc}{chapter}{Introduction}

The history of topological materials is just a little over thirty years old.
A good point to start is the discovery of the quantized Hall conductance in two-dimensional semiconductor samples by von Klitzing in the early 1980s~\cite{Klitzing1980,Klitzing1992}.
He found that the Hall conductance develops plateaus as a function of the magnetic field which are exactly quantized in multiples of a fundamental constant that depends on the elementary charge and Planck's constant, independent of any material properties or external conditions.
Due to the high precision of the quantization levels, for which an explanation was given in the following years by Laughlin and Halperin~\cite{Laughlin1981,Halperin1982}, this effect immediately found applications in metrology as a direct measurement of the fine structure constant and as a standard for the unit of resistance.
The discovery by von Klitzing was awarded with the 1985 Nobel Prize in physics.

The first connection to topological properties was found by Thouless and others a few years after the discovery~\cite{Thouless1982,Niu1985,Kohmoto1985,Avron1985,Kohmoto1989,Bellissard1994,Avron2003}.
They found that the Hall conductance is directly related to a topological invariant called Chern number.
In much the same way that the number of \qu{handles} of a closed two-dimensional manifold can be calculated by an integration over its curvature, the Chern number of a Hamiltonian can be calculated by integrating its Berry curvature over a periodic two-dimensional configuration space.
Similar to the Gaussian curvature of the manifold, the Berry curvature of the quantum mechanical system quantifies the geometric changes of the wave functions under transport around closed loops~\cite{Berry1984,Zak1989}.
The connection of the quantized Hall conductance to a topological invariant manifests itself in the robustness of the physical effect against local perturbations.

A related, but considerably more complex phenomenon was experimentally discovered by Tsui, Störmer and Gossard in 1982 at even lower temperatures in cleaner samples~\cite{Tsui1982}.
They found that the Hall conductance could additionally develop plateaus at certain fractional values of the filling factor, the ratio between the number of electrons and the number of magnetic flux quanta threading through the sample.
These plateaus correspond to fractionally filled Landau levels and could not be explained by a single-particle treatment.
Once again, it was Laughlin who was able to explain the phenomenon~\cite{Laughlin1983}, winning him the 1998 Nobel Prize in physics together with Tsui and Störmer.
He found that the two-dimensional electron gas condenses into a new state of matter, a quantum fluid with fractionally charged excitations and anyonic statistics.
This strongly correlated state of matter is an example for a topologically ordered state, a state with a ground state degeneracy that depends on the topology of the underlying space and is robust against local perturbations~\cite{Wen1990,Wen1995}.
The structure of some fractional quantum Hall states still remains unexplained.
The most prominent example is the even-denominator state at a filling of $\sfrac{5}{2}$ that was experimentally observed as early as 1987 by Willet \etal~\cite{Willett1987}.
Particular interest in this state draws from work by Moore and Read~\cite{Moore1991}, suggesting that this state might give rise to quasiparticles with non-Abelian statistics.
Interchange of non-Abelian anyons leads to a change in the ground state manifold of the system. This property can be utilized for fault-tolerant quantum computation, an idea that has been proposed by Kitaev in 1997~\cite{Kitaev2003}

Fundamental questions about the nature of these states as well as their prospective use in topological quantum computation spur the research in this field today.
Traditional experiments with semiconductor samples remain challenging due to immense requirements on the sample quality, low temperatures and high magnetic fields.
With the turn of the century and the advent of ultracold gases experiments, new ideas how to reach the Quantum Hall regime emerged.
Unmatched control over system parameters as well as the ability to manipulate and observe on the single-particle level turn these systems into an optimal platform to advance our understanding in the field of Quantum Hall physics.
A fundamental problem appears when trying to emulate the effect of the magnetic field.
Electrically neutral atoms clearly do not couple to the magnetic vector potential in the way that electrons do.
Various solutions to this problem have been proposed and experimentally implemented.
Following an analogy that goes back to ideas by Larmor in 1933, it is possible to use a rapid rotation to induce an effective magnetic field for the neutral particles.
In the two-dimensional system, the frequency of rotation corresponds to the effective magnetic field strength parametrized by the cyclotron frequency.
Likewise, the Coriolis force is in one-to-one correspondence with the Lorentz force.
Starting in the early 2000s, experiments in this respect have advanced over the years~\cite{Schweikhard2004,Bretin2004,Cooper2008,Fetter2009}.

An alternative route was followed by Haldane~\cite{Haldane1988}.
In 1988, he proposed a lattice model with broken time-reversal symmetry that showed a quantum Hall effect without the requirement of Landau levels that would be generated by an external magnetic field.
The Haldane model utilizes complex tunneling phases that respect the symmetry of the lattice and generate a topological band structure.
It is a showcase for a class of materials called Chern insulators.
They behave similar to ordinary band insulators, but have conducting states at the edge of the material: a physical manifestation of their non-trivial Chern number~\cite{Hatsugai1993}.
For charged particles, the required complex tunneling phases are connected to the external magnetic field through a Peierls substitution~\cite{Peierls1933}.
In this regard, synthetic magnetic fields can be created for neutral particles by realizing complex tunneling phases.
Powerful approaches are optical flux lattices~\cite{Cooper2011}, laser-assisted tunneling~\cite{Aidelsburger2011,Aidelsburger2013,Miyake2013} or lattice shaking methods~\cite{Struck2012,Struck2013}.
The latter has recently been used by Jotzu \etal to realize the Haldane \qu{toy model} with ultracold fermions in an optical lattice~\cite{Jotzu2014}.

Finally, a third strategy is to use spin-orbit coupling techniques~\cite{Lin2011,Cheuk2012,Wang2012,Hamner2014,Jimenez-Garcia2015} to realize topological phases.
The interplay between external and internal degrees of freedom can lead to similar phenomena.
In 2005, Kane and Mele showed that spin-orbit coupled electrons in graphene can realize a topological system which encapsulates two time-reversed copies of Haldane's model~\cite{Kane2005a,Kane2005}.
The resulting arrangement is an example for a time-reversal invariant topological insulator with a quantum spin Hall effect where the two spin-components have a Hall conductance with opposite sign~\cite{Qi2011,Hasan2010}.
A physical realization in semiconductor quantum wells was proposed by Bernevig \etal in 2006~\cite{Bernevig2006a,Bernevig2006b} and experimentally demonstrated by König \etal one year later~\cite{Konig2007}.

% this thesis is concerned with

% dipolar spin systems
% - Hazzard2014
% - Barredo2014
% polar molecules
%
% experimental works on realizing spin systems (non-dipolar):~\cite{Fukuhara2013,Simon2011}
% dipolar:~\cite{DePaz2013}
% new effects in spin systems due to dipolar:~\cite{Avellino2006,Hauke2010,Peter2012b,Hazzard2014}
