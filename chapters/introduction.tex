\chapter*{Introduction}
\addcontentsline{toc}{chapter}{Introduction}

% Resources
% - http://benasque.org/2014numerical/talks_contr/021_BenasqueBergholtzPart1.pdf
% - Bernevig book (cite)
% http://www.uam.es/personal_pdi/ciencias/alevyyey/FMC/phystoday.pdf
% http://www.nobelprize.org/nobel_prizes/physics/laureates/1998/advanced-physicsprize1998.pdf

%%% History
The history of topological materials is just a little over thirty years old.
A good point to start is the discovery of the quantized Hall conductance in two-dimensional semiconductor samples by von Klitzing in the early 1980s~\cite{Klitzing1980,Klitzing1992}.
He found that the Hall conductance develops plateaus as a function of the magnetic field which are exactly quantized in multiples of a fundamental constant that depends on the elementary charge and Planck's constant, independent of any material properties or external conditions.
Due to the high precision of the quantization levels, for which an explanation was given in the following years by Laughlin and Halperin~\cite{Laughlin1981,Halperin1982}, this effect immediately found applications in metrology as a direct measurement of the fine structure constant and a standard for the unit of resistance.
The discovery was awarded with the 1985 Nobel Prize in physics.

The first connection to topological properties was found by Thouless and others a few years after the discovery~\cite{Thouless1982,Niu1985,Kohmoto1985,Avron1985,Kohmoto1989,Bellissard1994,Avron2003}.
Nowadays, integer quantum Hall samples are seen as examples of materials called topological insulators. They are insulating in much the same way as an ordinary insulator, but have conducting states at the edge of the material that contribute to the quantized Hall conductance.
What Thouless found was that the Hall conductance is directly related to a topological invariant called Chern number.
In much the same way that the number of \qu{handles} of an ordinary two-dimensional manifold can be calculated by an integration over its curvature, the Chern number can be calculated by integrating the Berry curvature over a periodic two-dimensional configuration space of the system. 

A related, but considerably more complex phenomenon was experimentally discovered by Tsui, Störmer and Gossard in 1982 at even lower temperature in cleaner samples~\cite{Tsui1982}.
They found that the Hall conductance could additionally develop plateaus at certain fractional values of the filling factor, the ratio between the number of electrons and the number of magnetic flux quanta threading through the sample.
These plateaus correspond to fractionally filled Landau levels and could not be explained by a single-particle treatment.
It was again Laughlin who was able to explain the phenomenon, winning him the 1998 Nobel Prize in physics together with Tsui and Störmer.
What he found was that the two-dimensional electron gas condensed into a new state of matter: a quantum fluid with fractionally charged excitations.

% 5/2 state: measured in 1987 and still remains unexplained -> need for model systems
\cite{Willett1987}

% no magnetic field -> Haldane model "quantum anomalous Hall effect"
\cite{Haldane1988}

% topological quantum computation
\cite{Kitaev2003}

% TRI topological insulators
\cite{Kane2005a,Kane2005,Hasan2010}

% experimental discovery of spin-Hall effect:
Zhang, Bernevig (science)


%%% Realization in ultracold systems

% realizing artificial magnetic fields (experiments)
see dip fermions references + update
Spielman: PRL 114, 125301 (2015)

% reaching the strong regime
\cite{Aidelsburger2011,Aidelsburger2013,Miyake2013}

% realizing topological states:
Bloch: Berry phase experiment
Haldane model \cite{Jotzu2014}

% dipolar spin systems
% - Hazzard2014
% - Barredo2014



% idea behind topological states: analogy with Gauss bonnet
% experimental works on realizing spin systems (non-dipolar):~\cite{Fukuhara2013,Simon2011}
% dipolar:~\cite{DePaz2013}
% new effects in spin systems due to dipolar:~\cite{Avellino2006,Hauke2010,Peter2012b,Hazzard2014}
