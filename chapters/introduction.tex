\chapter*{Introduction}
\addcontentsline{toc}{chapter}{Introduction}

% Resources
% - http://benasque.org/2014numerical/talks_contr/021_BenasqueBergholtzPart1.pdf
% - Bernevig book (cite)

%%% History
The history of topological materials is just a little over thirty years old. In the early 1980s,
Klitzing discovered the quantized Hall resistance in two-dimensional semiconductor samples.

% Klitzing
\cite{Klitzing1980,Klitzing1992}

% single particle explanation QHE
\cite{Laughlin1981,Halperin1982}

% 'topological' in QHE:
\cite{Niu1985,Thouless1982,Kohmoto1985,Kohmoto1989}

% FQHE
\cite{Tsui1982}

% 5/2 state: measured in 1987 and still remains unexplained -> need for model systems
\cite{Willett1987}

% Haldane model "quantum anomalous Hall effect"
\cite{Haldane1988}

% topological quantum computation
\cite{Kitaev2003}

% TRI topological insulators
\cite{Kane2005a,Kane2005,Hasan2010}

% experimental discovery of spin-Hall effect:
Zhang, Bernevig (science)

%%% Realization in ultracold systems

% realizing artificial magnetic fields (experiments)
see dip fermions references + update
Spielman: PRL 114, 125301 (2015)

% reaching the strong regime
\cite{Aidelsburger2011,Aidelsburger2013,Miyake2013}

% realizing topological states:
Bloch: Berry phase experiment
Haldane model \cite{Jotzu2014}

% dipolar spin systems
% - Hazzard2014
% - Barredo2014



% idea behind topological states: analogy with Gauss bonnet
% experimental works on realizing spin systems (non-dipolar):~\cite{Fukuhara2013,Simon2011}
% dipolar:~\cite{DePaz2013}
% new effects in spin systems due to dipolar:~\cite{Avellino2006,Hauke2010,Peter2012b,Hazzard2014}
