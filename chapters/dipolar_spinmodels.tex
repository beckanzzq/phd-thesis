\chapter{Introduction: dipolar spin models}
\label{dipolar_spinmodels}

% \section{Dipolar spin systems}
% \subsection{Lattice structures}
% \subsubsection{One-dimensional}
% \subsubsection{Two-dimensional}

% \section{Topological band structures}
% \subsection{Chern number}
% \subsection{Edge states}
% \subsection{Quantum Hall effect}

\section{Dipolar systems: physical realizations}
\todo{
    This section has several goals. First, we describe a common framework for deriving the effective
    models which are subject of \cref{anomalous_behavior,dipolar_fermions,topological_bands} of this thesis.
}
In this section we briefly review some of the possible realizations for dipolar spin
systems in ultracold atomic and molecular systems and discuss some of the similarities as well as important differences.

Proposals for simulating spin systems
\begin{itemize}
    \item with polar molecules \cite{Micheli2006,Hauke2010,Gorshkov2011,Gorshkov2011b,Peter2012b,Syzranov2014,Peter2014}
\end{itemize}

\todo{NV centers, Rydberg atoms, polar molecules, dipolar atoms}

\subsection{Polar molecules and Rydberg atoms}
While we focus on polar molecules, the physics and energy scale are very similar
in Rydberg atoms.
\todo{Rotor states, DC and AC dressing}

KRb \cite{Ni2008b,Ospelkaus2010}
NaK \cite{Wu2012a}
NaLi \cite{Heo2012}

in optical lattice \cite{Chotia2012}

seeing long-range dipolar effects:
\cite{Yan2013}

dipolar effects in Rydberg: \cite{Westermann2006,Nipper2012b}

\fig{.8}{pm-levelscheme}{pm-levelscheme}{kjsdf}

\subsection{Atoms with high magnetic moments}
review \cite{Lahaye2009}

Chromium \cite{Griesmaier2005a,Beaufils2008}, fermionic \cite{Naylor2015}
Dysprosium \cite{Lu2011}, fermionic \cite{Lu2012}
erbium \cite{Aikawa2012} fermionic \cite{Aikawa2014}

\todo{permanent dipole moment, hyperfine ground state, Lande factor}


\section{Dipole-dipole interaction}
The aim of this introductory section is to rewrite the familiar interaction between two dipoles $\vec{d}_i$ and $\vec{d}_j$ at positions $\vecR_i, \vecR_j$,
\begin{align} \tlabel{ddint}
\Hdd(\vecR_{ij}) = \frac{\kappa}{\absv{\vecR_{ij}}^3} \bc{ \vecd_i \cdot \vecd_j -3 (\vecd_i \cdot \hat{\vecR}_{ij})(\vecd_j \cdot \hat{\vecR}_{ij}) }
\end{align}
in a spherical tensor representation~\cite{Micheli2007,Gorshkov2011c} which will be useful throughout this thesis.
\Tref{ddint} is given in terms of the vector $\vecR_{ij} = \vecR_j - \vecR_i$ and its normalized form $\hat{\vecR}_{ij}$.
For electric (magnetic) dipoles, the constant prefactor is given by $\kappa = 1/4\pi\epsilon$ ($\kappa=\mu_0/4\pi$).
We focus on the distance-independent part $D_{ij} \equiv \Hdd \absv{\vecR_{ij}}^3/\kappa=\vecd_i \cdot \vecd_j -3 (\vecd_i \cdot \hat{\vecR}_{ij})(\vecd_j \cdot \hat{\vecR}_{ij})$ of the dipole-dipole interaction which can be written in terms of the spherical tensor $T^2(\vecd_i, \vecd_j)$ of rank two with components
\begin{align}
T^2_0(\vecd^i, \vecd^j) &= \frac{1}{\sqrt{6}} (d_i^+ d_j^- + 2d_i^0 d_j^0 + d_i^{-} d_j^{+}), \\
T^2_{\pm 1}(\vecd^i, \vecd^j) &= \frac{1}{\sqrt{2}} (d_i^{\pm} d_j^0 + d_i^0 d_j^{\pm}), \\
T^2_{\pm 2}(\vecd^i, \vecd^j) &= d_i^{\pm} d_j^{\pm} \vphantom{\frac{1}{42}}.
\end{align}
It is constructed from the two dipole moments which are themselves rank-one tensors with spherical components
\begin{align} \tlabel{sphericalcomp}
d^0_i = d^z_i, \qquad d_i^\pm = \mp\frac{1}{\sqrt{2}}\bb{d_i^x\pm i d_i^y}.
\end{align}
Using this, the dipole-dipole interaction can be written as a contraction of two rank-two tensors \cite{Brown2003}, leading to
\begin{align}
    D_{ij} &= -\sqrt{6} \: T^2(C)\cdot T^2(\vec{d_1},\vec{d_2})\\
           &= -\sqrt{6}  \sum_{m=-2}^{2} (-1)^m C^2_{-m}(\theta,\phi) \, T^2_m(\vecd_i,\vecd_j).
\end{align}
Here, $C^l_m(\theta,\phi)=\sqrt{\frac{4\pi}{2l+1}} Y^l_m(\theta,\phi)$ are the modified (unnormalized) spherical harmonics and $\theta = \theta_{ij}, \phi = \phi_{ij}$ are the spherical angles of the vector $\hat{\vecR}_{ij}$ in the coordinate system of the quantization axis. We can expand this expression to get
\begin{align}\tlabel{doperator}
    D_{ij} &= (1-3\cos^2 \theta) \BC{ d_i^0 d_j^0 + \frac{1}{2}\Bb{d_i^+ d_j^- + d_i^- d_j^+} } \\
           &\quad -\frac{3}{\sqrt{2}}\sin\theta \,\cos\theta\, \BC{ \Bb{ d_i^0 d_j^- + d_i^- d_j^0 } \ef{+i\phi} - \Bb{ d_i^0 d_j^+ + d_i^+ d_j^0 } \ef{-i\phi} } \\
           &\quad -\frac{3}{2}\sin^2\theta \, \BC{ \ef{+2i\phi} d_i^- d_j^- + \ef{-2i\phi} d_i^+ d_j^+ }.
\end{align}
It is worth noting that the $T^2_{m=0}(\vecd_i, \vecd_j)$-terms in the first row conserve the ``internal'' angular momentum while the $m=1$ ($m=2$) terms in the second (third) row increase or decrease the internal angular momentum by one (two) quanta.

For most applications, we will be concerned with two-dimensional systems where the dipoles are aligned perpendicular to the plane. Then, the dipoles are also perpendicular to the interconnecting axis $\vecR_{ij}$, implying $\theta = \pi/2$. In this case, the $m=1$ terms drop out and the tensorial part reduces to
\begin{align} \tlabel{dop2d}
    D^\text{(2D)}_{ij} = d_i^0 d_j^0 + \frac{1}{2}\bb{d_i^+ d_j^- + d_i^- d_j^+} -\frac{3}{2} \bb{ d_i^- d_j^- \ef{+2i\phi} + d_i^+ d_j^+ \ef{-2i\phi} }
\end{align}
For a one-dimensional geometry there are two high-symmetry geometries. If the dipoles are perpendicular to the lattice direction (say, the $x$-direction) we can set $\phi=0$, further simplifying $D_{ij}$ to
\begin{align}
    D_{ij}^{(1D,\perp)} &= d_i^0 d_j^0 + \frac{1}{2}\bb{d_i^+ d_j^- + d_i^- d_j^+} -\frac{3}{2} \bb{d_i^- d_j^- + d_i^+ d_j^+} \\
 &= d_i^z d_j^z + d_i^y d_j^y - 2 d_i^x d_j^x.
\end{align}
Conversely, if the dipoles point along the lattice direction, we can set $\theta=0$ in \tref{doperator} to get
\begin{align}
    D^{(1D,\parallel)}_{ij} &= -2 d_i^0 d_j^0 - \bb{d_i^+ d_j^- + d_i^- d_j^+} = -2 d_i^z d_j^z+ d_i^x d_j^x + d_i^y d_j^y
\end{align}

\subsection{Tilted electric field geometry}
\fig{.5}{lattice-geometry}{lattice-geometry}{Illustration of the relevant axes and angles. The lattice lies in the $xy$ plane while the electric field is tilted from the $z$ axis by an angle $\Theta_0$ and rotated around the $z$ axis by an angle $\Phi_0$ with respect to the $x$ axis. The direction of the vector $\vec{R}$, connecting two dipoles, is determined by the polar angle $\Phi$.}

\noindent
Here we consider a more general situation (see \tref{lattice-geometry}) with a two-dimensional system in the $xy$ plane and an DC electric field $\vec{E}$ pointing in an arbitrary direction~\cite{Gorshkov2011c}, determined by the spherical angles $\Theta_0, \Phi_0$:
\begin{align}
    \vec{\hat{E}}=\begin{pmatrix}
        \sin\Theta_0\cos\Phi_0 \\
        \sin\Theta_0\sin\Phi_0 \\
        \cos\Theta_0
    \end{pmatrix}.
\end{align}
As before, we are interested in the interaction between two dipoles which are separated
by the vector
\begin{align}
\vecR_{ij} = \begin{pmatrix}R_{ij} \cos\Phi_{ij} \\ R_{ij} \sin\Phi_{ij} \\ 0\end{pmatrix}.
\end{align}
For the angle $\theta$ between the dipole orientation $\vec{\hat{E}}$ and the interconnection line between the dipoles $\hat{\vecR}_{ij}$, we find the relation
\begin{align}
    \cos\theta = \vec{\hat{E}} \cdot \hat{\vecR}_{ij} &= \sin\Theta_0\bb{\cos\Phi_0\cos\Phi_{ij}+\sin\Phi_0\sin\Phi_{ij}}\\
                                                      &=\sin\Theta_0\cos(\Phi_{ij}-\Phi_0).
\end{align}
With the difference $\bar\Phi=\Phi_{ij}-\Phi_0$, we can express the relevant terms in the dipole-dipole interaction as
\begin{align} \tlabel{fmfunctions}
    f_0(\Theta_0, \bar \Phi)&\equiv 1-3\cos^2\theta = 1-3\sin^2\Theta_0\cos^2\bar\Phi, \vphantom{\bb{\bar \Phi}} \\
    f_1(\Theta_0, \bar \Phi)&\equiv \sin\theta\cos\theta \ef{i\phi} = \sin\Theta_0 \cos\bar\Phi \bb{\cos\Theta_0 \cos\bar\Phi + i \sin \bar\Phi}, \\
    f_2(\Theta_0, \bar \Phi)&\equiv\sin^2\theta \ef{2i\phi} = \bb{\cos\Theta_0 \cos\bar\Phi + i \sin \bar\Phi}^2.
\end{align}
% which are easily seen to reduce to the old expressions in the case $\Theta_0=0$, implying $\theta=\pi/2$ and $\bar\Phi=\Phi_{ij}$.
In total, the tensorial part of the dipole-dipole interaction from \tref{doperator} is given by
\begin{align} \tlabel{tensortilted}
    D_{ij}(\Theta_0, \bar\Phi) = f_0(\Theta_0, \bar \Phi) &\big[ d_i^0 d_j^0 + \frac{1}{2}\bb{d_i^+ d_j^- + d_i^- d_j^+} \big] \\
      \qquad-\frac{3}{\sqrt{2}} &\Big[ f_1(\Theta_0, \bar \Phi)  (d_i^0 d_j^- + d_i^- d_j^0) - f_1(\Theta_0, -\bar \Phi) (d_i^0 d_j^+ + d_i^+ d_j^0) \Big]\\
      \qquad-\frac{3}{2} &\Big[ f_2(\Theta_0, \bar \Phi) d_i^- d_j^- + f_2(\Theta_0, -\bar \Phi) d_i^+ d_j^+\Big]
\end{align}
% The dipole-dipole interaction is symmetric under spatial inversion, implying:
% \begin{align}
%     f_m(\Theta_0, \bar \Phi + \pi) = f_m(\Theta_0, \bar \Phi).
% \end{align}
% Under complex conjugation, we find
% \begin{align}
%     f_m^*(\Theta_0, \bar \Phi) = f_m(\Theta_0, -\bar \Phi)
% \end{align}
% In addition, we can write
% \begin{align}
%     f_1(\Theta_0, \bar \Phi) = \frac{1}{2} \sin \Theta_0 \bc{\cos \Theta_0 (1+\cos 2\bar \Phi) + i \sin 2 \bar \Phi}
% \end{align}
% For small angles $\Theta_0$, we can rewrite
% \begin{align}
%     f_0(\Theta_0, \bar \Phi) &= 1 \\
%     f_1(\Theta_0, \bar \Phi) &= \frac{1}{2} \Theta_0 \bc{1+\ef{2 i \bar \Phi}}=\Theta_0 \cos \bar \Phi \ef{i \bar \Phi}\\
%     f_2(\Theta_0, \bar \Phi) &= \ef{2 i \bar \Phi}
% \end{align}
% Defining $\Theta_0^* = \pi/2 - \theta_\text{magic}=\text{asin}(1/\sqrt{3})$, we have
% \begin{align}
%     f_0(\Theta_0^*, \bar \Phi) = 1-\cos^2\bar \Phi
% \end{align}

% \subsubsection{In-plane electric field}
% For $\Theta_0=\pi/2$, the expressions above reduce to
% \begin{align}
%     1-3\cos^2\theta &= 1-3\cos^2\bar\Phi \\
%     \sin\theta\cos\theta \ef{\pm i\phi} &= \pm \frac{i}{2} \sin 2\bar\Phi \\
%     \sin^2\theta \ef{\pm 2i\phi} &= - \sin^2 \bar\Phi
% \end{align}

% \subsubsection{One-dimensional setup}
% For a 1D setup we can set $\Phi_0=0$. The angle $\Phi$ is limited
% to the values $0$ or $\pi$, indicated by $\sin\Phi=0$ and
% $\cos\Phi=\pm 1=\text{sign}(X)$. Then, the expressions above reduce to
% \begin{align}
%     1-3\cos^2\theta &= 1-3\sin^2\Theta_0 \\
%     \sin\theta\cos\theta \ef{\pm i\phi} &= \sin\Theta_0 \cos\Theta_0 \\
%     \sin^2\theta \ef{\pm 2i\phi} &= \cos^2 \Theta_0
% \end{align}


\section{Choosing a particular model}
The different models in this thesis will be largely determined by a particular choice of internal states.
In addition to choosing a certain set of atomic or molecular states, external fields can modify these states..

in the following we look at three different choices. each of them is a rise in complexity, but also intruduces one additional feature.


\fig{.35}{dipole-moments}{dipole-moments}{Internal rotational states of a dipole with a ground state $\ketz$ and excited state $\keto$ with $M=0$, as well as two states $\ketpm$ with $M=\pm 1$. The relevant static dipole moments (blue) and transition dipole moments (red) are shown. The plus sign is for the transitions going \qu{upwards} and the minus sign for transitions \qu{downwards}.}


\subsection{Two-level dipoles: Ising and XY interacations}
Here, we assume that we have chosen two states, denoted as $\ketz$ and $\keto$, which have the
same $M$ quantum number (see~\tref{dipole-moments}).
Then, only the $d^0_id^0_j$ part of the dipolar interaction is relevant.
We define the (transition) dipole elements
\begin{align*}
    d_0 = \smallbraketop{0}{d^0}{0},\qquad
    d_1 = \smallbraketop{1}{d^0}{1},\qquad
    q_1 = |\smallbraketop{1}{d^0}{0}| = |\smallbraketop{0}{d^0}{1}|.
\end{align*}
Then, using the projectors $P^0 = \ketbra{0}{0}$ and $P^1 = \ketbra{1}{1}$, we can express the
dipole-dipole interaction \labeltref{dop2d} in the $\ketz, \keto$ subspace as
\begin{align*}
    \Hdd_{ij} &= \frac{\kappa}{R_{ij}^3} \bc {d_1^2 P^1_i P^1_j + d_0^2 P^0_i P^0_j + d_1d_0 (P^1_i P^0_j + P^0_i P^1_j) + q_1^2 (\splus_i \sminus_j + \sminus_i \splus_j) }.
\end{align*}
Here, we have neglected all processes which are not energy-conserving,
i.e. terms that do not conserve the number of excitations, and would
cancel out in the rotating wave approximation.
Using $\sz = P^1 - P^0$ and $\id = P^1 + P^0$ we find
\begin{align*}
    \Hdd_{ij} &= \frac{\kappa}{R_{ij}^3} \Big[ \frac{(d_1-d_0)^2}{4} \sz_i \sz_j  + q_1^2 (\splus_i \sminus_j + \sminus_i \splus_j)\\
              &\qquad\qquad\qquad + \frac{d_1^2 - d_0^2}{4} (\sz_i + \sz_j)  + \frac{(d_1+d_0)^2}{4} \Big].
\end{align*}
Here, the first two terms describe Ising- and XY-type interactions between the two-level dipoles. The third term is equivalent to a magnetic field in $z$ direction and the last term is just a constant energy offset\footnote{Note, however, that the \qu{magnetic field} and the constant offset depend on the positions of all other dipoles. If the system is not translationally invariant, these terms can play a role.}.
% \begin{align*}
% D_\text{int}&=\frac{(d_1-d_0)^2}{4} \sigma_z \sigma_z + \absvsq{d_{\up}} (\sigma_+ \sigma_- + \sigma_- \sigma_+)\\
% &= (d_1-d_0)^2 S_z S_z + 2\absvsq{d_{\up}} (S_x S_x + S_y S_y)
% \end{align*}

\todo{
    Introducing the spin $\sfrac{1}{2}$ operators, we can rewrite this to
\begin{align*}
    \Hdd &= \frac{1}{2} \sum_{i\ne j} \Hdd_{ij} = \sum_{i\ne j} \frac{\kappa}{R_{ij}^3} \Big[ \frac{(d_1-d_0)^2}{2} S^z_i S^z_j  + q_1^2 (S^x_i S^x_j + S^y_i S^y_j) \Big] \\
         &= \sum_{i\ne j} \frac{a^3}{R_{ij}^3} \Big[ \frac{\kappa (d_1-d_0)^2}{2a^3} S^z_i S^z_j  + \frac{\kappa q_1^2}{a^3} (S^x_i S^x_j + S^y_i S^y_j) \Big].
\end{align*}
Then, $J\cos\theta = \kappa (d_1-d_0)^2/2a^3$ and $J\sin\theta = \kappa q_1^2/a^3$. Thus,
$\tan \theta = 2q_1^2/(d_1-d_0)^2$.
}

\subsection{Two-level dipoles with \texorpdfstring{$\Delta M = 1$}{dM = 1}}
In this related case, we replace the upper state $\keto$ by the $M=1$ state $\ketp$ (see \tref{dipole-moments}) and proceed similar as before.
The relevant dipole matrix elements are\footnote{The sign difference in the last term is due to $\smallbraketop{0}{d^-}{+} = \smallbraketop{+}{(d^-)^\dag}{0}^* = \smallbraketop{+}{- d^+}{0}^*$.Note that we can choose the phases of $\ketz$ and $\ketp$ freely, allowing us to choose real values for the transition dipole elements.}
\begin{align}
    d_0 = \smallbraketop{0}{d^0}{0},\qquad
    d_+ = \smallbraketop{+}{d^0}{+},\qquad
    q_+ = |\smallbraketop{+}{d^+}{0}| = |\smallbraketop{0}{d^-}{+}|
\end{align}
Due to the different angular momentum of the two states, the term $d^0_i d^0_j$ only generates an interaction term $\sz_i \sz_j$. However, the term $d^+_i d^-_j$ provides an excitation conserving tunneling term proportional to $\splus\sminus$, just as before. The different nature of the states leads to a flipped sign compared to the previous model:
\begin{align*}
    \Hdd_{ij} &= \frac{\kappa}{R_{ij}^3} \Big[ \frac{(d_+-d_0)^2}{4} \sz_i \sz_j  - \frac{q_+^2}{2} (\splus_i \sminus_j + \sminus_i \splus_j) \\
              &\qquad\qquad\qquad + \frac{d_+^2 - d_0^2}{4} (\sz_i + \sz_j)  + \frac{(d_++d_0)^2}{4} \Big].
\end{align*}

\subsubsection{Magnetic dipoles}
Using the magnetic dipole moment $\vec{d}=\mu_\text{B} g \vecsigma / 2$ of a dipolar atom with spin $S=1/2$,
the dipole matrix elements are given by
\begin{align}
    d_0 &= \smallbraketop{0}{d^0}{0} = -\mu_\text{B}g/2, \vphantom{\sqrt{2}} \\
    d_+ &= \smallbraketop{+}{d^0}{+} = +\mu_\text{B}g/2, \vphantom{\sqrt{2}} \\
    q_+ &= |\smallbraketop{+}{d^+}{0}| = \smallbraketop{+}{\sqrt{2} \sigma^+}{0} = +\mu_\text{B}g/\sqrt{2}.
\end{align}
Then, including the energy non-conserving terms proportional to $\sminus_i\sminus_j$, the Hamiltonian reduces to
\begin{align}
    \Hdd_{ij} = \frac{\kappa \mu_\text{B}^2 g^2}{4 R_{ij}^3} \bc{ \sz_i \sz_j - \bb{\splus_i \sminus_j+ 3 \sminus_i\sminus_j \ef{+2i\phi} + \hc } }.
\end{align}

\subsection{Three-level dipoles}
We create a \textsf{V}-level scheme including the $\ketz, \ketp$ and $\ketm$ levels with
dipole matrix elements
\begin{align}
    d_0 = \smallbraketop{0}{d^0}{0}, \qquad
    d_\pm = \smallbraketop{\pm}{d^0}{\pm}, \qquad
    q_\pm &= |\smallbraketop{\pm}{d^\pm}{0}|.
\end{align}
%
% Then, we define $R=\ketbra{0}{{+}}$ and $L=\ketbra{0}{{-}}$, as well as projectors $P^0, P^\pm$.
% \begin{align}
%     d^0 &= d_0 P^0 + d_+ P^+ + d_- P^- \\
%     d^+ &= d_{+0} R^\dag - d_{-0} L \\
%     d^- &= d_{+0} R - d_{-0} L^\dag
% \end{align}
% using this, we can write
% \begin{align}
%     d^0_i d^0_j &= (d_0 P^0_i + d_+ P^+_i + d_- P^-_i)(d_0 P^0_j + d_+ P^+_j + d_- P^-_j) \\
%     d^+_i d^-_j + d^-_i d^+_j &= d_{+0}^2 R^\dag_i R_j + d_{-0}^2 L^\dag_i L_j + \hc \\
%     d^+_i d^+_j &= - d_{+0}d_{-0} R^\dag_i L_j
% \end{align}
% neglecting excitation-nonconserving terms
%
Then, we define $\bop_\pm=\ketbra{0}{{\pm}}$, and write the dipole operator in terms of the bosonic operators
\begin{align}
    d^0 &= d_+ \nop_+ + d_- \nop_- + d_0 (1-\nop_+-\nop_-) = d_0 + (d_+-d_0) \nop_+ + (d_--d_0) \nop_-, \\
    d^+ &= q_+ \bopd_+ - q_- \bop_-, \\
    d^- &= q_+ \bop_+  - q_- \bopd_-.
\end{align}
using this, we can write
\begin{align}
    d^0_i d^0_j &= d_0^2 + d_0(d_+-d_0) (\nop_{+,i}+\nop_{+,j}) + d_0(d_--d_0) (\nop_{-,i}+\nop_{-,j}) \\
                &\quad + (d_+ - d_0)^2 \nop_{+,i} \nop_{+,j} + (d_- - d_0)^2 \nop_{-,i} \nop_{-,j} \\
                &\quad + (d_+ - d_0)(d_- - d_0) (\nop_{+,i} \nop_{-,j} + \nop_{-,i} \nop_{+,j}) \\
    d^+_i d^-_j &= q_+^2 \bopd_{+,i} \bop_{+,j} + q_-^2 \bopd_{-,j} \bop_{-,i} \\
    d^+_i d^+_j &= - q_+q_- \, \bopd_{+,i} \bop_{-,j}
\end{align}
neglecting excitation-nonconserving terms. Then, dropping the constant terms and \qu{magnetic fields}, we find:
\begin{align}
    \Hdd_{ij} = \sum_{\alpha,\beta} \bc{ t_{ij}^{\alpha\beta} \bopd_{\alpha,i} \bop_{\beta,j} + V_{ij}^{\alpha\beta} \nop_{\alpha,i} \nop_{\beta,j} }
\end{align}
where
\begin{align}
    t_{ij} &= \frac{\kappa}{R_{ij}^3} \begin{pmatrix}
    -\frac{q_+^2}{2} & \frac{3q_+q_-}{2} \ef{-2i \phi_{ij}} \\
    \frac{3q_+q_-}{2} \ef{+2i \phi_{ij}} & -\frac{q_-^2}{2}
\end{pmatrix}, \\
    V_{ij}^{\alpha\beta} &= \frac{\kappa}{R_{ij}^3} (d_\alpha-d_0)(d_\beta-d_0).
\end{align}

\paragraph{One-dimensional system}
In one dimension, we can always choose $\phi_{ij}=0$. Furthermore, if $q_+=q_-$, the tunneling
elements simplify to
\begin{align}
    t_{ij} &= \frac{\kappa q^2}{2R_{ij}^3} (3\sigma_x-\sigma_z).
\end{align}
By transforming to $\ket{x}=(\ketp-\ketm)/\sqrt{2}$ and $\ket{y}=(\ketp+\ketm)/\sqrt{2}$, the
tunneling part can be diagonalized:
\begin{align}
    t_{ij} &= \frac{\kappa q^2}{R_{ij}^3} \begin{pmatrix}
    -1 & 0 \\
    0 & 2
    \end{pmatrix}.
\end{align}
If the energy of the $\ket{x}, \ket{y}$ manifold is low enough compared to the $\ketz$ state, this
causes excitations to condense in the $\ket{x}$ state, building a ferroelectric state of matter. This quantum phase transition has been investigated beyond mean-field theory in~\cite{Klinsmann2014}.
\todo{rewrite this paragraph}

\subsection{General case: Four-level dipoles in a tilted field}
Including all states in \tref{dipole-moments} and a (possibly) tilted electric field, we find
\begin{align}
    t_{ij} &= \frac{\kappa}{R_{ij}^3} \begin{pmatrix}
    -\frac{q_+^2}{2} f_0 & \frac{3q_+ q_1}{\sqrt{2}} f_1^* & \frac{3q_+q_-}{2} f_2^* \\
    \frac{3q_+q_1}{\sqrt{2}} f_1 & q_1^2 f_0 & \frac{3q_1q_-}{\sqrt{2}} f_1^* \\
    \frac{3q_+q_-}{2} f_2 & \frac{3q_1q_-}{\sqrt{2}} f_1 & -\frac{q_-^2}{2} f_0
\end{pmatrix}
\end{align}
