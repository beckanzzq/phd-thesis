\chapter{Introduction: dipolar spin models}

% \section{Dipolar spin systems}
% \subsection{Lattice structures}
% \subsubsection{One-dimensional}
% \subsubsection{Two-dimensional}

% \section{Topological band structures}
% \subsection{Chern number}
% \subsection{Edge states}
% \subsection{Quantum Hall effect}

\section{Physical realizations}
\todo{NV centers, Rydberg atoms, polar molecules, dipolar atoms}


\section{Dipolar interactions}
The tensor part of the dipole-dipole interaction $D\equiv \Vdd R_{ij}^3/\Cdd=\vec{d_1}\vec{d_2}-3(\vec{d_1}\hat{\vec{r}})(\vec{d_2}\hat{\vec{r}})$ where $\Cdd = \mu_0 /4\pi$ or $1/4\pi\epsilon_0$ can be written in terms of a spherical tensor $T^2(\vec{d_i},\vec{d_j})$ of rank 2, build from the two dipole moments, with components
\begin{align}
T^2_0(\vec{d_i},\vec{d_j}) &= \frac{1}{\sqrt{6}} (d^i_+ d^j_- + 2d^i_0 d^j_0 + d^i_{-} d^j_{+})\\
T^2_{\pm 1}(\vec{d_i},\vec{d_j}) &= \frac{1}{\sqrt{2}} (d^i_{\pm} d^j_0 + d^i_0 d^j_{\pm})\\
T^2_{\pm 2}(\vec{d_i},\vec{d_j}) &= d^i_{\pm} d^j_{\pm} \phantom{\frac{1}{\sqrt{42}}}
\end{align}
[ Note the additional factor of $\mp 1/\sqrt{2}$ compared to the definition of $j_\pm$. ]
where the dipole moments are spherical tensors with components
\begin{align}\label{eq:dipsph}
d_0 = d_z\qquad d_\pm = \mp\frac{1}{\sqrt{2}}\bb{d_x\pm i d_y}
\end{align}
The dipole-dipole interaction is then given by
\begin{align}
D &= -\sqrt{6} \,T^2(C)\cdot T^2(\vec{d_1},\vec{d_2})\\
&\equiv-\sqrt{6}  \sum_p (-1)^p C^2_{-p}(\theta,\phi) T^2_p(\vec{d_1},\vec{d_2})
\end{align}
where $C^k_p(\theta,\phi)=\sqrt{\frac{4\pi}{2k+1}} Y^k_p(\theta,\phi)$ are modified (unnormalized) spherical harmonics. We can expand
\begin{align}\label{eq:doperator}
D &= (1-3\cos^2 \theta) \left[d_0 d_0 + \frac{1}{2}\bb{d_+ d_- + d_- d_+} \right] \\
&\,\,-\frac{3}{\sqrt{2}}\sin\theta \,\cos\theta \left[ \ef{+i\phi} (d_0 d_- + d_- d_0) - \ef{-i\phi} (d_0 d_+ + d_+ d_0) \right]\\
&\,\,-\frac{3}{2}\sin^2\theta \left[ \ef{+2i\phi} d_- d_- + \ef{-2i\phi} d_+ d_+\right]
\end{align}
where in each term the first dipole operator acts on $i$ and the second on $j$. Notice that this coincides with $\Vdd$ from the first part if we change from $d_\pm \rightarrow \mp j_\pm/\sqrt{2}$ and $d_0\rightarrow j_0$.
For a 2D geometry we set $\theta=\pi/2$ and get
\begin{align}
D = d_0 d_0 + \frac{1}{2}\bb{d_+ d_- + d_- d_+} -\frac{3}{2} \left[ \ef{+2i\phi} d_- d_- + \ef{-2i\phi} d_+ d_+\right]
\end{align}
For a 1D geometry in $x$-direction we set $\phi=0$ to get
\begin{align}
D &= d_0 d_0 + \frac{1}{2}\bb{d_+ d_- + d_- d_+} -\frac{3}{2} \bb{d_- d_- + d_+ d_+}\\
 &= d_z d_z + d_y d_y - 2 d_x d_x
\end{align}
while for a 1D geometry in polarization-direction $z$ we set $\theta=0$ in \eqref{eq:doperator} to get
\begin{align}
D &= -2 d_0 d_0 - \bb{d_+ d_- + d_- d_+} = -2 d_z d_z+ d_x d_x + d_y d_y
\end{align}

\section{Dipolar interactions in a tilted electric field}
\fig{.5}{lattice-geometry}{lattice-geometry}{Illustration of the relevant axes and angles. The lattice lies in the $xy$ plane while the electric field is tilted from the $z$ axis by an angle $\Theta_0$ and rotated around the $z$ axis by an angle $\Phi_0$ with respect to the $x$ axis. The direction of the vector $\vec{R}$, connecting two dipoles, is determined by the polar angle $\Phi$.}

\noindent
We consider a two-dimensional system where the lattice lies in the $xy$ plane and the DC electric field $\vec{E}$ points in an arbitrary direction~\cite{Gorshkov2011c}, determined by the spherical angles $\Theta_0, \Phi_0$:
\begin{align}
    \vec{\hat{E}}=\begin{pmatrix}
        \sin\Theta_0\cos\Phi_0 \\
        \sin\Theta_0\sin\Phi_0 \\
        \cos\Theta_0
    \end{pmatrix}.
\end{align}
The geometry is illustrated in \tref{lattice-geometry}.
We are interested in the dipole-dipole interaction between two dipoles which are separated
by a vector
\begin{align}
    \vec{R}=\begin{pmatrix}R\cos\Phi \\ R\sin\Phi \\ 0\end{pmatrix}.
\end{align}
Then, for the angle $\theta$ between the dipole orientation $\vec{\hat{E}}$ and the interconnection line between the dipoles $\vec{\hat{R}}$, we find the relation
\begin{align}
    \cos\theta = \vec{\hat{E}} \cdot \vec{\hat{R}} &= \sin\Theta_0\bb{\cos\Phi_0\cos\Phi+\sin\Phi_0\sin\Phi}\\
&=\sin\Theta_0\cos(\Phi-\Phi_0).
\end{align}
With the difference $\bar\Phi=\Phi-\Phi_0$, we can express the relevant terms in the dipole-dipole interaction
\begin{align}
    f_0(\Theta_0, \bar \Phi)&\equiv 1-3\cos^2\theta = 1-3\sin^2\Theta_0\cos^2\bar\Phi  \\
    f_1(\Theta_0, \bar \Phi)&\equiv \sin\theta\cos\theta \ef{i\phi} = \sin\Theta_0 \cos\bar\Phi \bb{\cos\Theta_0 \cos\bar\Phi + i \sin \bar\Phi}\\
    f_2(\Theta_0, \bar \Phi)&\equiv\sin^2\theta \ef{2i\phi} = \bb{\cos\Theta_0 \cos\bar\Phi + i \sin \bar\Phi}^2
\end{align}
which are easily seen to reduce to the old expressions in the case $\Theta_0=0$,
implying $\theta=\pi/2$ and $\bar\Phi=\phi$.

In total, the tensorial part of the dipole-dipole interaction is given by
\begin{align} \tlabel{tensortilted}
    D(\Theta_0, \bar\Phi) = f_0(\Theta_0, \bar \Phi) &\Big[d_0 d_0 + \frac{1}{2}\bb{d_+ d_- + d_- d_+} \Big] \\
      \qquad-\frac{3}{\sqrt{2}} &\Big[ f_1(\Theta_0, \bar \Phi)  (d_0 d_- + d_- d_0) - f_1(\Theta_0, -\bar \Phi) (d_0 d_+ + d_+ d_0) \Big]\\
      \qquad-\frac{3}{2} &\Big[ f_2(\Theta_0, \bar \Phi) d_- d_- + f_2(\Theta_0, -\bar \Phi) d_+ d_+\Big]
\end{align}
The dipole-dipole interaction is symmetric under spatial inversion, implying:
\begin{align}
    f_m(\Theta_0, \bar \Phi + \pi) = f_m(\Theta_0, \bar \Phi).
\end{align}
Under complex conjugation, we find
\begin{align}
    f_m^*(\Theta_0, \bar \Phi) = f_m(\Theta_0, -\bar \Phi)
\end{align}
In addition, we can write
\begin{align}
    f_1(\Theta_0, \bar \Phi) = \frac{1}{2} \sin \Theta_0 \bc{\cos \Theta_0 (1+\cos 2\bar \Phi) + i \sin 2 \bar \Phi}
\end{align}
For small angles $\Theta_0$, we can rewrite
\begin{align}
    f_0(\Theta_0, \bar \Phi) &= 1 \\
    f_1(\Theta_0, \bar \Phi) &= \frac{1}{2} \Theta_0 \bc{1+\ef{2 i \bar \Phi}}=\Theta_0 \cos \bar \Phi \ef{i \bar \Phi}\\
    f_2(\Theta_0, \bar \Phi) &= \ef{2 i \bar \Phi}
\end{align}
Defining $\Theta_0^* = \pi/2 - \theta_\text{magic}=\text{asin}(1/\sqrt{3})$, we have
\begin{align}
    f_0(\Theta_0^*, \bar \Phi) = 1-\cos^2\bar \Phi
\end{align}

\subsection{In-plane electric field}
For $\Theta_0=\pi/2$, the expressions above reduce to
\begin{align}
    1-3\cos^2\theta &= 1-3\cos^2\bar\Phi \\
    \sin\theta\cos\theta \ef{\pm i\phi} &= \pm \frac{i}{2} \sin 2\bar\Phi \\
    \sin^2\theta \ef{\pm 2i\phi} &= - \sin^2 \bar\Phi
\end{align}

% \subsection{In-plane $\Phi_0=\pi/4$ setup}
% To make things symmetric, assume $\Phi_0=\pi/4$. Then, for nearest neighbors
% along the X (Y) direction, we have

\subsection{1D setup}
For a 1D setup we can set $\Phi_0=0$. The angle $\Phi$ is limited
to the values $0$ or $\pi$, indicated by $\sin\Phi=0$ and
$\cos\Phi=\pm 1=\text{sign}(X)$. Then, the expressions above reduce to
\begin{align}
    1-3\cos^2\theta &= 1-3\sin^2\Theta_0 \\
    \sin\theta\cos\theta \ef{\pm i\phi} &= \sin\Theta_0 \cos\Theta_0 \\
    \sin^2\theta \ef{\pm 2i\phi} &= \cos^2 \Theta_0
\end{align}

