\chapter{Realizing dipolar spin models}
\label{dipolar_spinmodels}

This introductory chapter is largely of technical nature and describes the detailed steps for the realization of dipolar spin systems, starting from the microscopic setup.
The goal is to introduce a common framework for the derivation of the effective models which are subject of \cref{anomalous_behavior,dipolar_fermions,topological_bands} of this thesis.

Proposals for simulating spin systems
\begin{itemize}
    \item with polar molecules \cite{Micheli2006,Hauke2010,Gorshkov2011,Gorshkov2011b,Peter2012b,Syzranov2014,Peter2014}
\end{itemize}


\section{Physical implementations of dipolar systems}

In this section we briefly review some of the possible realizations for dipolar spin
systems in ultracold atomic and molecular systems and discuss some of the similarities as well as important differences.

mention NV centers
review \cite{Lahaye2009}

\subsection{Atoms with high magnetic moments}

Highly dipolar atoms where among the first experimental realizations
of ultracold dipolar systems~\cite{Griesmaier2005a}. These atoms have
large magnetic dipole moments $xxx$ due to their particular electronic
structure. In the meantime, a large number of different atoms have been
successfully cooled. Chromium with a magnetic moment of $6\muB$ with
bosonic \cite{Griesmaier2005a,Beaufils2008} and fermionic isotopes
\cite{Naylor2015}, Dysprosium with a magnetic moment of $10\muB$
with bosonic \cite{Lu2011} and fermionic isotopes \cite{Lu2012} as
well as Erbium with a magnetic moment of $7\muB$, also in bosonic
\cite{Aikawa2012} and fermionic forms \cite{Aikawa2014}.


write about:
hyperfine ground state
huge amount of internal states
Lande factor
Zeeman shift

advantages:
easier to cool and manipulate than polar molecules
no chemical reactions
permanent dipole moments

disadvantages:
weaker dipole moments

\subsection{Polar molecules}

Compared to magnetic atoms, the dipolar effects in polar molecules
can be much larger due to the electric nature of the dipole moments
which typically leads to an interaction strength which is stronger by
up to four orders of magnitude due to the $\alpha^2$ factor between
electric and magnetic systems. For a detailed comparison of the
different interaction strengths, see \cite{}.
Different species of polar molecules have been successfully cooled,
among them KRb \cite{Ni2008b,Ospelkaus2010}, NaK \cite{Wu2012a} and NaLi \cite{Heo2012}.

In this thesis, we restrict ourselves to the rotational structure of the
polar molecules, neglecting any vibrational and electronic excitations,
as well as possible hyperfine structure effects. Then, the internal
structure of the molecules is given by the rotational degree of freedom.
Describing the molecule as a rigid rotor, the internal structure is
given by $H_\text{rot}=B J^2 - \vecd \vec{E}$, where $B$ is the rotational
constant $\vecd$ is the dipole moment which couples to an external
electric field $\vec{E}$. In the absence of any external fields, the
eigenstates are simply given by the angular momentum states $\ket{J,M}$
where $J$ is the total angular momentum and $M$ is the projection on the
quantization axis.

In contrast to magnetic atoms, polar molecules do not have a dipole moment
in the absence of external fields, i.e. $\smallbraketop{J,M}{\vecd}{J,M}=0$.
An external field can induce static dipole moments ...
Another important difference compared to magnetic atoms is the rotational splitting $B$
which is typically much larger than the dipolar interaction strength $\kappa d^2/a^3$
at realistic inter-molecule separations $a$.

in optical lattice \cite{Chotia2012}

seeing long-range dipolar effects:
\cite{Yan2013}

% \fig{.8}{pm-levelscheme}{kjsdf}

\subsection{Rydberg atoms}
While we focus on polar molecules, the physics and energy scale are very similar
in Rydberg atoms.

dipolar effects in Rydberg: \cite{Westermann2006,Nipper2012b}


\section{Dipole-dipole interaction}
The aim of this section is to rewrite the familiar interaction between two dipoles $\vec{d}_i$ and $\vec{d}_j$ at positions $\vecR_i, \vecR_j$, namely
\begin{align} \tlabel{ddint}
\Hdd_{ij} = \frac{\kappa}{\absv{\vecR_{ij}}^3} \bc{ \vecd_i \cdot \vecd_j -3 \bb{ \vecd_i \cdot \hat{\vecR}_{ij} }\bb{ \vecd_j \cdot \hat{\vecR}_{ij} } }
\end{align}
in a spherical tensor representation~\cite{Micheli2007,Gorshkov2011c} which will be useful throughout this thesis.
\Tref{ddint} is given in terms of the relative vector $\vecR_{ij} = \vecR_j - \vecR_i$ and its normalized form $\hat{\vecR}_{ij}$.
The constant prefactor is given by $\kappa = 1/4\pi\epsilon$ for electric dipoles and by
$\kappa = \mu_0 / 4\pi$ for magnetic dipoles.
In the following, we focus on the distance-independent part $D_{ij} \equiv \Hdd_{ij} \cdot \absv{\vecR_{ij}}^3/\kappa=\vecd_i \cdot \vecd_j -3 (\vecd_i \cdot \hat{\vecR}_{ij})(\vecd_j \cdot \hat{\vecR}_{ij})$ of the dipole-dipole interaction which can be written in terms of the spherical tensor $T^2(\vecd_i, \vecd_j)$ of rank two with the components
\begin{align}
T^2_0(\vecd^i, \vecd^j) &= \frac{1}{\sqrt{6}} (d_i^+ d_j^- + 2d_i^0 d_j^0 + d_i^{-} d_j^{+}), \\
T^2_{\pm 1}(\vecd^i, \vecd^j) &= \frac{1}{\sqrt{2}} (d_i^{\pm} d_j^0 + d_i^0 d_j^{\pm}), \\
T^2_{\pm 2}(\vecd^i, \vecd^j) &= d_i^{\pm} d_j^{\pm} \vphantom{\frac{1}{42}}.
\end{align}
It is constructed from the two dipole moments which are themselves rank-one tensors with spherical components
\begin{align} \tlabel{sphericalcomp}
d^0_i = d^z_i, \qquad d_i^\pm = \mp\frac{1}{\sqrt{2}}\bb{d_i^x\pm i d_i^y}.
\end{align}
Using this, the dipole-dipole interaction can be written as a contraction of two rank-two tensors \cite{Brown2003}, leading to
\begin{align}
    D_{ij} &= -\sqrt{6} \: T^2(C)\cdot T^2(\vec{d_1},\vec{d_2})\\
           &= -\sqrt{6}  \sum_{m=-2}^{2} (-1)^m C^2_{-m}(\theta,\phi) \, T^2_m(\vecd_i,\vecd_j).
\end{align}
Here, $C^l_m(\theta,\phi)=\sqrt{\frac{4\pi}{2l+1}} Y^l_m(\theta,\phi)$ are the modified (unnormalized) spherical harmonics and the spherical angles $\theta = \theta_{ij}$ and $\phi = \phi_{ij}$ describe the direction of the vector $\hat{\vecR}_{ij}$ in the coordinate system of the quantization axis. We can expand this expression to get
\begin{align}\tlabel{doperator}
    D_{ij} &= (1-3\cos^2 \theta) \BC{ d_i^0 d_j^0 + \frac{1}{2}\Bb{d_i^+ d_j^- + d_i^- d_j^+} } \\
           &\quad -\frac{3}{\sqrt{2}}\sin\theta \,\cos\theta\, \BC{ \Bb{ d_i^0 d_j^- + d_i^- d_j^0 } \ef{+i\phi} - \Bb{ d_i^0 d_j^+ + d_i^+ d_j^0 } \ef{-i\phi} } \\
           &\quad -\frac{3}{2}\sin^2\theta \, \BC{ \ef{+2i\phi} d_i^- d_j^- + \ef{-2i\phi} d_i^+ d_j^+ }.
\end{align}
It is worth noting that the $T^2_{m=0}(\vecd_i, \vecd_j)$-terms in the first row conserve the ``internal'' angular momentum while the $m=1$ ($m=2$) terms in the second (third) row increase or decrease the internal angular momentum by one (two) quanta.

\subsection{High-symmetry alignments}
For most applications, we will be concerned with two-dimensional systems where the dipoles are aligned perpendicular to the plane. Then, the dipoles are also perpendicular to the interconnecting axis $\vecR_{ij}$, implying $\theta = \pi/2$. In this case, the $m=1$ terms drop out and the tensorial part reduces to
\begin{align} \tlabel{dop2d}
    D^\text{(2D)}_{ij} = d_i^0 d_j^0 + \frac{1}{2}\bb{d_i^+ d_j^- + d_i^- d_j^+} -\frac{3}{2} \bb{ d_i^- d_j^- \ef{+2i\phi} + d_i^+ d_j^+ \ef{-2i\phi} }
\end{align}
For a one-dimensional geometry there are two high-symmetry alignments. If the dipoles are perpendicular to the lattice direction (say, the $x$-direction) we can set $\phi=0$, further simplifying $D_{ij}$ to
\begin{align}
    D_{ij}^{(1D,\perp)} &= d_i^0 d_j^0 + \frac{1}{2}\bb{d_i^+ d_j^- + d_i^- d_j^+} -\frac{3}{2} \bb{d_i^- d_j^- + d_i^+ d_j^+} \\
 &= d_i^z d_j^z + d_i^y d_j^y - 2 d_i^x d_j^x.
\end{align}
Conversely, if the dipoles point along the lattice direction, we can set $\theta=0$ in \tref{doperator} to get
\begin{align}
    D^{(1D,\parallel)}_{ij} &= -2 d_i^0 d_j^0 - \bb{d_i^+ d_j^- + d_i^- d_j^+} = -2 d_i^z d_j^z+ d_i^x d_j^x + d_i^y d_j^y
\end{align}

\subsection{Tilted field geometry}
\doublefig{.4}{lattice-geometry}{Tilted-field 2D geometry}
{.4}{dipole-moments}{Schematic level structure}
{\sfA~Illustration of the relevant axes and angles. The lattice lies in the $xy$ plane while the static external field $\vec{E}$ is tilted from the $z$ axis by an angle $\Theta_0$ and rotated around it by an angle $\Phi_0$ with respect to the $x$ axis. The direction of the vector $\vec{R}$, connecting two dipoles, is determined by the polar angle $\Phi$.
\sfB~Internal rotational states of a dipole with a ground state $\ketz$ and excited state $\keto$ with $M=0$, as well as two states $\ketpm$ with $M=\pm 1$. The relevant static dipole moments (blue) and transition dipole moments (red) are shown. The plus sign is for the transitions going \qu{upwards} and the minus sign for transitions going \qu{downwards}.}


If we are not in a high-symmetry geometry, we can consider a more general situation (see \tref{lattice-geometry}) with a two-dimensional system in the $xy$ plane and an external polarizing field $\vec{E}$ pointing in an arbitrary direction~\cite{Gorshkov2011c}, determined by the spherical angles $\Theta_0$ and $\Phi_0$:
\begin{align}
    \vec{\hat{E}}=\begin{pmatrix}
        \sin\Theta_0\cos\Phi_0 \\
        \sin\Theta_0\sin\Phi_0 \\
        \cos\Theta_0
    \end{pmatrix}.
\end{align}
As before, we are interested in the interaction between two dipoles which are now separated
by the in-plane vector
\begin{align}
\vecR_{ij} = \begin{pmatrix}R_{ij} \cos\phi_{ij} \\ R_{ij} \sin\phi_{ij} \\ 0\end{pmatrix}.
\end{align}
For the angle $\theta$ between the dipole orientation $\vec{\hat{E}}$ and the interconnection line between the dipoles $\hat{\vecR}_{ij}$, we find the relation
\begin{align}
    \cos\theta = \vec{\hat{E}} \cdot \hat{\vecR}_{ij} &= \sin\Theta_0\bb{\cos\Phi_0\cos\phi_{ij}+\sin\Phi_0\sin\phi_{ij}}\\
                                                      &=\sin\Theta_0\cos(\phi_{ij}-\Phi_0).
\end{align}
By defining the difference $\bar\Phi=\phi_{ij}-\Phi_0$, we can express the relevant terms in the dipole-dipole interaction from \tref{doperator} as
\begin{align} \tlabel{fmfunctions}
    f_0(\Theta_0, \bar \Phi)&\equiv 1-3\cos^2\theta = 1-3\sin^2\Theta_0\cos^2\bar\Phi, \vphantom{\bb{\bar \Phi}} \\
    f_1(\Theta_0, \bar \Phi)&\equiv \sin\theta\cos\theta \ef{i\phi} = \sin\Theta_0 \cos\bar\Phi \bb{\cos\Theta_0 \cos\bar\Phi + i \sin \bar\Phi}, \\
    f_2(\Theta_0, \bar \Phi)&\equiv\sin^2\theta \ef{2i\phi} = \bb{\cos\Theta_0 \cos\bar\Phi + i \sin \bar\Phi}^2.
\end{align}
which are easily seen to reduce to the expressions \labeltref{dop2d} in the case of a perpendicular external field with $\Theta_0=0$, implying $\theta=\pi/2$ and $\bar\Phi=\phi_{ij}$.
In total, the tensorial part of the dipole-dipole interaction is given by
\begin{align} \tlabel{tensortilted}
    D_{ij}(\Theta_0, \bar\Phi) = f_0(\Theta_0, \bar \Phi) &\BC{ d_i^0 d_j^0 + \frac{1}{2}\bb{d_i^+ d_j^- + d_i^- d_j^+} } \\
    \qquad-\frac{3}{\sqrt{2}} &\BC{ f_1(\Theta_0, \bar \Phi)  (d_i^0 d_j^- + d_i^- d_j^0) - f_1(\Theta_0, -\bar \Phi) (d_i^0 d_j^+ + d_i^+ d_j^0) }\\
    \qquad-\frac{3}{2} &\BC{ f_2(\Theta_0, \bar \Phi) d_i^- d_j^- + f_2(\Theta_0, -\bar \Phi) d_i^+ d_j^+ }.
\end{align}
% The dipole-dipole interaction is symmetric under spatial inversion, implying:
% \begin{align}
%     f_m(\Theta_0, \bar \Phi + \pi) = f_m(\Theta_0, \bar \Phi).
% \end{align}
% Under complex conjugation, we find
% \begin{align}
%     f_m^*(\Theta_0, \bar \Phi) = f_m(\Theta_0, -\bar \Phi)
% \end{align}
% In addition, we can write
% \begin{align}
%     f_1(\Theta_0, \bar \Phi) = \frac{1}{2} \sin \Theta_0 \bc{\cos \Theta_0 (1+\cos 2\bar \Phi) + i \sin 2 \bar \Phi}
% \end{align}
% For small angles $\Theta_0$, we can rewrite
% \begin{align}
%     f_0(\Theta_0, \bar \Phi) &= 1 \\
%     f_1(\Theta_0, \bar \Phi) &= \frac{1}{2} \Theta_0 \bc{1+\ef{2 i \bar \Phi}}=\Theta_0 \cos \bar \Phi \ef{i \bar \Phi}\\
%     f_2(\Theta_0, \bar \Phi) &= \ef{2 i \bar \Phi}
% \end{align}
% Defining $\Theta_0^* = \pi/2 - \theta_\text{magic}=\text{asin}(1/\sqrt{3})$, we have
% \begin{align}
%     f_0(\Theta_0^*, \bar \Phi) = 1-\cos^2\bar \Phi
% \end{align}

% \paragraph{In-plane field:}
% For $\Theta_0=\pi/2$, the expressions above reduce to
% \begin{align}
%     1-3\cos^2\theta &= 1-3\cos^2\bar\Phi \\
%     \sin\theta\cos\theta \ef{\pm i\phi} &= \pm \frac{i}{2} \sin 2\bar\Phi \\
%     \sin^2\theta \ef{\pm 2i\phi} &= - \sin^2 \bar\Phi
% \end{align}

% \paragraph{One-dimensional setup:}
% For a 1D setup we can set $\Phi_0=0$. The angle $\Phi$ is limited
% to the values $0$ or $\pi$, indicated by $\sin\Phi=0$ and
% $\cos\Phi=\pm 1=\text{sign}(X)$. Then, the expressions above reduce to
% \begin{align}
%     1-3\cos^2\theta &= 1-3\sin^2\Theta_0 \\
%     \sin\theta\cos\theta \ef{\pm i\phi} &= \sin\Theta_0 \cos\Theta_0 \\
%     \sin^2\theta \ef{\pm 2i\phi} &= \cos^2 \Theta_0
% \end{align}


\section{Dipolar spin models}
The different dipolar models throughout this thesis will be largely determined by a particular choice of internal states of the dipoles. These states can be additionally \qu{dressed} by external DC and AC fields, with the details depending on the particular physical realization.
In the following, we look at several specific choices. Each of them corresponds to a rise in complexity compared to the previous one, but also introduces additional properties and characteristics.

\subsection{Realizing Ising and XY interactions}
\tlabel{xxz}
Here, we assume that we have chosen two states, denoted as $\ketz$ and $\keto$, which have the
same $M$ quantum number (see~\tref{dipole-moments}).
Then, only the $d^0_id^0_j$ part of the dipolar interaction is relevant.
We define the (transition) dipole elements
\begin{align}
    d_0 = \smallbraketop{0}{d^0}{0},\qquad
    d_1 = \smallbraketop{1}{d^0}{1},\qquad
    q_1 = |\smallbraketop{1}{d^0}{0}| = |\smallbraketop{0}{d^0}{1}|.
\end{align}
The evaluation of these dipole matrix elements for polar molecules in the presence of external fields is straightforward and has been described in detail elsewhere~\cite{Micheli2007}.
Using the projectors $P^0 = \ketbra{0}{0}$ and $P^1 = \ketbra{1}{1}$, we can now express the
dipole-dipole interaction \labeltref{dop2d} in the $\ketz, \keto$ subspace as
\begin{align}
    \Hdd_{ij} &= \frac{\kappa}{R_{ij}^3} \Big[ d_1^2 P^1_i P^1_j + d_0^2 P^0_i P^0_j + d_1d_0 (P^1_i P^0_j + P^0_i P^1_j)
              + q_1^2 (\splus_i \sminus_j + \sminus_i \splus_j) \Big]
\end{align}
where we have neglected all processes which are not energy-conserving,
i.e. terms that do not conserve the number of excitations.
Using $\sz = P^1 - P^0$ and $\id = P^1 + P^0$ we find
\begin{align} \tlabel{xyfull}
    \Hdd_{ij} &= \frac{\kappa}{R_{ij}^3} \Big[ \frac{(d_1-d_0)^2}{4} \sz_i \sz_j  + q_1^2 (\splus_i \sminus_j + \sminus_i \splus_j)\\
              &\qquad\qquad + \frac{d_1^2 - d_0^2}{4} (\sz_i + \sz_j)  + \frac{(d_1+d_0)^2}{4} \Big].
\end{align}
Here, the first two terms describe Ising- and XY-type interactions between the two-level dipoles. The third term is equivalent to a magnetic field in $z$ direction and the last term is a constant energy offset. Typically, we will be interested in the interaction terms in the first line. Note, however, that the \qu{magnetic field} term and the constant offset depend on the positions of all other dipoles. If the system is not translationally invariant, these terms
describe spatially dependent contributions.

\paragraph{Dipolar XXZ Hamiltonian:}
Using the spin one-half operators $S^\alpha_i = \hbar \sigma^\alpha_i/2$, we can write the interaction Hamiltonian for a system of interacting dipoles as
\begin{align} \tlabel{hxxz}
    H &= \frac{1}{2} \sum_{i\ne j} \Hdd_{ij} = \sum_{i\ne j} \frac{\kappa}{\hbar^2 R_{ij}^3} \Big[ \frac{(d_1-d_0)^2}{2} S^z_i S^z_j  + q_1^2 (S^x_i S^x_j + S^y_i S^y_j) \Big] \\
         % &= \sum_{i\ne j} \frac{a^3}{R_{ij}^3} \Big[ \frac{\kappa (d_1-d_0)^2}{2a^3} S^z_i S^z_j  + \frac{\kappa q_1^2}{a^3} (S^x_i S^x_j + S^y_i S^y_j) \Big] \\
         &= \frac{Ja^3}{\hbar^2} \sum_{i\ne j} \frac{\cos\theta\, S^z_i S^z_j  + \sin\theta\, (S^x_i S^x_j + S^y_i S^y_j)}{R_{ij}^3}
\end{align}
where $J\cos\theta = \kappa (d_1-d_0)^2/2a^3$ and $J\sin\theta = \kappa q_1^2/a^3$. The lattice constant $a$ has been introduced for future convenience.
This model is reminiscent of the famous XXZ Hamiltonian, where the nearest neighbor interactions are replaced by the dipolar $R_{ij}^{-3}$ interaction. For particular values of the $\theta$ parameter, this model describes an Ising model ($\theta = 0, \pi$), Heisenberg model ($\theta=\pm \pi/2$) or XY model ($\theta=\pi/4, 3\pi/4$). The modifications due to the dipolar interaction are subject of \cref{anomalous_behavior} of this thesis.

\subsection{Angular momentum difference}
\tlabel{twoleveldm}
Staying in the regime of two-level dipoles, we can reach a different but related situation, if the upper state $\keto$ is replaced by the $M=1$ state $\ketp$, see \tref{dipole-moments}.
Proceeding similarly as before, we define the relevant dipole matrix elements
\begin{align}
    d_0 = \smallbraketop{0}{d^0}{0},\qquad
    d_+ = \smallbraketop{+}{d^0}{+},\qquad
    q_+ = |\smallbraketop{+}{d^+}{0}| = |\smallbraketop{0}{d^-}{+}|.
\end{align}
Note that we can choose the phases of $\ketz$ and $\ketp$ freely, allowing us to choose real values for the transition dipole elements. Be aware, however, that $\smallbraketop{0}{d^-}{+} = \smallbraketop{+}{(d^-)^\dag}{0}^* = -\smallbraketop{+}{d^+}{0}^*$ due to the definition of $d^\pm$.
In contrast to the previous section, the term $d^0_i d^0_j$ only generates an interaction term proportional to $\sz_i \sz_j$, as the angular momentum of the two states is different. However, the term $d^+_i d^-_j$ in the dipole-dipole interaction provides an excitation-conserving tunneling term proportional to $\splus\sminus$, just as before. The different nature of the states leads to a flipped sign compared to the previous model:
\begin{align}
    \Hdd_{ij} &= \frac{\kappa}{R_{ij}^3} \Big[ \frac{(d_+-d_0)^2}{4} \sz_i \sz_j  - \frac{q_+^2}{2} (\splus_i \sminus_j + \sminus_i \splus_j) \\
              &\qquad\qquad\qquad + \frac{d_+^2 - d_0^2}{4} (\sz_i + \sz_j)  + \frac{(d_++d_0)^2}{4} \Big].
\end{align}
This allows us to tune the model in \tref{hxxz} to different $\theta$ values.

\paragraph{Magnetic dipoles:}
Using the magnetic dipole moment $\vec{d}=\muB g \vecsigma / 2$ of a dipolar atom with spin $S=1/2$,
the dipole matrix elements are given by
\begin{align}
    d_0 &= \smallbraketop{0}{d^0}{0} = -\muB g/2, \vphantom{\sqrt{2}} \\
    d_+ &= \smallbraketop{+}{d^0}{+} = +\muB g/2, \vphantom{\sqrt{2}} \\
    q_+ &= |\smallbraketop{+}{d^+}{0}| = \smallbraketop{+}{\sqrt{2} \sigma^+}{0} = +\muB g/\sqrt{2}.
\end{align}
Then, including the energy non-conserving terms proportional to $\sminus_i\sminus_j$, the Hamiltonian reduces to
\begin{align} \tlabel{hdd2dmagnetic}
    \Hdd_{ij} = \frac{\kappa \muB^2 g^2}{4 R_{ij}^3} \bc{ \sz_i \sz_j - \bb{\splus_i \sminus_j+ 3 \sminus_i\sminus_j \ef{+2i\phi} + \hc } }.
\end{align}
This type of interaction between two magnetic dipoles will be utilized in~\cref{dipolar_fermions}, where we make use of the dipolar relaxation terms.


\section{Excitation hopping: mapping to hard-core bosons}
For spin models it is often useful to think in terms of excitations above a certain well defined (ground) state~\cite{Holstein1940}.
As an exemplary case, we take the spin $1/2$ model from \tref{hxxz} and write it in terms of
$S^+ = (S^x_i+iS^y_i)/\hbar$ and $S^- = (S^x_i-iS^y_i)/\hbar$:
\begin{align}
    H &= \frac{Ja^3}{\hbar^2} \sum_{i\ne j}  \frac{1}{R_{ij}^3} \bc{ \cos\theta\, S^z_i S^z_j  + \frac{1}{2}\sin\theta\, (S^+_i S^-_j + S^-_i S^+_j) }.
\end{align}
Assume that for some set of parameters, the system is in the state $\ket{G}=\prod_i \ketdn_i$ and we are interested in the excitations.
Then, for each site, we introduce the operator $\bop_i=S^-_i/\hbar = \ketbra{\downarrow}{\uparrow}_i$ as well as its adjoint $\bopd_i=S^+_i/\hbar$.
It is easy to see that they satisfy the commutation relation $\smash{[\bop_i, \bopd_j]}=(1 - 2 \nop_i) \delta_{ij}$, where $\nop_i = \smash{\bopd_i \bop_i} = S^z_i/\hbar + 1/2$.
For $i\ne j$, these are just bosonic commutation relations.
However, on a single site, we find $\bop_i \bop_i = \smash{\bopd_i \bopd_i} = 0$ and $\smash{\{\bop_i, \bopd_i\}=1}$.
When interpreting $\bopd_i$ as the creation of a single particle (excitation) at site $i$, these equations formalize the so-called hard core constraint: only a single excitation can be present at each site.
Keeping the constraint in mind, we can treat these operators as bosonic creation and annihilation operators for single excitations above the vacuum $\ket{G}$ with $\bop_i\ket{G}=0$ and write the model as
\begin{align}
    H &= J \sum_{i\ne j}  \frac{a^3}{R_{ij}^3} \bc{ \cos\theta\, \nop_i \nop_j  + \frac{1}{2}\sin\theta \bb{\bopd_i\bop_j + \bop_i\bopd_j} }.
\end{align}
For a detailed treatment, see \cref{anomalous_behavior} and \cref{spinwave_analysis}. In the following, we will extend this idea to dipoles with more than two internal states.

\subsection{Three-level dipoles: spin-orbit coupling}
\tlabel{threelevel}
First, we investigate a \textsf{V}-type level scheme including three internal states of a dipole, $\ketz, \ketp$ and $\ketm$, as shown in~\tref{dipole-moments}. The relevant dipole matrix elements are
\begin{align} \tlabel{dipolemomentspm}
    d_0 = \smallbraketop{0}{d^0}{0}, \qquad
    d_\pm = \smallbraketop{\pm}{d^0}{\pm}, \qquad
    q_\pm &= |\smallbraketop{\pm}{d^\pm}{0}|.
\end{align}
As before, we define a vacuum state $\ket{G}=\prod_i \ketz_i$ as well as hardcore bosonic operators $\bop_\pm=\ketbra{0}{{\pm}}$ and $\nop_{\pm}=\bopd_{\pm}\bop_{\pm}$. We can either think of two different kinds of bosons (`$+$' excitations and `$-$' excitations) or think of a single boson with an internal spin degree of freedom. It is useful to write the spherical components of the dipole operator in terms of the bosonic operators:
\begin{align}
    d^0 &= d_+ \nop_+ + d_- \nop_- + d_0 (1-\nop_+-\nop_-) = d_0 + (d_+-d_0) \nop_+ + (d_--d_0) \nop_-, \\
    d^+ &= q_+ \bopd_+ - q_- \bop_-, \\
    d^- &= - q_+ \bop_+  + q_- \bopd_-.
\end{align}
Restricting ourselves to a two-dimensional geometry with a perpendicular polarization, we can express the relevant parts of the dipole-dipole interaction from \tref{dop2d} as
\begin{align}
    d^0_i d^0_j &= d_0^2 + d_0(d_+-d_0) (\nop_{+,i}+\nop_{+,j}) + d_0(d_--d_0) (\nop_{-,i}+\nop_{-,j}) \\
                &\quad + (d_+ - d_0)^2 \nop_{+,i} \nop_{+,j} + (d_- - d_0)^2 \nop_{-,i} \nop_{-,j} \\
                &\quad + (d_+ - d_0)(d_- - d_0) (\nop_{+,i} \nop_{-,j} + \nop_{-,i} \nop_{+,j}) \\
    d^+_i d^-_j &= - q_+^2 \bopd_{+,i} \bop_{+,j} - q_-^2 \bopd_{-,j} \bop_{-,i} \\
    d^+_i d^+_j &= - q_+q_- \, \bopd_{+,i} \bop_{-,j}
\end{align}
where we have neglected any excitation-non conserving terms. Further dropping the constant terms and \qu{magnetic field} terms, we find the many-body Hamiltonian
\begin{align}
    H = \frac{1}{2} \sum_{i\ne j} \Hdd_{ij} = \sum_{i\ne j}  t_{ij}^{\alpha\beta} \bopd_{\alpha,i} \bop_{\beta,j}
    +\frac{1}{2}\sum_{i\ne j} V_{ij}^{\alpha\beta} \nop_{\alpha,i} \nop_{\beta,j}.
\end{align}
Here, a summation over the $\alpha, \beta$ indices, which label the internal state of the excitation, is assumed. This is a generalized (hard core) Bose-Hubbard Hamiltonian, including long-range hopping terms and long-range density-density interactions. In our case, the tunneling rates and interaction matrix elements are given by
\begin{align}
    t_{ij} &= \frac{\kappa}{R_{ij}^3} \begin{pmatrix}
    -\frac{q_+^2}{2} & \frac{3q_+q_-}{2} \ef{-2i \phi_{ij}} \\
    \frac{3q_+q_-}{2} \ef{+2i \phi_{ij}} & -\frac{q_-^2}{2}
\end{pmatrix}, \\
    V_{ij}^{\alpha\beta} &= \frac{\kappa}{R_{ij}^3} (d_\alpha-d_0)(d_\beta-d_0).
\end{align}
By introducing the lattice spacing $a$, we can define the nearest-neighbor tunneling rates
\begin{align} \tlabel{nnrates}
    t^+ = \frac{\kappa q_{+}^2}{2a^3},\qquad
    t^- = \frac{\kappa q_{-}^2}{2a^3},\qquad
    w = \frac{3 \kappa q_{+}q_{-}}{2a^3}.
\end{align}
Using these, the single-particle tunneling part of the Hamiltonian can be written as
\begin{align} \tlabel{pmmodel}
    H_\text{single} = \sum_{i\ne j} \frac{a^3}{R_{ij}^3}
    \begin{pmatrix}
        \bop_{i,+} \\
        \bop_{i,-}
    \end{pmatrix}^{\!\dag}
    \begin{pmatrix}
        -t^+ & w \ef{-2i \phi_{ij}} \\
        w \ef{2i \phi_{ij}} & -t^- \\
    \end{pmatrix}
    \begin{pmatrix}
        \bop_{j,+} \\
        \bop_{j,-}
    \end{pmatrix}^{\vphantom\dag}.
\end{align}
This model is the basis for the realization of topological band structures in \cref{topological_bands}. The off-diagonal tunneling elements are a manifestation of the spin-orbit coupling which is present in these dipolar spin models.



\paragraph{One-dimensional system:}
In one dimension, we can always choose $\phi_{ij}=0$. Furthermore, if $q_+=q_-$, the tunneling
elements simplify to
\begin{align}
    t_{ij} &= \frac{\kappa q^2}{2R_{ij}^3} (3\sigma_x-\id).
\end{align}
By transforming to $\ket{x}=(\ketp-\ketm)/\sqrt{2}$ and $\ket{y}=(\ketp+\ketm)/\sqrt{2}$, the
tunneling part can be diagonalized:
\begin{align}
    t_{ij} &= \frac{\kappa q^2}{R_{ij}^3} \begin{pmatrix}
    -2 & 0 \\
    0 & 1
    \end{pmatrix}.
\end{align}
If the energy of the $\ket{x}, \ket{y}$ manifold is low enough compared to the $\ketz$ state, this
causes excitations to condense in the $\ket{x}$ state, building a ferroelectric state of matter. The accompanying quantum phase transition has been investigated in \refscite{Klinsmann2011,Klinsmann2014}.

\subsection{General case in a tilted external field}
\tlabel{generalTilted}
Finally, we briefly discuss the most general case when all four states in \tref{dipole-moments} are involved. We assume the geometry from \tref{lattice-geometry} with a possibly tilted external field.
Introducing a bosonic operator $\bop_1=\ketbra{1}{0}$ for the additional state, we can write the tunneling rates in the basis $\{{+}, 1, {-}\}$:
\begin{align}
    t_{ij} &= \frac{\kappa}{R_{ij}^3} \begin{pmatrix}
    -\frac{q_+^2}{2} f_0 & \frac{3q_+ q_1}{\sqrt{2}} f_1^* & \frac{3q_+q_-}{2} f_2^* \\
    \frac{3q_+q_1}{\sqrt{2}} f_1 & q_1^2 f_0 & -\frac{3q_1q_-}{\sqrt{2}} f_1^* \\
    \frac{3q_+q_-}{2} f_2 & -\frac{3q_1q_-}{\sqrt{2}} f_1 & -\frac{q_-^2}{2} f_0
\end{pmatrix}.
\end{align}
Here, $f_m = f_m(\Theta_0, \phi_{ij}-\Phi_0)$ are the functions defined in \tref{fmfunctions}.
% For the tunneling rates we find
% \begin{align}
%     t^{\pm \pm}_{ij} &= -\frac{\kappa q_{\pm}^2}{2R_{ij}^3} f_0(\Theta, \phi_{ij} - \Phi), \\
%     t^{11}_{ij} &= \frac{\kappa q_{1}^2}{R_{ij}^3} f_0(\Theta, \phi_{ij} - \Phi), \\
%     t^{1+}_{ij} &= \frac{3\kappa q_{1}q_{+}}{\sqrt{2}R_{ij}^3} f_1(\Theta, \phi_{ij} - \Phi), \\
%     t^{1-}_{ij} &= -\frac{3\kappa q_{1}q_{-}}{\sqrt{2}R_{ij}^3} f_1^*(\Theta, \phi_{ij} - \Phi) \\
%     t^{-+}_{ij} &= \frac{3\kappa q_{+}q_{-}}{2 R_{ij}^3} f_2(\Theta, \phi_{ij} - \Phi)
% \end{align}
Note that $f_1(\Theta_0, \phi_{ij}-\Phi_0)=0$ for an external field which is perpendicular to the two-dimensional plane ($\Theta=0$). This leads to an effective decoupling of the $\keto$ state from the other two, taking us back to the model in \tref{pmmodel}.

% \paragraph{Gamma scheme:}
% Using the $\ketp$ and the $\keto$ state (first excited $m=0$ state), the microwave field is no longer necessary, as the model intrinsically breaks time-reversal symmetry. However, the electric field has to be rotated away from the $z$ axis to open a gap in the spectrum. Let $\Theta, \Phi$ denote the angles of the electric field axis in a spherical coordinate system with the lattice in the equatorial plane.

% We assume a generic level structure of the kind shown in \tref{dipole-moments} with a ground state called $\ketz$ and three excited states $\keto, \ketpm$. The only requirement on these states is that
% \begin{align}
%     \braketop{m+\delta}{d_\Delta}{m} \ll d_0 \text{ for } \delta \ne \Delta
% \end{align}
% where $d_0, d_\pm$ are the dipole operators, defined with respect to the external field axis.
% In particular, this requirement will be fulfilled for arbitrary electric field strengths. However, this also includes microwave dressed states, even with circularly polarized light, coupling states with different $m$.
% Then, using \refandpage{ds:tensortilted} we can write the tunneling part as
% \begin{align}
%     H = \sum_{i\ne j} \frac{\kappa}{\absv{\vecR_{ij}}^3}
%     \begin{pmatrix}
%         \bop_{j,+} \\
%         \bop_{j,1} \\
%         \bop_{j,-}
%     \end{pmatrix}^{\dag}
%     \begingroup
%         \renewcommand*{\arraystretch}{1.5}
%         \begin{pmatrix}
%             -\frac{d_+^2}{2} f_0 & -\frac{3d_0d_+}{\sqrt{2}} f_1^* & \frac{3d_+d_-}{2} f_2^* \\
%             \cdot & \frac{d_0^2}{2} f_0 & -\frac{3d_0d_-}{\sqrt{2}} f_1^* \\
%             \cdot & \cdot & -\frac{d_-^2}{2} f_0 \\
%         \end{pmatrix}
%     \endgroup
%     \begin{pmatrix}
%         \bop_{i,+} \\
%         \bop_{i,1} \\
%         \bop_{i,-}
%     \end{pmatrix}^{\vphantom\dag}
% \end{align}
% where the functions $f_m = f_m(\Theta_0, \bar\Phi) = f_m(\Theta_0, \phi_{ij} - \Phi_0)$ depend on the external field direction and the relative angle $\phi_{ij}$ between the dipoles, see \cref{ds:fmfunctions}.

% To simplify the notation, we introduce the expressions
% \begin{align}
%     t^\alpha &= \frac{\kappa d_\alpha^2}{2a^3} \\
%     w &= \frac{3\kappa d^+d^-}{2a^3}
% \end{align}

% To derive the parameters of the bosonic Hamiltonian, we introduce the vacuum state $\ket{\text{vac}}=\prod_k \ketz_k$ and the single particle states $\ket{\Psi_{i,\pm}}=\bopd_{i,\pm} \ket{\text{vac}}$.
% Then, the hopping amplitudes are given by
% \begin{align}
    % t^{\alpha\beta}_{ij} &= \smallbraketop{\Psi_{i,\alpha}}{H^\text{dd}}{\Psi_{j,\beta}}.
% \end{align}
% We set the spin-conserving term $t^{\pm\pm}_{ij}=-t^\pm \cdot a^3/|\vec{R}_{ij}|^3$ and the spin-flip tunneling $t^{-+}_{ij}=w \ef{2i \phi_{ij}} \cdot a^3/|\vec{R}_{ij}|^3$ to get the final expressions for the nearest-neighbor tunneling rates
