\chapter[Driving dipolar fermions into the Quantum Hall regime]{Driving dipolar fermions\texorpdfstring{\\}{ }into the Quantum Hall regime}
\chaptermark{Driving dipolar fermions into the Quantum Hall regime}
\label{dipolar_fermions}


% A new method to drive a system of neutral dipolar fermions into the lowest Landau level regime is proposed.
% By employing adiabatic spin-flip processes in combination with a diabatic transfer, the fermions are pumped to higher orbital angular momentum states in a repeated scheme that allows for the precise control over the final angular momentum.
% A simple analytical model is derived to quantify the transfer and compare the approach to rapidly rotating systems.
% Numerical simulations of the transfer process have been performed for small, interacting systems.


Despite being ideal models for complicated solid state systems, ultracold quantum gases lack one important aspect of the electronic complex: because of the charge neutrality of the atoms, there are no mobile charge carriers that possess a direct coupling to the magnetic vector potential. Plenty of interesting effects, however, arise when charged particles are subject to high magnetic fields in low dimensional systems. The most prominent ones are the integer quantum Hall effect \cite{Klitzing1980} as an example for the appearance of topological states, as well as the fractional quantum Hall effect \cite{Laughlin1983}, potentially giving rise to fundamental excitations with non-Abelian statistics.

Several schemes have been proposed to simulate the effect of magnetic fields for neutral particles. Artificial gauge fields can be created by imprinting phases, making use of the Peierls substitution in optical lattices \cite{Aidelsburger2011,Jimenez-Garcia2012,Struck2012}, or by tailoring spatially dependent Hamiltonians to generate geometric phases \cite{Lin2009}, for an overview see \cite{Dalibard2011}. Rapidly rotating quantum gases provide an alternative route via Larmor's theorem, using the analogy between a charged particle in a constant magnetic field and a neutral particle in a rotating frame
\cite{Cooper2008,Fetter2009}. Several theoretical proposals demonstrate the appearance of highly correlated quantum Hall states for dipolar bosons \cite{Cooper2005} and fermions \cite{Baranov2005,Osterloh2007}. However, the experimental realization of quantum Hall states has been elusive so far. For rotating systems, the main problem is the precise control on the rotation frequency, which is required to reach the lowest Landau level without crossing the rotational instability \cite{Schweikhard2004}.

Here we propose a new scheme to access the regime of fast rotation for a dipolar Fermi gas such as $^{161}\text{Dy}$ or $^{167}\text{Er}$, where quantum degeneracy has been demonstrated recently~\cite{Lu2012,Aikawa2014}. Starting from a spin-polarized state, dipolar interactions can lead to spin relaxation with a net angular momentum transfer \cite{Hensler2003}. This is known as the Einstein--de Haas effect \cite{Einstein1915} and has been proposed to create rotating Bose-Einstein condensates \cite{Santos2006,Kawaguchi2006}. We suggest to use this mechanism in a trapped, quasi-two-dimensional system to control the amount of angular momentum, and -- by repeated application of the transfer scheme -- reach the lowest Landau level (LLL). This scheme allows for direct control over the total angular momentum instead of the rotation frequency and circumvents the prime experimental difficulties toward the realization of the quantum Hall regime in harmonically trapped gases.



% The outline of the paper is as follows. We will first describe the basic mechanism of dipolar relaxation and how it is applied in the proposed transfer scheme. We show that this scheme is able to drive a dipolar Fermi gas to the lowest Landau level regime and derive a simple analytical model to describe the adiabatic transfer. For small system sizes we explicitly simulate the process, including all interactions and demonstrate that the adiabaticity assumption can be fulfilled. Finally, we comment on the different ground states which are accessible in the Quantum Hall regime.



\section{Setup}
\fig{.4}{fig1a}{fig1a}{Dipolar particles, trapped in a quasi-2D geometry with a radial confinement $\omega$. When the external magnetic field $\vec{B}$ is tuned in resonance, dipolar interactions \Vdd can induce spin relaxation processes, leading to a net angular momentum increase of $1\hbar$ per particle.}

\noindent
We consider a system of $N$ fermionic atoms with magnetic dipole moments $\vecd$. While extensions to schemes with polar molecules are possible, the permanent dipole moments of the atoms lead to some simplifications.
%Highly magnetic atoms typically possess a large hyperfine manifold.
To shorten the discussion, we consider only two internal levels (pseudospin 1/2). The particles are confined in a quasi-two-dimensional harmonic trap with a radial frequency $\omega$ and an axial frequency $\omega_z$. For strong $z$ confinement $\hbar\omega_z \gg \Ef$, where $\Ef$ is the Fermi energy derived below, the system is effectively two dimensional, see~\tref{fig1a}.
The interactions between the particles are described by the dipolar interaction potential
\begin{align} \tlabel{dfvdd}
\Vdd(\vecR) = \frac{\kappa}{\absv{\vecR}^3} \bc{ \vecd_i \cdot \vecd_j -3 (\vecd_i \cdot \hat{\vecR})(\vecd_j \cdot \hat{\vecR}) }
\end{align}
as defined in \cref{ds:ddint}, where $\vecR=\vecR_i-\vecR_j$ is the relative distance between the two particles.
Note that a weak $s$-wave scattering length does not change the general behavior of our transfer scheme and is ignored in the following.
For two internal levels, we can write the
dipole moment $\vecd =  \mu_\text{B}g \vec{S} / \hbar = \mu_\text{B}g \vecsigma/2$ in terms of the Land\'e factor $g$ and the Pauli matrices $\vecsigma$.

In two dimensions, using the spin raising and lowering operators $\sigma^{\pm} = (\sigma^x \pm i \sigma^y)/2$, the interaction between these two-level dipoles reduces to
\begin{align}
\Vdd(R,\phi) =  \frac{\Cdd}{R^3} \left[
\sigma^z_i \sigma^z_j - (\sigma^+_i \sigma^-_j +3 \ef{2i\phi} \sigma^-_i \sigma^-_j  +\hc)
\right],
\end{align}
which is the expression given in \refandpage{ds:hdd2dmagnetic}. Here, $R$ and $\phi$ are polar coordinates in the $xy$ plane and $\Cdd = \kappa \mu_\text{B}^2 g^2 / 4$ characterizes the strength of the interaction.
Note that this expression can also be derived from the full interaction \labeltref{dfvdd} by integrating out the fast motion perpendicular to the $xy$ plane in the limit $\omega_z\rightarrow\infty$.

The dipolar interaction features three different processes. The first term proportional to $\sigma^z_i\sigma^z_j$ describes spin-preserving collisions, while the second term \smash{$\sigma^+_i\sigma^-_j$} accounts for spin-exchange collisions. These terms conserve separately the total spin and the total angular momentum.  Finally, the third operator \smash{$\ef{2i\phi} \sigma^-_i \sigma^-_j$} describes the relaxation process that transfers spin to orbital angular momentum, see \tref{fig1b},\subref*{df:fig1c}. The sum $L+S$ is still conserved and the spin flip leads to an orbital motion with an increase of relative angular momentum of $2\hbar$.

\begin{figure}[t]
    \centering
    \subfloat[]{
        \ig{0.31}{fig1b}
        \tlabel{fig1b}
    }
    \subfloat[]{
        \ig{0.31}{fig1c}
        \tlabel{fig1c}
    }
    \subfloat[]{
        \ig{0.31}{fig1d}
        \tlabel{fig1d}
    }
    \caption{\tlabel{fig1}(a) Energy levels of a 2D harmonic oscillator. (b) One of the possible spin-flip processes, bringing both particles to higher angular momentum states. (c) Eventually, after repeated application of the driving scheme, all particles occupy the lowest Landau level.}
\end{figure}

It is this process that allows us to drive the dipolar particles to higher angular momentum states. Assuming the gas is initially in a spin-polarized state with the external magnetic field pointing in the positive $z$ direction, the particles will undergo spin relaxation when the field is adiabatically ramped through zero and finally pointing in the negative $z$ direction. During this adiabatic ramping,
the total orbital angular momentum is increased by $N\hbar$ with $N$ the number of particles in the system.  For the goal to reach the lowest Landau level regime, it is required to transfer  $\liqhe\equiv N(N-1)/2 \cdot \hbar$ angular momentum to the orbital degrees of freedom, as described below. It is therefore necessary to reverse the magnetic field and the spins to their original position, in a way that guarantees repeated application of the transfer scheme without affecting the orbital angular momentum.

To achieve this, we propose rotating the magnetic field by $180^\circ$ around an arbitrary axis lying in the $xy$ plane (say, the $y$ axis), slowly enough such that the spins rotate adiabatically, but fast enough such that the orbital degrees of freedom cannot follow. To satisfy the adiabaticity with respect to the spins and diabaticity with respect to the external degrees of freedom, the speed of the rotation $\gamma_\text{rot}$ has to satisfy
$\omega \ll \gamma_\text{rot} \ll \omega_\text{L}$, where $\omega_\text{L}=g\mu_B B/\hbar$ is the Larmor frequency. After the rotation, the magnetic field has enclosed a \textsf{D}-shaped path in the $xz$ plane. The spins are now pointing upward (in analogy to \tref{fig1b} but with
increased angular momentum) and the transfer scheme can be applied again. Multiple repetitions are realistic and only limited by the finite lifetime of the trapped ensemble.

High angular momentum states are indeed related to the quantum Hall regime, as there is a close connection between the Landau levels and the states $\ket{n,m}$ of a two-dimensional harmonic oscillator in terms of a radial quantum number $n=0,1,\dots$ and angular momentum $\hbar m$, see \tref{fig1b}. In particular, the ground state of $N$ fermions filled into the harmonic oscillator with the constraint $L=\liqhe$ is given by the many-body state
\begin{align}
\Psi&= \bra{\{z_i\}}\mathcal{A} \prod_{m=0}^{N-1} \ket{0,m} = \mathcal{N}\Bigg[\prod_{i<j} (z_i-z_j) \Bigg] \ef{-\frac{1}{2}\sum\absvsq{z_k}}.
\end{align}
Here $z_k=(X_k+i Y_k)/l_\text{HO}$ are complex coordinates of the particles, $\mathcal{A}$ is the antisymmetrizer, $\mathcal{N}$ is a normalization constant, and \smash{$l_\text{HO}=\sqrt{\hbar/m\omega}$} is the harmonic oscillator length. This wave function is equivalent to the Laughlin wave function for integer filling $\nu=1$, with   \smash{$l_\text{HO}=\sqrt{\hbar/m\omega}$} replacing the magnetic length \smash{$\sqrt{2}l_m=\sqrt{2\hbar c/eB}$} for electronic systems. Quite generally, the states with $n=0$ and $m\ge 0$ correspond to the states in the lowest Landau level, see \tref{fig1d}.
%The integer QHE state has a total angular momentum of $L=\sum \hbar m=\liqhe$.
To reach the LLL regime, we have to repeat the transfer scheme at least $\liqhe/N\hbar=(N-1)/2$ times.

\section{Modelling the transfer process}

To quantify a single transfer process, our first aim is to calculate the total energy of $N$ harmonically trapped fermions for a fixed total angular momentum $L$ (polarized state, one spin component). For the noninteracting system, the energy can be obtained by simple summations. We start with the ground state for $L=0$, where all energy shells up to the Fermi energy are completely filled. The energy of the single particle states~$\ket{n,m}$ is given by $E_{nm}=\hbar\omega(2n+\absv{m}+1)$. To avoid cluttering of notation, we introduce dimensionless quantities indicated by a $\dl{\ }$ sign. These quantities are measured in oscillatory units. That is, energy in units of $\hbar\omega$, angular momentum in units of $\hbar$, lengths in units of $l_\text{HO}$ and time in units of $\omega^{-1}$. The degeneracy of each energy level is simply given by $g(\dl{E}) = \dl{E}$.
With $N=\sum g(\dl{E}) = \dlEf(\dlEf+1)/2$ the Fermi energy is determined by
\begin{align}
\dlEf &= \frac{1}{2}\bb{\sqrt{8N+1}-1} \xrightarrow{N\gg 1} \sqrt{2N}
\end{align}
The total energy for $N$ particles is then given by
\begin{align} \tlabel{energyzero}
\dl{E}(N) = \sum_{\dl{E}=1}^{\dlEf} g(\dl{E}) \dl{E} = \frac{N}{3} \sqrt{8N+1} \xrightarrow{N\gg 1} \frac{(2N)^{3/2}}{3}
\end{align}
which shows the known scaling of a trapped 2D Fermi gas~\cite{Yoshimoto2003}.
Note that the energy for the unpolarized system is given by $2\,\dl{E}(N/2)=\frac{2}{3} N^{3/2}$.
To derive the total energy $E(N,L)$ for $L\ne 0$, we define $N_m$ as the number of particles with angular momentum $m$. The energy in terms of $N_m$ is given by
\begin{align} \tlabel{energyf}
\dl{E} = \sum_m \sum_{n=0}^{N_m-1} \dl{E}_{nm} = \sum_m N_m\left(N_m + \absv{m} \right).
\end{align}

\begin{figure}[t]
    \centering
    \ig{0.5}{fig2}
    \caption{\tlabel{fig2} Exact ground state energy (dots) for $N=10$ particles at fixed angular momentum $L$, compared to the approximate expression (solid line) as given in~\tref{etotnl}. For $L>\liqhe=45\hbar$, the energy increases linearly. Inset shows $L$ as a function of the rotation frequency $\Omega$ in the analytic model. $L$ diverges at the critical rotation frequency $\Omega=\omega$, when the rotation exceeds the trap frequency.}
\end{figure}

\noindent
The exact ground state energy can be found combinatorially for small particle numbers by varying the $N_m$ for fixed $N$ and $L$. The result for $N=10$ is shown in~\tref{fig2}. For larger particle numbers this method is not feasible, but an analytic solution can be found for large particle numbers. Then, we can treat $N_{m}$ as a continuous function. To find the minimum of \tref{energyf} at fixed $N$ and $L$, we introduce two Lagrange multipliers $\mu$, $\Omega$ for the conditions $N = \sum_m N_m$ and $\dl{L} = \sum_m N_m m$, respectively. Taking the functional derivative with respect to $N_m$ yields
%\begin{align}
$N_m=(\dl{\mu}-\absv{m}+\dl{\Omega} m)/2$.
%\end{align}
The parameters can be determined by solving the constraints and summing from $m_-=-\dl{\mu}/(1+\dl{\Omega})$ to $m_+=\dl{\mu}/(1-\dl{\Omega})$, where $N_{m_\pm}=0$. One finds
\begin{align}
\dl{\Omega} = \frac{3 \dl{L}}{\sqrt{\big(2N\big)^3 + \big(3\dl{L}\big)^2}}, \qquad \dl{\mu}= \frac{N^2}{\sqrt{\big(2N\big)^3 + \big(3\dl{L}\big)^2}}.
\end{align}
%\begin{align}
%\Omega = \left[1+\frac{(2N)^3}{(3L)^2}\right]^{-1/2}, \qquad N_0= \sqrt{\frac{N}{2}}\left[1+\frac{(3L)^2}{(2N)^3}\right]^{-1/2}.
%\end{align}
By using these relations and omitting correction terms of order $1/L$ and $\sqrt{N}$, we obtain the total energy
\begin{align} \tlabel{etotnl}
\dl{E}(N,\dl{L})= \frac{1}{3} \sqrt{\big(2N\big)^3 + \big(3\dl{L}\big)^2}.
\end{align}
This result agrees with the exact behavior for $L=0$ as derived in~\tref{energyzero}, and even for particle numbers as small as $N=10$ it is close to the exact ground state energy, as shown in \tref{fig2}. For $L\ge\liqhe$, the minimization problem becomes trivial as all particles occupy the lowest Landau level. The energy is exactly given by $\dl{E}=\dl{L}$, which is also obtained asymptotically from~\tref{etotnl} in the limit $\dl{L}\gg N$.

It is now possible to quantify the link between our approach (fixed angular momentum) and rapidly rotating systems (fixed rotation frequency) explicitly. Both are connected by a Legendre transform and we should in fact interpret the Lagrange multiplier $\Omega=\frac{\partial E}{\partial L}$
as the rotation frequency. In a harmonic trap, the system becomes unstable if $\Omega$ exceeds the value of the trap frequency $\omega$, as the harmonic confinement in the rotating frame is effectively given by $\omega-\Omega$. The angular momentum
\begin{align}
\dl{L}=\frac{(2N)^{3/2}}{3} \frac{\dl{\Omega}}{\sqrt{1-\dl{\Omega}^2}}
\end{align}
has a singularity for $\dl{\Omega}=\Omega/\omega=1$ and large values of $L$ can only be achieved by tuning $\Omega$ close to the critical value. It is this precise control on the rotation frequency that so far prevented the experimental realization of the quantum Hall regime in harmonically trapped gases. In contrast, for the present situation, the system is always stable as $\Omega(L)<\omega$ for all $L$. An arbitrary orbital angular momentum can be transferred to the system by the ramping scheme with high precision.


%In turn, the value of $\Omega(N,\liqhe)/\omega = 1-\frac{16}{9N}+\mathcal{O}(N^{-2})$ which is %needed to reach the QHE regime, quickly converges to 1. In our setup, the rotation frequency $%\Omega(L)$ is inferred from the given angular momentum. In the limit $L\rightarrow \infty$, it %approaches the critical value, but never exceeds it\todo{reference subfig?}. \todo{Note that for $L%\ge\liqhe$, the energy is exactly given by $E=L$ and the frequency is therefore fixed at $%%\Omega=1$.}

\begin{figure}[t]
    \centering
    \ig{0.5}{fig3a}
    \caption{\tlabel{fig3a}Description of the transfer in the analytical model with $N=100$ particles for decreasing energy splitting $\Delta$ between the two components $\up$ and $\down$. The transfer starts at $\Delta=\Ef$ with particles continuously being transferred into the $\down$ state as $\Delta$ is lowered to $-\Ef$. Notice that during the transfer, both components rotate in the same direction. The crossing $N_\up=N_\down$ is not precisely at $\Delta=0$ due to the initial bias.}
\end{figure}

Starting from expression \labeltref{etotnl} for the energy, we are now able to describe the transfer process in the adiabatic limit. Let $N_\up$ be the number of particles in the spin-up state and $N_\down = N-N_\up$ the particles in the spin-down state. We describe both components separately and write the total energy as
%\begin{align} \tlabel{minimize}
$E(N_\up,L_\up) + E(N_\down,L_\down) + \Delta\cdot N_\down$
%\end{align}
where we have introduced the Zeeman energy shift $\Delta=\mu_\text{B} g B$ (energy measured with respect to the energy of the lower Zeeman state). We assume that every particle eventually takes part in the transfer process (adiabaticity) and consequently one quanta of angular momentum is transferred per particle. Starting from the nonrotating state at $L=0$, this imposes the transfer condition $L_\up + L_\down = L = N_\down\hbar$. Adding this condition with another Lagrange multiplier, one can quantify the transfer process as a function of $\Delta$, see \tref{fig3a}. Coming from high fields where $\Delta > \Ef$, the transfer starts right at the Fermi energy. Note that during the transfer, while $\Ef > \Delta > -\Ef$, both components ($\up$, $\down$) rotate in the same direction. Eventually all particles get transferred to the lower spin state and the total angular momentum equals $L=L_\down=N\hbar$.

\section{Numerical simulation}
\begin{figure}[t]
    \centering
    \subfloat[]{
        \ig{0.4}{fig3b}
        \tlabel{fig3b}
    }
    \subfloat[]{
        \ig{0.4}{fig3c}
        \tlabel{fig3c}
    }
    \caption{\tlabel{fig3}(a) Full simulation of the transfer scheme for $N=4,6$, and $8$~particles in the adiabatic limit $\gamma\rightarrow 0$. As the Zeeman splitting $\Delta$ is tuned through zero, the angular momentum increases in steps of $2\hbar$, indicating the transfer of two particle at a time. The interaction strength is given by $\dlCdd=0.1$. (b) Angular momentum at the end of the transfer for $N=6$ particles at different values of the Landau-Zener parameter $\lambda=\dlCdd^2/\dl{\gamma}$. The data points for different rates collapse onto a single curve. The solid line is a probabilistic model, fitted to the data points.}
\end{figure}

To justify the adiabaticity assumption above, we simulate the transfer process for small systems of few particles. We include all interactions mediated by $\Vdd(R,\phi)$, and assume, that the strength of the interaction $\dlCdd=(\Cdd/l_\text{HO}^3)/\hbar\omega \ll 1$ is weak compared to the Landau level splitting. Then, only a few excited states have to be taken into account.
% -- cut --
%We start by choosing an appropriate set of single-particle states $\{\ket{n,m,\sigma}\}$ with quantum numbers $n,m$ and spin $\sigma$, spanning a subspace $\mathcal{S}$ of the single-particle Hilbert space $\mathcal{H}$. The selection of states depends on the particle number and the initial angular momentum. Here, we assume that we start from a spin-polarized (multi-particle) state with $S=N \hbar/2$ and $L=0$. Out of the $\binom{\dim \mathcal{S}}{N}$ possible Fock states, only a reduced subset couples to the initial state. We use angular momentum conservation $L+S=N\hbar/2$ and the selection rule ${S/\hbar\equiv N/2 \pmod 2}$ due to pair-wise spin flips, to reduce the number of multi-particle states.
% -- cut --
The system dynamics is described by
\begin{align}
H=\sum_{i} \left[ E_{nm}  + \Delta(t)\,\delta_{\sigma,\down} \right] \copd_i\cop_i +  \frac{1}{2}\sum_{ijkl} V_{ijkl}\, \copd_i \copd_j \cop_l \cop_k
\end{align}
where each of the indices $ijkl$ of the fermionic operators labels a set of quantum numbers $(n,m,\sigma)$ and $\Delta(t)/\hbar\omega = -\gamma t$ is the time-dependent Zeeman shift, controlled by the linearly decreasing magnetic field. The calculation of the dipolar interaction matrix elements $V_{ijkl}\sim\Cdd$ is presented in~\cref{matrix_elements}. The only relevant parameters in this model are the transfer rate  $\dl{\gamma}=\gamma/\omega$ and the interaction strength $\dlCdd$.
For the perfect adiabatic transfer, in the limit $\gamma\rightarrow 0$, we can find the instantaneous ground state of $H$ as $\Delta$ decreases. The results are shown in \tref{fig3b} for $N=4,6$, and $8$ particles. The total angular momentum $L(\Delta)$ increases gradually from $L=0$ to $L=N\hbar$ in steps of $2\hbar$, indicating that two particles are transferred at a time.

To obtain results for a finite transfer rate $\gamma$, we simulate the full time-dependent many-body problem.
The total angular momentum $L(t\rightarrow\infty)$ at the end of the transfer for $N=6$ particles is shown in \tref{fig3c} for different values of $\dlCdd$ and $\dl{\gamma}$.  Remarkably, the data points collapse onto a single line using  $\lambda = \dlCdd^2/\dl{\gamma}$. This parameter arises in the Landau-Zener formula of a single level crossing, and the collapse indicates that
each pair transfer is dominated by an individual avoided level crossing.

\subsection{Probabilistic Landau-Zender model}
\tlabel{probab}
We can derive a very simple model that accounts for this behavior and describes the final angular momentum observed in the full simulation (see solid line).
To derive the total amount of angular momentum after the transfer we assume that each 2-particle process is described by an independent Landau-Zener (avoided) crossing, neglecting any interference effects. For each Landau-Zener process, we define the probability to transfer the $n$-th pair of particles by $P_n = 1-\ef{-\lambda/\lambda_n}$ with $\lambda=\dlCdd^2/\dl{\gamma}$ the Landau-Zener parameter and $\lambda_n$ an effective coupling strength, describing the $n$-th pair-transfer process. The total angular momentum for $N$ particles after one cycle is then given by weighting each possible outcome ($\dl{L}=0,\dl{L}=2,\dots,\dl{L}=N$) by the respective probability
\begin{align}
\dl{L}_N&=\sum_{n=1}^{N/2-1} 2n \,P_1\cdots P_n (1-P_{n+1})+N P_1\cdots P_{N/2}\\
&=2P_1\bb{1+P_2\bb{1+P_3\bb{1+\dots\bb{1+P_{N/2}}}}}
\end{align}
For $N=6$ particles this reduces to
\begin{align}
\dl{L}_6&=2 P_1(1+P_2(1+P_3))\\
&=2 \big(1-e^{-\lambda/\lambda_1}\big) \big(1+\big(1-e^{-\lambda/\lambda_2}\big) \big(2-e^{-\lambda/\lambda_3}\big)\big)
\end{align}
The assumption of independent crossings can now be justified a-posteriori. By fitting $\dl{L}_6$ to the simulation data we find $\lambda_1=0.0056$, $\lambda_2=0.025$, $\lambda_3=1.74$ with $\lambda_1 \ll \lambda_2 \ll \lambda_3$. While we suspect this approximation to break down for larger $N$, the model describes the transfer for small particle numbers in good agreement with the simulation.

\section{Experimental realization and detection}

The preparation of the integer quantum Hall state with an orbital angular momentum of $\dlliqhe = N(N-1)/2$ is finally achieved by a sequence of ramping cycles: Starting with an unpolarized sample with the fermions equally distributed between the two spin states, i.e.,  $N_\up=N_\down=N/2$, a first transfer increases the orbital angular momentum by only $\dl{L}=N/2$. Then, $N/2-1$ subsequent cycles will transfer exactly the required orbital angular momentum to reach the integer quantum Hall state.

In an experimental realization with $^{161}\text{Dy}$ atoms, the number of cycles can be significantly reduced due to the total spin of $F=21/2$ in the hyperfine ground state. Although calculations for $22$ internal levels are too complex, we expect no qualitative modifications, except that $21\hbar$ of angular momentum are transferred per particle and cycle\footnote{Other highly dipolar fermions used in cold atom experiments are $^{167}\text{Er}$ and $^{53}\text{Cr}$ with a total angular momentum of $F=19/2$ and $9/2$, respectively \cite{Berglund2007,Chicireanu2006}. They could therefore provide $19\hbar$ or $9\hbar$ of angular momentum per atom and transfer.}. Two important experimental requirements are a precise magnetic field control \cite{Pasquiou2011} as well as a deterministic preparation scheme for a certain particle number, as demonstrated in \cite{Serwane2011}. For the magnetic field ramp we can estimate an optimal minimum value for the rate $\dl{\gamma}=2\dlEf/\dl{t}_\text{e}=2\sqrt{2N}/\omega t_\text{e}$ by observing that the Zeeman splitting has to be tuned at least once from $\Ef$ to $-\Ef$ within the experimental accessible time $t_\text{e}$, which is limited by the lifetime of the atoms in the trap. The Landau-Zener parameter is finally given by
\begin{align}
\lambda = \frac{\omega t_\text{e}}{2\sqrt{2N}} \left(\frac{l_\text{DDI}}{4 l_\text{HO}}\right)^2
\end{align}
where the length $l_\text{DDI} = m\kappa d^2/\hbar^2$ parametrizes the strength of the interaction \cite{Lu2012}. In a setup with $N\sim 10$ fermionic $^{161}\text{Dy}$ atoms, a long lifetime of $t_\text{e}=10\text{s}$ and a radial frequency of $\omega = 3\text{kHz}$ are needed to reach values of $\lambda$ on the order of $1$. We comment, however, that the transfer scheme works already for smaller values of $\lambda$.

\fig{.7}{density-comparison}{density-comparison}{Density distribution $n(r)$ for different non-interacting states of $N=55$ particles. The fermionic LLL state at $L=\liqhe$ has a perfectly flat density of $1/\pi l_\text{HO}^2$ in a circular region of radial size $\sqrt{N}l_\text{HO}\approx 7.4l_\text{HO}$ while the $L=0$ ground-state is close to the well-known parabolic Thomas-Fermi distribution. The bosonic ground state is shown for comparison.}

% in a trap of radial frequency $\omega=1\text{kHz}$, the interaction parameter is $\dlCdd\simeq 0.01$.
%The rate $\dl{\gamma}=\gamma/\omega = \mu \dot{B}/\hbar\omega^2$ is determined by the ramping speed $\dot{B}$ of the magnetic field. For a value of $\dot{B}=1\text{mG}/\text{s}$, the Landau-Zener parameter would be on the order of $\lambda=0.01$. It depends, however, on the radial frequency as $\lambda\sim\omega^3$ and is therefore tuneable to higher values in deeper traps.

A particularly interesting property of the integer quantum Hall state, potentially useful to detect the successful generation, is the perfectly flat density $n=1/\pi l_\text{HO}^2$ within a circular region of radial size $\sqrt{N}l_\text{HO}$, shown in~\tref{density-comparison}.

%Consequently, the presented method of dipolar relaxation allows for the exploration of integer and fractional quantum Hall states, but avoids the experimentally challenging requirement of precise control of the rotation frequency by directly tuning the orbital angular momentum.

%\newcommand{\p}{\mathop{\tikz{\filldraw [lightblue] (0,0) circle(0.3em);\draw (0,0) circle(0.3em);}}}
%\newcommand{\h}{\mathop{\tikz{\draw (0,0) circle(0.3em);}}}

%Finally, we discuss the available states in the lowest Landau level. After $k$ transfer cycles the total angular momentum is $\dl{L}=kN$ for even $N$ and $\dl{L}=k(N-1)$ for odd $N$, as only pairs of particles can be transferred. To achieve $\dl{L}=\dlliqhe$, $k=(N-1)/2$ cycles are needed for even $N$ and $k=N/2$ cycles for odd $N$, a half-integer number in both cases. To circumvent this problem, it is possible to start from an unpolarized state with $N_\up=N_\down=N/2$. Assuming $N/2$ is even, the first transfer drives the system to a polarized $\dl{L}=N/2$ state while $N/2-1$ subsequent cycles (including all particles) will bring the system exactly to $\dl{L}=N/2 + (N/2-1)N = \dlliqhe$.


%By simply continuing the transfer scheme\todo{(which might be necessary due to non-perfect transfer or imprecise knowledge of $N$)}, it is possible to reach states with $L>\liqhe$. The non-interacting ground states with angular momentum offset $\Delta L\equiv L-\liqhe$ are highly degenerate \footnote{The degeneracy $g(N,\Delta L)$ is given by the number of integer partitions of $\Delta L$ into at most $N$ integers.}. The dipolar interaction lifts this degeneracy and highly correlated ground states appear as ground states as discussed in \cite{Osterloh2007} for rotating systems. The same analysis applies for our setup, with the difference that it is possible to directly tune to certain angular momentum values instead of obtaining $L$ from a given rotation frequency $\Omega$.


\section{Creation of fractional Quantum Hall states}
In addition, it is possible to reach states with $L>\liqhe$ by continuing the transfer scheme.
%The non-interacting ground states with angular momentum offset $\Delta L\equiv L-\liqhe$ are highly degenerate \footnote{The degeneracy $g(N,\Delta L)$ is given by the number of integer partitions of $\Delta L$ into at most $N$ integers.}.
%The dipolar interaction lifts this degeneracy and
In this regime,
highly correlated ground states appear that are closely connected to fractional quantum Hall states, see~\cite{Osterloh2007} for a discussion in the context of rotating systems.
We consider a situation where we start from an unpolarized state and run the initial half-cycle to the polarized state (transferring $F\cdot N$ angular momentum). The subsequent $k$ cycles transfer $2F \cdot N$ of angular momentum. If we want to generate a FQHE state with $\nu=\sfrac{1}{m}$, the total angular momentum is $m\liqhe$. In total we find the condition
\begin{align}
m\liqhe=m\,\frac{N(N-1)}{2} = k\cdot 2F N + F N.
\end{align}
Solving for the number of particles, we obtain
\begin{align}
N=\frac{2F(2k+1)}{m} + 1
\end{align}
For just one cycle ($k=1$), we find
\begin{align}
N=\frac{3\cdot 2F}{m}+1
\end{align}
Since $F$ is half-integer, we find that $m$ has to be equal to $3$ or one of the prime factors of $2F$.
For  $^{161}\text{Dy}$, $^{167}\text{Er}$ and $^{53}\text{Cr}$ we have $2F=21, 19$ and $9$, respectively. For all of them, the $\nu = \sfrac{1}{3}$ state is accessible (for $N=2F+1$ particles), but in addition, for $^{161}\text{Dy}$ it is possible to generate $\nu=\sfrac{1}{7}$ and $\nu=\sfrac{1}{9}$ states in one cycle (for $10$ resp. $8$ particles).

For one ,,magical`` particle number $N=2F+1$ it is possible to generate all FQHE states with any $\nu=\sfrac{1}{m}$ state within $k=m$ cycles (except for $F=\sfrac{1}{2}$, see below).

\begin{table}[ht]
    \centering
    \begin{tabular}{cccc}
    \toprule
    Species & cycles & particles & obtained state ($\nu$) \\
    \midrule
    $^{161}\text{Dy}$ & 1 & 64 & 1 \\
    & 1 & $22^\star$ & $\sfrac{1}{3}$ \\
    & 1 & 10 & $\sfrac{1}{7}$ \\
    & 1 & 8 & $\sfrac{1}{9}$ \\
    & 2 & $22^\star$ & $\sfrac{1}{5}$ \\
    & 3 & $22^\star$ & $\sfrac{1}{7}$ \\
    \midrule
    $^{167}\text{Er}$ & 1 & 58 & 1 \\
    & 1 & $20^\star$ & $\sfrac{1}{3}$ \\
    & 2 & $20^\star$ & $\sfrac{1}{5}$ \\
    \midrule
    $^{53}\text{Cr}$ & 1 & 28 & 1 \\
    & 1 & $10^\star$ & $\sfrac{1}{3}$ \\
    & 2 & $10^\star$ & $\sfrac{1}{5}$ \\
    \bottomrule
    \end{tabular}
    \caption{Examples of accessible final states with more than $4$ particles. ,,Magical`` particle number $N^\star=2F+1$ is marked.}
\end{table}

We also find, that for $F=\sfrac{1}{2}$, the number of particles has to be $N=\frac{2k+1}{m}+1$. In principle, $2k+1$ can be any odd number, but if $m=2k+1$, we have $N=2$ and the whole scheme breaks down (the initial half-cycle does not work). The lowest non-prime odd number is $9$, which means it is possible to generate a $\nu=\sfrac{1}{3}$ state with $4$ particles in $k=4$ cycles. Any other FQHE state requires $7$ or more cycles.


\subsection{Interacting many-body states}
\todo{Compare dipolar LLL states to Laughlin wave functions, cite Lewenstein papers etc.}
